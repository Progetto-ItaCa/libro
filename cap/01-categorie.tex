\chapter{Categorie, funtori, trasformazioni naturali}
\subsubsection*{Esercizi}
\begin{enumerate}
    \item 
    \item 
    \item 
    \item 
    \item 
\end{enumerate}

\section{Monomorfismi ed epimorfismi}

Un monoide, non essendo necessariamente commutativo, può avere sia elementi cancellabili a sinistra,
sia elementi cancellabili a destra.
Un elemento \(x\) di un monoide \((M, \cdot, 1)\) è cancellabile a sinistra se,
per ogni coppia di elementi \(y_0\) e \(y_1\) in \(M\) tali che \(x \cdot y_0 = x \cdot y_1\),
si ha che \(y_0 = y_1\),
e gli elementi cancellabili a destra sono definiti analogamente.
Siccome i monoidi corrispondono precisamente alle categorie con un solo oggetto (\ref{???}),
la definizione di elementi cancellabili a destra o sinistra si può applicare direttamente ai morfismi di tali categorie.
Estendiamo ora questa definizione ad una categoria generica,
dove dunque possono esserci più oggetti e, di conseguenza, la composizione è un'operazione parziale.

\begin{definition}[Monomorfismo]
	Un morfismo \(m \colon B \to X\) in una categoria \(\ctC\) è un \emph{monomorfismo} (o \emph{mono}) se,
	per ogni coppia di frecce parallele \(f, g \colon A \to B\) in \(\ctC\) tali che \(m \cmp f = m \cmp g\), si ha che \(f = g\).
	%Equivalentemente, \(m\) è un monomorfismo se è cancellabile a sinistra, ovvero \(m \cmp f = m \cmp g\) implica \(f = g\) ogni qual volta la scrittura ha senso (ovvero quando \(f\) e \(g\) sono frecce parallele e pre-componibili con \(m\)).
\end{definition}

\begin{definition}[Epimorfismo]
	Un morfismo \(e \colon X \to A\) in una categoria \(\ctC\) è un \emph{epimorfismo} (o \emph{epi}),
	per ogni coppia di frecce parallele \(f, g \colon A \to B\) in \(\ctC\) tali che \(f \cmp e = g \cmp e\), si ha che \(f = g\).
	%Equivalentemente, \(e\) è un epimorfismo se è cancellabile a destra, ovvero \(f \cmp e = g \cmp e\) implica \(f = g\) ogni qual volta la scrittura ha senso (ovvero quando \(f\) e \(g\) sono frecce parallele e post-componibili con \(e\)).
\end{definition}

\begin{example}
	In \(\ctSet\), la categoria degli insiemi, i monomorfismi sono precisamente le funzioni iniettive
	e gli epimorfismi sono precisamente le funzioni surgettive.
	
	Consideriamo il caso delle funzioni surgettive.
	Se \(e \colon X \to A\) è una funzione surgettiva e \(f, g \colon A \to B\) sono funzioni tali che \(f \cmp e = g \cmp e\),
	allora per ogni elemento \(a \in A\) esiste un \(x \in X\) tale che \(e(x) = a\), per la surgettività di \(e\).
	Osserviamo che \((f \cmp e)(x) = (g \cmp e)(x)\), ovvero \(f(e(x)) = g(e(x))\),
	ma allora \(f(a) = g(a)\).
	Per la genericità di \(a\) concludiamo che \(f = g\), e dunque \(e\) è epi.

	Nel caso opposto, in cui \(e \colon X \to A\) è epi, sia \(a \in A\).
	Assumiamo per assurdo che non esista \(x \in X\) tale che \(f(x) = a\).
	Allora siano \(k_0 \colon A \to \{0, 1\}\) la funzione costante in \(0\),
	e \(\delta_a \colon A \to \{0, 1\}\) la funzione caratteristica di \(a\) in \(A\).
	Siccome \(a\) non appartiene all'immagine di \(e\), abbiamo che \(k_0 \cmp e = \delta_a \cmp e\)
	in quanto entrambe le composizioni assumono costantemente il valore \(0\) su tutto il dominio \(X\).
	Dunque, \(k_0 = \delta_a\), il che è assurdo.
	Dalla negazione dell'ipotesi per assurdo, concludiamo che \(e\) è surgettiva.
\end{example}

Si noti che la dimostrazione che le funzioni surgettive sono epi sarebbe immediata
se usassimo il fatto che ogni funzione surgettiva ha un'inversa destra.
L'esistenza dell'inversa destra, però, è una conseguenza dell'assioma di scelta,
che invece non viene impiegato nella dimostrazione di cui sopra.

\begin{exercise}
	Dimostrare che una funzione è iniettiva se e solo se è un mono in \(\ctSet\),
	ma senza utilizzare il fatto che le funzioni iniettive hanno inversa sinistra.
\end{exercise}

Per ragioni che saranno evidenti nella prossima sezione,
abbiamo evidenziato come l'equivalenza tra funzioni surgettive (iniettive)
e epi (mono) in \(\ctSet\) sia indipendente dall'esistenza di un'inversa destra (sinistra).



\section{Sezioni e retrazioni}

\begin{definition}[Sezione]
	Una \emph{sezione} di un morfismo \(f \colon A \to B\) in una categoria \(\ctC\) è un'inversa destra di \(f\), ovvero un morfismo \(s \colon B \to A\) tale che \(f \cmp s = \id_{B}\).
\end{definition}

\begin{definition}[Retrazione]
	Una \emph{retrazione} di un morfismo \(f \colon A \to B\) in una categoria \(\ctC\) è un'inversa sinistra di \(f\), ovvero un morfismo \(r \colon B \to A\) tale che \(r \cmp f = \id_{A}\).
\end{definition}