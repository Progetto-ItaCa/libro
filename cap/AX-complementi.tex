\chapter{Complementi di vario genere}
\section{Complementi al capitolo \ref{cap_preludi}}
\section{Complementi al capitolo \ref{chap_cat_fun_nat}}
\section{Complementi al capitolo \ref{chap_limiti_colimiti}}
\section{Complementi al capitolo \ref{cap_aggiunti}}
\section{Complementi al capitolo \ref{cap_yoneda}}
\section{Complementi al capitolo \ref{cap_yon_adj_limiti}}
\section{Complementi al capitolo \ref{cap_monadi}}
\section{Complementi al capitolo \ref{cap_fattorizzazione}}
\section{Complementi al capitolo \ref{cap_monoidali_interne}}
\section{Complementi al capitolo \ref{cap_nervi_real}}
% Non so se lo voglio mettere:
% \section{Nota di traduzione}
% Una tabella che compara la terminologia che è (quasi) universale nei riferimenti in inglese con la terminologia adottata nel libro.

% La lista è ordinata alfabeticamente in inglese (è più utile così, dato che dove è possibile menzioniamo già nel corso del testo qual è l'analogo inglese del termine italiano).

