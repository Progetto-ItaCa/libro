\chapter{Sistemi di Fattorizzazione}
\label{cap_fattorizzazione}
\section{La relazione di ortogonalità}
La nozione di \emph{sistema di fattorizzazione} in una categoria è uno dei concetti più antichi della teoria; Mac Lane produsse un esempio di sistema di fattorizzazione nella categoria dei gruppi nel 1948, \cite{maclane_grp_duality}.\footnote{L'articolo in questione è molto breve, e istruttivo a proposito dell'intuizione che, pochi anni dopo la sua prima definizione esplicita, la nozione di categoria aveva per il suo inventore; è anche un reperto storico sulla terminologia matematica e su come essa è evoluta: una categoria equipaggiata con un sistema di fattorizzazione viene chiamata una \emph{bicategoria} da Mac Lane, termine che oggi è riservato a una struttura che nulla ha a che vedere con la fattorizzazione di morfismi. Notiamo anche che la definizione che daremo in \ref{fat_funno} è leggermente diversa, ma equivalente, a quella proposta da Mac Lane, che chiede che ogni morfismo \(f : X\to Y\) si fattorizzi come una composizione \(m\cmp u\cmp e\), dove \(m\) è un monomorfismo, \(e\) un epimorfismo, e \(u\) un isomorfismo. L'assiomatica proposta da Mac Lane serviva a determinare una nozione di decomposizione di una categoria \(\ctC\) mediante due sottoclassi \(\ctE,\ctM\subseteq\ctC_1\) tali che se \((\ctE,\ctM)\) determina una fattorizzazione su \(\ctC\), allora \((\ctM^\op,\ctE^\op)\) determina una fattorizzazione su \(\ctC^\op\).} Da quel momento, la nozione ha dimostrato di essere fondamentale per unificare diverse costruzioni universali in algebra, logica e (soprattutto) in topologia.

\medskip
L'intuizione fondamentale dietro la definizione è la seguente osservazione/lemma:
\begin{lemma}\label{fatt_in_set}
	Nella categoria \(\ctSet\) di insiemi e funzioni, descritta in \ref{ex_cat_insiemi}, ogni funzione \(f : X \to Y\) si fattorizza come \(f = m \cmp e\), dove \(m\) è iniettiva (un monomorfismo, si veda \ref{monoepi_in_set}) ed \(e\) è suriettiva.% (un epimorfismo, si veda \ref{def_Epi}).
\end{lemma}
In più, abbiamo preannunciato in \ref{caratt_epi_con_ort} che la nozione di epimorfismo e di monomorfismo ammettono una ulteriore caratterizzazione nei termini seguenti (rammentiamo il contenuto di \ref{caratt_epi_con_ort} per facilità di lettura): il lemma \ref{fatt_in_set} permette di dimostrare questo fatto in modo molto elegante.
\begin{proposition*}
	Sia \(p : E \to B\) una funzione tra insiemi; le seguenti condizioni sono equivalenti:
	\begin{itemize}
		\item \(p\) è un epimorfismo (si veda \ref{def_Mono});
		\item in ogni diagramma di insiemi e funzioni come il seguente:
		      \[\begin{tikzcd}
				      E \ar[r, "f"]\ar[d, "p"']& X \ar[d, "m"]\\
				      B \ar[r, "g"'] \ar[ur, dashed, "u"]& Y
			      \end{tikzcd}\]
		      dove \(m : X\to Y\) è un monomorfismo, esiste un unico morfismo \(u : B\to X\) tale che \(u\cmp p = f\) e \(m\cmp u = g\).
	\end{itemize}
\end{proposition*}
\begin{proof}
	\Todo{Qual era la dimostrazione elegante che avevo in mente? Non me lo ricordo più}
\end{proof}
Questo motiva la nozione di \emph{ortogonalità} tra morfismi (si veda \ref{def_ortogona} sotto) e di \emph{fattorizzazione} di ogni morfismo \(f : X\to Y\) di una categoria \(\ctC\) come una certa composizione in termini di due sottoclassi \(\ctE,\ctM\subseteq\ctC\)
\[
	\begin{tikzcd}
		f : X \ar[r, "e", two heads]& F \ar[r, "m", hook] & Y.
	\end{tikzcd}
\]
dove \(e\in\ctE\) e \(m\in\ctM\) sono morfismi nelle due sottoclassi.

L'intuizione che chi sta leggendo dovrebbe avere sulla definizione è che una struttura del genere imposta su una categoria \(\ctC\) permette di `generarla' a partire dalle classi \(\ctE,\ctM\), in una notazione suggestiva, cioè, è vero che \(\hom(\ctC) = \ctM\cmp \ctE\). In più, le due classi \(\ctE,\ctM\) sono `ortogonali' nel senso che hanno intersezione più piccola possibile: \(\ctE\cap \ctM\) consta del \emph{cuore} (si veda \ref{def_cuore}) di \(\ctC\), cioè della categoria di tutti gli isomorfismi in \(\ctC\).
\color{red}
L'esempio più importante (sebbene non l'unico interessante) di fattorizzazione di questo tipo sarà proprio quando \(\ctE\) è la classe \(\Epi\) degli epimorfismi ed \(\ctM\) è la classe \(\Mono\) dei monomorfismi di \(\ctC\). Quando le proprietà di ortogonalità e fattorizzazione valgono per la coppia \((\ctE,\ctM)\) questa si dice un \emph{sistema di fattorizzazione} su \(\ctC\).
\begin{remark}
	Chi legge non deve con questo pensare che data una qualsiasi categoria \(\ctC\), la coppia di classi \((\Epi,\Mono)\) formi sempre un sistema di fattorizzazione su \(\ctC\).

	In effetti, non è nemmeno vero che \((\Epi,\Mono)\) è sempre una coppia di classi ortogonali nel senso di \ref{def_ortogona} (si pensi al caso dei preordini, dove \emph{ogni} morfismo è sia un monomorfismo che un epimorfismo, cosicché \(\Epi[\ctPos]\cap\Mono[\ctPos]=\hom(\ctPos)\), ma non tutti gli oggetti di un preordine sono tra loro isomorfi).
\end{remark}
\color{black}
Sulla classe dei morfismi di una generica categoria \(\ctC\) è possibile definire una relazione binaria detta \emph{ortogonalità}, nel modo che segue.
\begin{definition}[Relazione di ortogonalità]\label{def_ortogona}\index{Ortogonalità}\index{Morfismi!--- ortogonali}\index{Relazione!---di ortogonalità}
	Dati due morfismi \(f : X\to Y\) e \(g : A\to B\) di una categoria \(\ctC\), diciamo che \(f\) è \emph{ortogonale a sinistra} a \(g\), o che \(g\) è \emph{ortogonale a destra} a \(f\), o che \(f\) e \(g\) sono \emph{ortogonali} se in ogni quadrato commutativo della forma
	\[
		\begin{tikzcd}
			X \ar[d, "f"']\ar[r, "u"]& A\ar[d, "g"] \\
			Y \ar[r, "v"']& B
		\end{tikzcd}
	\]
	esiste un unico morfismo \(a : Y\to A\) tale che \(a\cmp f=u\) e \(g\cmp a=v\).
\end{definition}
La relazione di ortogonalità è una sottoclasse \(O^\ctC \subseteq \hom(\ctC)\times\hom(\ctC)\) e il fatto che \((f,g)\in O^\ctC\) verrà sempre denotato con un simbolo infisso \(f\perp^\ctC g\), o semplicemente come \(f\perp g\) quando il contesto permette di determinare quale sia la categoria \(\ctC\) di riferimento (ossia, quasi sempre).

La relazione di ortogonalità è molto lontana dall'essere riflessiva, simmetrica o transitiva; \ref{} sostanzia questa osservazione con delle maniere di costruire dei controesempi.
\begin{proposition}\label{ort_to_terminale}
	Sia \(\ctC\) una categoria che ha un oggetto terminale \(\term\). Allora, dati un morfismo \(f : X\to Y\), e un oggetto \(A\in \ctC_0\), le condizioni seguenti sono equivalenti:
	\begin{enumtag}{ot}
		\item\label{ot_1} il morfismo \(f\) è ortogonale a sinistra al morfismo terminale \(A\to \term\);
		\item\label{ot_2} il funtore \(\Hom{\ctC}(\blank,A)\) manda il morfismo \(f\) in una biiezione
		\[\begin{tikzcd}
				\Hom{\ctC}(Y,A) \ar[r, "{g\cmp \blank}"] & \Hom{\ctC}(X,A).
			\end{tikzcd}\]
	\end{enumtag}
	Dualmente, sia \(\ctC\) una categoria con un oggetto iniziale \(\init\). Allora, dati un morfismo \(g : A\to B\), e un oggetto \(Y\in\ctC_0\), le condizioni seguenti sono equivalenti:
	\begin{enumtag}{oi}
		\item \label{oi_1} il morfismo \(g\) è ortogonale a destra al morfismo iniziale \(\init\to Y\);
		\item \label{oi_2} il funtore \(\Hom{\ctC}(B,\blank)\) manda il morfismo \(g\) in una biiezione
		\[\begin{tikzcd}
				\Hom{\ctC}(Y,A) \ar[r, "{\blank\cmp f}"] & \Hom{\ctC}(Y,B).
			\end{tikzcd}\]
	\end{enumtag}
\end{proposition}
\begin{proof}
	La condizione che per ogni data freccia \(u : X\to A\), esista un unico morfismo \(\bar u : Y\to A\) tale che \(\bar u\cmp f = u\) è esattamente la richiesta che la funzione \(u\mapsto u\cmp f\) sia suriettiva (esistenza) e iniettiva (unicità). Questo dimostra l'equivalenza di \ref{ot_1} e \ref{ot_2}, e similmente si procede per l'equivalenza di \ref{oi_1} e \ref{oi_2}.
\end{proof}
\begin{remark}\label{perche_ortogonale}
	\Todo{Diverse intuizioni sulla nomenclatura `ortogonale'.}

	Una intuizione a proposito del nome `ortogonale' scelto per \ref{} viene dall'algebra lineare: così come due sottospazi \(U,W\) di uno stesso spazio vettoriale sono in somma ortogonale, scritto \(U\perp W\), se \(u\cdot w=0\) per ogni \((u,w)\in U\times W\); questo è equivalente alla condizione \(U\le W^\perp\), o alla condizione \(W\le U^\perp\), dove per un sottospazio \(Z\) si definisce \(Z^\perp = \{v\in V\mid \forall z\in Z,\,z\cdot v=0\}\). Questa analogia formale verrà sostanziata in \ref{} dove vedremo che mandare una classe di morfismi nel suo ortogonale definisce un operatore di chiusura (così come accade per l'ortogonale di sottospazi vettoriali).

	Una analogia meno formale ma suggestiva: grazie a \ref{ort_to_terminale} la relazione di ortogonalità può essere considerata tra un morfismo e un \emph{oggetto}. Da qui\dots
	%   It means X
	%  “thinks” f
	%  is an isomorphism in the sense that hom(f,X):hom(B,X)→hom(A,X)
	%  is an isomorphism (i.e., as far as X
	% -probes or coprobes are concerned, f
	%  is an iso). If we think of an isomorphism as a notion of sameness, then f
	%  makes A
	%  and B
	%  look just the same as gauged by the instrument X
	% .
	%
	% Now in elementary physics, say if we are measuring temperature by a thermometer X
	% , there are certain trajectories f:A→B
	%  between points A,B
	%  in space which exhibit sameness as gauged by X
	% , namely by traveling along isothermal lines. These are orthogonal to the “direction” of X
	%  given by gradients (really the ⊥
	%  here is a relation between tangent vectors which are infinitesimal f
	%  and 1-forms dX
	% , but hopefully you get the idea).
	%
	% When you transport all this to the slice category over Y
	% , you get the standard notion of an object g:X→Y
	%  being orthogonal to an arrow f
	%  in the slice.
\end{remark}
Miscellanea di definizioni: classi ampie, sature, cellulari, chiuse per (co)limiti, cancellative a sinistra (destra, bilatere).
\begin{definition}
	Una \emph{successione transfinita} di morfismi in una categoria \(\ctC\) consiste di un funtore \(f : \ctOrd \to \ctC\), dove \(\ctOrd\) è la categoria di \ref{}, che preserva i colimiti; in parole semplici, una successione transfinita in \(\ctC\) consiste di una successione di oggetti \(X_\lambda\), uno per ogni ordinale \(\lambda\), e di morfismi
	\[
		\begin{tikzcd}
			X_0 \ar[r] & X_1\ar[r] & X_2 \ar[r]& \dots \ar[r] & X_\alpha \ar[r] & X_{\alpha+1}\ar[r] & \dots
		\end{tikzcd}
	\]
	tale che per ogni ordinale limite \(\lambda = \colim_{\beta<\lambda}\beta\) si abbia che \(X_\lambda\cong \colim_{\beta < \lambda} X_\beta\). I morfismi di questa catena sono quindi essenzialmente di due tipi:
	\begin{itemize}
		\item \(X_\alpha \to X_{\alpha+1}\) ogni volta che \(\beta=\alpha+1\) è un ordinale successore;
		\item morfismi da un ordinale \(\theta<\lambda\) dentro l'`oggetto limite' \(\colim_{\beta<\lambda}X_\beta\cong X_\lambda\).
	\end{itemize}
	Spesso si limita la taglia di queste successioni a uno specifico ordinale limite \(\lambda\), cioè si considerano solo funtori \(\ctOrd_{<\lambda}\to\ctC\) (in questo modo, la categoria delle successioni transfinite \(\ctOrd \to \ctC\) è legittima); in questo caso parliamo di una \emph{\(\lambda\)-successione in \(\ctC\)}. Data una (\(\lambda\)-)successione \(f : X_0\to X_1 \to \dots\) chiamiamo la sua \emph{composizione} l'unico morfismo \(X_0\to \colim_{\beta<\lambda} X_\beta\).
\end{definition}
\begin{proposition}
	Se una \(\omega\)-successione \(f : \ctOrd_{<\omega}\to\ctSet\) ha la proprietà che ogni \(X_n \to X_{n+1}\) è una funzione iniettiva, è iniettiva anche la composizione transfinita \(f_\infty : X_0\to \colim_{n<\omega} X_n\).
\end{proposition}
\begin{proof}
	La mappa \(f_\infty\) è definita dalla proprietà universale del colimite \(\colim_{n<\omega} X_n\); se \(f_\infty(x)=f_\infty(x')\), allora esiste un \(n<\omega\) tale che \(f_{0n}(x) = f_{:n}(x')\) dove \(f_{:n}\) è la funzione composta
	\[\begin{tikzcd}
			X_0 \ar[r, "f_1"]& X_1\ar[r, "f_2"] & \dots \ar[r, "f_n"] & X_n
		\end{tikzcd}\]
	che è iniettiva; allora, \(x=x'\).

	(Evidentemente un argomento simile dimostra, per induzione transfinita, che la composizione di ogni \(\lambda\)-successione \(f : \ctOrd_{<\lambda}\to\ctSet\) tale che ogni \(f_\alpha\) sia un monomorfismo è ancora un monomorfismo.)
\end{proof}
\begin{definition}
	Una classe di morfismi \(\ctK\subseteq\ctC_1\) di una category \(\ctC\) si dice \emph{chiusa per composizioni transfinite} se, per ogni \(\lambda\)-successione \(f : \ctOrd_{<\lambda}\to\ctK\) la composizione transfinita \(f_{:\lambda} : X_0\to \colim_{\alpha < \lambda} X_\alpha\) è ancora un morfismo di \(\ctK\).
\end{definition}
\begin{definition}[Classi speciali di morfismi]
	Sia \(\ctC\) una categoria. Una classe di morfismi \(\ctK\subseteq\ctC_1\) si dice
	\begin{itemize}
		\item \emph{ampia} se la sottocategoria individuata da \(\ctK\) è ampia nel senso di \ref{def_subcat}; in particolare, \(\ctK\) contiene tutti i morfismi identici, e quindi tutti gli oggetti di \(\ctC\), ed è chiusa per composizione.
		\item \emph{presatura} se ogni volta che sia dato un diagramma
		      \[
			      \begin{tikzcd}
				      A \ar[d, "k"']\ar[r, "f"]& X \\
				      B &
			      \end{tikzcd}
		      \]
		      tale che \(k\in\ctK\), se la somma amalgamata \(\push BXA\) esiste, allora il morfismo \(k' : B\to \push BXA\) è anch'esso in \(\ctK\).
		\item \emph{quasisatura} se è presatura e chiusa per retratti (nella categoria dei morfismi \(\ctC^\to\), si veda \ref{def_cat_frecce}): significa che, dato un diagramma
		      \[
			      \begin{tikzcd}
				      A \ar[r, "i"]\ar[d, "k'"'] & K \ar[r, "r"]\ar[d, "k"]& A \ar[d, "k'"']\\
				      B \ar[r, "i'"'] & K' \ar[r, "r'"'] & B
			      \end{tikzcd}
		      \]
		      dove \(r\cmp i = 1_A\) e \(r'\cmp i' = 1_B\), se \(k\in \ctK\), allora anche \(k'\in\ctK\).
		\item \emph{cellulare} se è presatura e chiusa per composizione transfinita nel senso di \ref{}.
		\item \emph{satura} se è quasisatura e cellulare.
		\item \emph{cancellativa a destra} (risp., \emph{a sinistra}) se\dots; cancellativa, se lo è da entrambi i lati.
		\item \emph{chiusa per limiti} (risp., \emph{per colimiti}) se\dots
	\end{itemize}
\end{definition}
\begin{examples}
	\Todo{Esempi di classi ampie, sature, etc. in varie categorie}
\end{examples}
\begin{esercizi}
	\item
	\item
	\item
	\item
	\item
\end{esercizi}
\section[Proprietà ed esempi]{Proprietà fondamentali ed esempi}
\begin{remark}
	Grazie a \ref{klext_delle_relazioni} possiamo estendere la relazione di ortogonalità a una coppia di funzioni sulle sottoclassi di \(\hom(\ctC)\),
	\[\lort{(-)} : 2^{\hom(\ctC)} \leftrightarrows 2^{\hom(\ctC)} : \rort{(-)}\]
	tali che \(\rort\ctU\subseteq\ctV\) se e solo se \(\ctU\subseteq\lort\ctV\) e definite alla maniera seguente: data una sottoclasse \(\ctU\subseteq \ctC_1\),
	\begin{gather*}
		\rort{\ctU} := \{f\in\ctC_1\mid \forall u\in\ctU.(u\perp f)\}\\
		\lort{\ctU} := \{f\in\ctC_1\mid \forall u\in\ctU.(f\perp u)\}
	\end{gather*}
	Così, possiamo definire la relazione di ortogonalità estesa alle sottoclassi di \(\hom(\ctC)\),
	\[\ctL\perp \ctR := \forall l\in\ctL,r\in\ctR.(l\perp r).\]
	Si osservi che le seguenti tre condizioni sono equivalenti (lo si dimostri come esercizio per familiarizzare con la definizione):
	\[\ctL\perp\ctR\qquad \ctL\subseteq\lort\ctR\qquad \rort\ctL\subseteq \ctR\]
\end{remark}
\begin{definition}[Prodotto di Leibniz]
	Siano \(f : A \to B\) e \(g : X\to Y\) due morfismi di ua categoria \(\ctC\). Il \emph{prodotto di Leibniz} di \(f\) e \(g\), denotato \(f\oast g\), si costruisce a partire dal prodotto fibrato
	\[
		\begin{tikzcd}
			\pull {\Hom{\ctC}(A,X)}{\Hom{\ctC}(B,Y)}{\Hom{\ctC}(A,Y)} \ar[r]\ar[d] & \Hom{\ctC}(B,Y) \ar[d, "\blank\cmp f"]\\
			\Hom{\ctC}(A,X) \ar[r, "g\cmp\blank"'] & \Hom{\ctC}(A,Y)
		\end{tikzcd}
	\]
	e consiste della mappa di confronto \(f\oast g : \Hom{\ctC}(B,X) \to \pull {\Hom{\ctC}(A,X)}{\Hom{\ctC}(B,Y)}{\Hom{\ctC}(A,Y)}\) indotta dalla proprietà universale.

	Diciamo che il prodotto di Leibniz di \(f,g\) è \emph{esatto} se \(f\oast g\) è una biiezione. Questo è equivalente ad affermare che il quadrato
	\[
		\begin{tikzcd}
			\Hom{\ctC}(B,X) \ar[d]\ar[r,"g\cmp\blank"]& \Hom{\ctC}(B,Y)\ar[d, "\blank\cmp f"]\\
			\Hom{\ctC}(A,X) \ar[r, "g\cmp\blank"'] & \Hom{\ctC}(A,Y)
		\end{tikzcd}
	\]
	è un prodotto fibrato.
\end{definition}
Per orientare chi legge, è utile rendere esplicita la definizione del prodotto di Leibniz \(f\oast g\): un morfismo \(u : B\to X\) viene mandato nella coppia \((u\cmp f, g\cmp u)\), che appartiene al prodotto fibrato in questione vista l'associatività della composizione: \(g\cmp (u\cmp f) = (g\cmp u)\cmp f\).

Il prodotto di Leibniz permette di ridefinire la relazione di ortogonalità mediante l'esattezza, ossia l'invertibilità di \(f\oast g\); la dimostrazione del seguente fatto consiste nel riscrivere le due condizioni in modo che entrambe si rifrasino così:
\begin{quote}
	Per ogni coppia di morfismi \(u : X\to A\), \(v : Y\to B\) tali che \(g\cmp u=v\cmp f\), esiste un unico morfismo \(a : Y\to A\) tale che \(g\cmp a=v\) e \(a\cmp f=u\).
\end{quote}
\begin{proposition}
	Due morfismi \(f,g\) sono ortogonali (cioè \(f\perp g\) nel senso di \ref{def_ortogona}) se e solo se il prodotto di Leibniz \(f\oast g\) è esatto.
\end{proposition}
\begin{proof}
	Evidente: la condizione su \(a\) in \ref{def_ortogona} si traduce in esistenza (suriettività di \(f\oast g\)) e unicità (iniettività).
\end{proof}
\begin{definition}[Sistema di prefattorizzazione]
	\Todo{}
\end{definition}
\begin{proposition}
	Ciascuna classe in un prefact definisce l'altra --dunque la def di prefact è ridondante
\end{proposition}
\begin{definition}\label{fat_funno}\index{Fattorizzazione!--- funtoriale}
	Fattorizzazione funtoriale
	(Remark: ci occuperemo solo di sistemi di fattorizzazione funtoriali in questo senso)
\end{definition}
\begin{definition}
	L'operatore punto e virgola e le sue proprietà
\end{definition}
\begin{definition}
	Sistema di fattorizzazione: due classi, tra loro strettamente ortogonali, tali che \(\ctC_1 = \ctM \fatsemi \ctE\).
\end{definition}
\Todo{missing figure here}
% \begin{figure}[h]
% 	\begin{center}
% 		\begin{tikzpicture}[
% 				x=1em, y=1em,
% 				dot/.style={
% 						circle,
% 						fill=#1,
% 						inner sep=0pt,
% 						outer sep=2pt,
% 						minimum size=4pt,
% 						draw=none,
% 					},
% 				wrap/.style={
% 						fill=black!5,
% 						rounded corners,
% 						inner sep=.5em,
% 					},
% 			]

% 			\def\items{%
% 				0/0/0/ 0.0,
% 				1/0/1/-0.5,
% 				1/1/1/+0.5,
% 				2/0/2/-1.0,
% 				2/1/2/ 0.0,
% 				2/2/2/+1.0,
% 				2/3/3/-1.0,
% 				2/4/3/ 0.0,
% 				2/5/3/+1.0,
% 				3/0/4/-1.5,
% 				3/1/4/-0.5,
% 				3/2/4/+0.5,
% 				3/3/4/+1.5}

% 			\foreach \n/\v/\row/\col in \items {
% 				\begin{scope}[xshift=7em*\col,yshift=-6.25em*\row,local bounding box=C\n\v]
% 					\foreach \i in {0,1,2} {
% 					\node[dot=black] (C\n\v-\i) at (\(({120*\i-210+180}:1.5)\)) {};
% 					\path[ibmPurple, loop, out=-60+120*\i, in=120*\i, distance=1em] (C\n\v-\i) edge (C\n\v-\i);
% 					}
% 				\end{scope}
% 			}

% 			\begin{scope}[on background layer]
% 				\foreach \n/\v/\row/\col in \items
% 				\path node[wrap, fit=(C\n\v)] (W\n\v) {};
% 			\end{scope}

% 			\begin{scope}[every path/.style={-latex}]
% 				\path[ibmMagenta] (C10-2) edge (C10-0);

% 				\path[ibmBlue]    (C11-2) edge (C11-0);

% 				\path[ibmBlue]    (C20-1) edge (C20-2);
% 				\path[ibmBlue]    (C20-1) edge (C20-0);

% 				\path[ibmBlue]    (C21-1) edge (C21-2);
% 				\path[ibmMagenta] (C21-1) edge (C21-0);

% 				\path[ibmMagenta] (C22-1) edge (C22-2);
% 				\path[ibmMagenta] (C22-1) edge (C22-0);

% 				\path[ibmBlue]    (C23-2) edge (C23-1);
% 				\path[ibmBlue]    (C23-0) edge (C23-1);

% 				\path[ibmBlue]    (C24-2) edge (C24-1);
% 				\path[ibmMagenta] (C24-0) edge (C24-1);

% 				\path[ibmMagenta] (C25-2) edge (C25-1);
% 				\path[ibmMagenta] (C25-0) edge (C25-1);

% 				\path[ibmBlue]    (C30-2) edge (C30-1);
% 				\path[ibmBlue]    (C30-1) edge (C30-0);
% 				\path[ibmBlue]    (C30-2) edge (C30-0);

% 				\path[ibmBlue]    (C31-2) edge (C31-1);
% 				\path[ibmMagenta] (C31-1) edge (C31-0);
% 				\path[black]      (C31-2) edge (C31-0);

% 				\path[ibmMagenta] (C32-2) edge (C32-1);
% 				\path[ibmBlue]    (C32-1) edge (C32-0);
% 				\path[black]      (C32-2) edge (C32-0);

% 				\path[ibmMagenta] (C33-2) edge (C33-1);
% 				\path[ibmMagenta] (C33-1) edge (C33-0);
% 				\path[ibmMagenta] (C33-2) edge (C33-0);
% 			\end{scope}
% 		\end{tikzpicture}
% 		\caption{TODO}
% 		\label{fig_sistema_di_fattorizzazione}
% 	\end{center}
% \end{figure}
\begin{itemize}
	% \item estensione ai sottoinsiemi dell'ortogonalità
	% \item Caratterizzazioni equivalenti della relazione di ortogonalità
	% \item def di prefact, interdefinibilità delle due classi
	% \item funtorialità della fattorizzazione
	% \item stabilità per lim/colim
	% \item proprietà di cancellazione
	% \item classi ampie, sature, cellulari
	\item induzione di pre-FS su categorie di funtori, slice, coslice, (comma, cocomma?)
	\item esempi in algebra, topologia, geometria, logica, eccetera (epi e mono in set; verticali e cartesiani; etc)
	\item il poset dei sistemi di fattorizzazione
\end{itemize}
\subsection{Esempi di sistemi di fattorizzazione}
\begin{example}
	Epi e mono in Set
\end{example}
Più in generale, epi e mono in ogni categoria di modelli per una segnatura
\begin{example}

\end{example}

\begin{example}

\end{example}

\begin{example}

\end{example}

\begin{example}

\end{example}

\begin{example}
	Induzione di (pre-)FS su categorie di funtori, slice, coslice, ecc.
\end{example}

\begin{esercizi}
	\item
	\item
	\item
	\item
	\item
\end{esercizi}
\section[Fattorizzazione]{Sistemi di fattorizzazione forti e deboli}
Ortogonalità non-unica
\begin{itemize}
	\item fattorizzabilità unica ed essenzialmente unica;
	\item fattorizzabilità non-unica
	\item esempi di sistemi con fattorizzabilità non-unica
	\item indurre sistemi di fattorizzazione (argomento degli oggetti piccini)
	\item
\end{itemize}
\begin{esercizi}
	\item
	\item
	\item
	\item
	\item
\end{esercizi}
\section[Riflessività]{Sistemi di fattorizzazione riflessivi}
\begin{esercizi}
	\item
	\item
	\item
	\item
	\item
\end{esercizi}
\section[Mono ed epimorfismi]{Classi di mono ed epimorfismi}
\begin{esercizi}
	\item
	\item
	\item
	\item
	\item
\end{esercizi}
\section[Iniettivi e proiettivi]{Oggetti iniettivi e proiettivi}
\Todo{Teorema di Gleason sui proiettivi in Top, da qualche parte, qui}
\begin{esercizi}
	\item
	\item
	\item
	\item
	\item
\end{esercizi}
\section[Fattorizzazione e algebre]{Sistemi di fattorizzazione come algebre}
\begin{esercizi}
	\item
	\item
	\item
	\item
	\item
\end{esercizi}

