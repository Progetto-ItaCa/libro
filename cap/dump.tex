Per risolvere questo esercizio è necessario ricordare la definizione di una \emph{successione esatta} di omomorfismi di gruppo: se \(H,G,G',K\) sono gruppi abeliani, una successione esatta corta è una sequenza di mappe componibili
% % https://q.uiver.app/#q=WzAsNixbMCwwLCIwIl0sWzEsMCwiSyJdLFsyLDAsIkciXSxbMywwLCJHJyJdLFs1LDAsIjAiXSxbNCwwLCJIIl0sWzAsMV0sWzEsMiwiaCJdLFsyLDMsImYiXSxbMyw1LCJnIl0sWzUsNF1d
% \[\begin{tikzcd}
% 		0 & K & G & {G'} & H & 0
% 		\arrow[from=1-1, to=1-2]
% 		\arrow["g", from=1-2, to=1-3]
% 		\arrow["f", from=1-3, to=1-4]
% 		\arrow["h", from=1-4, to=1-5]
% 		\arrow[from=1-5, to=1-6]
% 	\end{tikzcd}\]
% tale che \(\ker f=\im g\), \(\ker h=\im f\), \(\ker g=(0)\) e \(\im h = H\). Chi non ha mai visto questo risultato, dimostri che ogni omomorfismo \(f : G\to G'\) tra due gruppi abeliani induce una successione esatta come questa, se \(K=\ker f\) e \(H=G'/\im f\), quando \(g : \ker f \mono G\) è l'inclusione ovvia, e \(h\) la proiezione al quoziente sul conucleo (si veda \ref{ker_e_coker}) di \(f\).

% Sia ora \(\ctC\) una categoria piccola e consideriamo la categoria dei funtori \(F : \ctC\funto\ctAb\) di codominio la categoria dei gruppi abeliani. Questo esercizio chiede di definire e studiare una nozione di nucleo associato a una trasformazione naturale:
% \begin{itemize}
% 	\item Definire \(\ker\alpha\) come la trasformazione naturale associata ad \(\alpha : F \natto G\) e avente per componenti
% 	      \[(\ker\alpha)_C := \ker(\alpha_C : FC\to GC) \to FC.\]
% 	\item Mostrare che \(X\mapsto\ker(\alpha_X)\) definisce un funtore \(K : \ctC\fun\ctAb\), e che \(\kappa_X : \ker(\alpha_X) \to FX\) è naturale e ha ciascuna componente un omomorfismo iniettivo di gruppi; a questa maniera, esiste una successione esatta \emph{di trasformazioni naturali} (cioè una successione di trasformazioni naturali, esatta quando saturata in ogni componente) della forma
% 	      \[0\natto\ker\alpha \overset\kappa\natto F \overset\alpha\natto G\]
% 	      dove chiaramente `\(0\)' è il funtore costante nel gruppo abeliano nullo.
% 	\item Dualizzare e definire il \emph{conucleo} di una trasformazione naturale; costruire la successione esatta
% 	      \[\begin{tikzcd}
% 			      0 & \ker\alpha & F & G & \coker\,\alpha & 0.
% 			      \arrow[from=1-1, to=1-2, Rightarrow]
% 			      \arrow["\kappa", from=1-2, to=1-3, Rightarrow]
% 			      \arrow["\alpha", from=1-3, to=1-4, Rightarrow]
% 			      \arrow["\pi", from=1-4, to=1-5, Rightarrow]
% 			      \arrow[from=1-5, to=1-6, Rightarrow]
% 		      \end{tikzcd}\]
% 	\item Mostrare, o trovare un controesempio, al seguente enunciato: per ogni funtore \(H : \ctB\fun\ctC\) come nel diagramma
% 	      % https://q.uiver.app/#q=WzAsMyxbMiwwLCJcXGN0QyJdLFs0LDAsIlxcY3REIl0sWzAsMCwiXFxjdEIiXSxbMiwwLCJIIl0sWzAsMSwiRiIsMCx7ImxhYmVsX3Bvc2l0aW9uIjozMH1dLFswLDEsIksiLDAseyJjdXJ2ZSI6LTN9XSxbMCwxLCJHIiwyLHsiY3VydmUiOjN9XSxbNSw0LCJcXGtlclxcYWxwaGEiLDAseyJzaG9ydGVuIjp7InNvdXJjZSI6MjAsInRhcmdldCI6MjB9fV0sWzQsNiwiXFxhbHBoYSIsMCx7InNob3J0ZW4iOnsic291cmNlIjoyMCwidGFyZ2V0IjoyMH19XV0=
% 	      \[\begin{tikzcd}
% 			      \ctB && \ctC && \ctD
% 			      \arrow["H", from=1-1, to=1-3]
% 			      \arrow[""{name=0, anchor=center, inner sep=0}, "F"{pos=0.3}, from=1-3, to=1-5]
% 			      \arrow[""{name=1, anchor=center, inner sep=0}, "K", curve={height=-18pt}, from=1-3, to=1-5]
% 			      \arrow[""{name=2, anchor=center, inner sep=0}, "G"', curve={height=18pt}, from=1-3, to=1-5]
% 			      \arrow["{\ker\alpha}", shorten <=2pt, shorten >=2pt, Rightarrow, from=1, to=0]
% 			      \arrow["\alpha", shorten <=2pt, shorten >=2pt, Rightarrow, from=0, to=2]
% 		      \end{tikzcd}\]
% 	      esiste una successione esatta
% 	      \[\begin{tikzcd}
% 			      0 & \ker\alpha H & F H & G H & \coker\,\alpha H & 0.
% 			      \arrow[from=1-1, to=1-2, Rightarrow]
% 			      \arrow["\kappa*H", from=1-2, to=1-3, Rightarrow]
% 			      \arrow["\alpha*H", from=1-3, to=1-4, Rightarrow]
% 			      \arrow["\pi*H", from=1-4, to=1-5, Rightarrow]
% 			      \arrow[from=1-5, to=1-6, Rightarrow]
% 		      \end{tikzcd}\]
% \end{itemize}