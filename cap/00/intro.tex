\chapter{Quattro preludi categoriali}\label{cap_preludi}
\Todo{(Questa è l'introduzione che abbozzai anni fa, la metto qui solo per esporla alla vista e ne parleremo seriamente solo verso la fine)}
La teoria delle categorie è una parte della matematica molto giovane se confrontata con il calcolo differenziale, utilizzato da secoli; è un germoglio cresciuto all'improvviso quando posta a fianco di alberi antichi come la geometria, la logica, l'aritmetica che sotto varie forme si sono sviluppate lungo i millenni in Eurasia (notoriamente), ma anche in Africa e sud America \cite{ascher1991ethnomathematics}.

Nello spazio di pochi decenni la teoria delle categorie è apparsa nel panorama matematico e si è espansa in ogni direzione; i suoi concetti più fondamentali sono stati introdotti da Samuel Eilenberg e Saunders Mac Lane in un singolo articolo nel 1945 \cite{gtone}; i due credevano ingenuamente che quel loro lavoro sarebbe stato `l'unico articolo di ricerca che sarebbe mai stato necessario scrivere su questo tema' (le parole sono di Mac Lane in \cite{maclane1988concepts}).

Al contrario, durante gli ultimi tre quarti di secolo, due generazioni di persone hanno iniziato a riorganizzare con enorme vitalità l'insieme delle conoscenze della matematica moderna mediante questo linguaggio, fino a tentare l'ambizioso progetto di diventare un loro possibile fondamento.

Un modo di raccontare cos'è la teoria delle categorie è quindi questo: un altro tentativo di unificare la matematica grazie a poche, ricorrenti idee universali, che sono evidenti alla pratica quotidiana di chi lavora con le sue strutture.
\medskip

Fermarsi qui a raccontare la teoria delle categorie però tradirebbe parte della sua storia. Essa nasce infatti anche con un intento molto più concreto e squisitamente \emph{pratico}, che si può riassumere nel desiderio, da parte di chi l'ha inventata, di astrarre ad una definizione generale una situazione come la seguente.
\begin{example}
	Sia \(A\) un anello commutativo e unitario; esiste una mappa detta `determinante',
	\[\textstyle\det_A : M(A,2) \to A\]
	che manda una generica matrice \(2\times 2\) a coefficienti in \(A\), diciamo \(\begin{smat} a_{11} & a_{12}\\a_{21} & a_{22}\end{smat}\), in \(a_{11}a_{22}-a_{12}a_{21}\).
\end{example}
L'insieme delle matrici \(2\times 2\) a coefficienti in \(A\) è in un senso evidente ottenuto `in funzione' di \(A\), perché è costruito esattamente in termini delle quaterne ordinate di elementi di \(A\); in più, le varie `componenti' \(\alpha_A\) variano al variare di \(A\), ossia non si ha solo \(\det_A\), ma una \emph{classe} di funzioni \(\det_A\), una per ogni anello \(A\), tutte definite alla maniera precedente, quale che sia l'anello \(A\) in questione.

La specifica delle funzioni `determinante' è ossia \emph{polimorfa nel parametro} \(A\), ossia si può pensare (in modo impreciso, ma evocativo) come una funzione che ad \(A\) assegna una funzione \(\det_A\) di tipo \(M(A,2) \to A\).

In più, la seguente proprietà vale: se \(f : A\to B\) è un omomorfismo di anelli commutativi e unitari, la composizione di funzioni
\[\xymatrix{
	M(A,2) \ar[r]^-{\det_A}& A\ar[d]^f \\
	& B
	}\]
definita da \(M\mapsto f(\det_A(M)) = f(a_{11})f(a_{22})-f(a_{12})f(a_{21})\) è uguale alla composizione di funzioni
\[\xymatrix{
	M(A,2) \ar[d]_{M(f,2)}&  \\
	M(B,2) \ar[r]_-{\det_B} & B
	}\] dove \(M(f,2)\) `applica \(f\) ad ogni entrata' della matrice \(M\), ossia
\[M(f,2) : \begin{smat} a_{11} & a_{12}\\a_{21} & a_{22}\end{smat} \mapsto \begin{smat} f(a_{11}) & f(a_{12})\\f(a_{21}) & f(a_{22})\end{smat}\]
Questo dice due cose:
\begin{itemize}
	\item \(M(f,2)\) è definita tra \(M(A,2)\) e \(M(B,2)\), dato \(f : A \to B\); risulta perciò dalla `azione sui morfismi' di una corrispondenza che potremmo denotare \(M(\blank,2)\) o simili.
	\item \`E possibile organizzare le due possibili composizioni coincidenti in un diagramma quadrato/rettangolare,
	      \[\xymatrix{
		      M(A,2) \ar[d]_{M(f,2)}\ar[r]^-{\det_A}& A\ar[d]^f \\
		      M(B,2) \ar[r]_-{\det_B} & B
		      }\]
	      e si dice, quando i due `percorsi' che uniscono il vertice in alto a sinistra con quello in basso a destra danno luogo alla stessa funzione, che il diagramma `commuta' o `è commutativo'.
\end{itemize}
Ciò che i matematici della prima metà del ventesimo secolo iniziarono a notare è che costruzioni del genere sono tutto tranne che rare; spesso, la specifica di un problema o di una costruzione matematica porta a definire una famiglia di funzioni
\[\alpha_A : FA \to GA \]
dove \(FA,GA\) sono insiemi `costruiti in termini' di un altro insieme \(A\) (esiste una \emph{funzione} \(F(\_)\) che assegna ad \(A\) l'insieme \(FA\)).

Si può poi richiedere che quando la corrispondenza \(F : A\mapsto FA\) prende \(A\) come parametro, ed è definita in modo da indurre una funzione \(u' : FA \to FB\) per ogni \(u : A \to B\), e altrettanto è vero per la corrispondenza \(A\mapsto GA\), le varie \(\alpha_A, \alpha_B,\dots\) sono `compatibili' con le \(u\), cioè si assemblano in un quadrato%: significa che, per ogni \(u\) come sopra, la composizione di funzioni
\[
	\vcenter{\xymatrix{
			FA\ar[r]^{u'}\ar[d]_{\alpha_A} & FB\ar[d]^{\alpha_B} \\
			GA \ar[r]_{Gu} & GB
		}}
\]
commutativo; esempi di corrispondenze \(F(\_)\) e \(G(\_)\) del genere sono abbondanti e si dicono \emph{funtori}; l'insieme di tutte le \(\alpha_A,\alpha_B,\dots\) si dice una \emph{trasformazione naturale} tra due funtori.

Bisogna avere una certa esperienza nella pratica matematica per notare come una congerie di esempi particolari di questa costruzione sia astraibile a un concetto generale. Senza dubbio, anche ad un occhio allenato, questa definizione non appare immediata: quale fenomeno stiamo cercando di catturare esattamente? Cosa significa poi definire un insieme `in termini' di un altro? Come sono stati ottenuti \(FA,GA\)? Perché questa condizione è comune nella pratica? Non è chiaro quali siano degli esempi di funtore, perché abbiamo disegnato un quadrato, e non, ad esempio, un pentagono o una stella?

Sta di fatto che la nozione di \emph{naturalità} (la `dipendenza polimorfa' delle \(\alpha_A\) dal parametro \(A\)) si nasconde in ogni angolo della pratica matematica. Forse anche per questo, essa è per lungo tempo sfuggita a una formalizzazione matematica precisa,\footnote{In questo senso, Kr\"omer parla in \cite{kromer} di `esempi non problematici': i ritardi nell'introduzione di alcuni centrali concetti matematici sono talvolta dovuti al fatto che gli esempi noti sono \emph{troppo banali} per necessitare la ricerca di un concetto generale ad essi sottostante. Kr\"omer scrive:
	\begin{quote}
		\emph{[...] [Negli esempi non problematici] ciò significa che [...] il concetto generale non spiega nulla di particolare (piuttosto, è il concetto generale stesso a essere spiegato dall'esempio particolare).}
	\end{quote}}
e in effetti ha costretto Mac Lane ed Eilenberg a definire preliminarmente gli altri concetti su cui riposa: quello di \emph{funtore}, per spiegare cosa sia la corrispondenza \(F : A\mapsto FA\), e quello di \emph{categoria}, per spiegare quale tipo di struttura possiedano il dominio e il codominio di \(F\).

In effetti, non è subito chiaro quale regola\footnote{In quale senso cioè \(FA\) sia producibile mediante `operazioni elementari' a partire da \(A\): quante ne esistono, e come sono definite? Il prodotto cartesiano di insiemi è una di queste operazioni? E costruire l'insieme di tutte le funzioni \(A\to A\) dipende da \(A\) in che modo esattamente?} assegni ad \(A\) un `nuovo' insieme \(FA\), né è chiaro in quale senso esattamente la `compatibilità' di cui sopra, soddisfatta dalle \(\alpha_A\), si possa spiegare; è possibile costruire molti esempi (ne vedremo centinaia in questo libro, e molti altri sono stati omessi per motivi di spazio), ma non è chiaro quali regole ne permettano la generazione. Per così dire: per un insieme di funzioni \(\alpha_A : FA \to GA\), `quanto è naturale essere naturale'?

\medskip
Con pazienza, durante i vari capitoli di questo libro, risponderemo a queste domande essenziali, e molte altre.

\medskip
Vale la pena indulgere, ora, in una nota terminologica; le parole \emph{categoria} e \emph{funtore} sono dei prestiti storici da aree dello scibile umano antiche e piuttosto illustri. Per quanto riguarda la prima, Mac Lane prese in prestito il nome dalle categorie aristoteliche o kantiane; per Aristotele \cite{Barnes2014-wz} una \emph{categoria} è un attributo dell'essere, qualcosa che può essere affermato su di un ente; tali categorie erano dieci, per noi saranno un po' più numerose: chi legge deciderà da sé se questo è un bene o un male.

La parola \emph{funtore}, invece, è invece attestata per la prima volta nel lavoro di Tadeusz Kotarbi\'nski \cite{kotarbione}, filosofo polacco che visse quasi un secolo (\(*\)1886-\(\dag\)1981), e Rudolf Carnap \cite{carnappio}, dove viene definito come una `\emph{corrispondenza che agisce su un insieme di frasi in una grammatica, eventualmente alterando la loro struttura, ma preservando le relazioni che sussistono tra gli elementi delle frasi}' (si veda anche Curry \cite{Curry1961SomeLA} dove i funtori vengono definiti, appoggiandosi a \cite{kotarbione}, come operatori che agiscono sulle frasi del linguaggio per formarne altre; nel linguaggio giocattolo dei sami, teteli e tanteti che Curry inventa in \cite{Curry1961SomeLA} il sufisso \({\_}_1b\) e l'infisso \({\_}_1 c {\_}_2\) agiscono come funtori sulle stringhe di simboli).

\medskip
In questo preciso senso il prestito terminologico ha senso: la matematica è un linguaggio, si sente ripetere ovunque; allora un funtore è una regola che trasforma alcuni componenti fondamentali \(A,B,C,\dots\) del linguaggio \(\mathcal{L}\), eventualmente alterando la loro struttura interna, ma preservando le \emph{relazioni} che sussistono tra quelle componenti e l'esterno.
\begin{figure}
	\begin{center}
		\begin{tikzpicture}[
				% Node styles
				circle_node/.style={circle, draw, fill=white, minimum size=0.8cm, font=\small},
				square_node/.style={rectangle, draw, fill=white, minimum size=0.8cm, font=\small},
				% Domain styles
				left_domain/.style={rounded corners=8pt, inner sep=15pt, fill=gray!40, opacity=0.3},
				right_domain/.style={rounded corners=8pt, inner sep=15pt, fill=blue!30, opacity=0.3},
				% Arrow style
				mapping/.style={-latex, dashed, thick, color=gray!70}
			]

			% Left domain nodes
			\node[circle_node] (A) at (0, 0) {$A$};
			\node[circle_node] (B) at (0, 2) {$B$};
			\node[circle_node] (C) at (1.5, 1) {$C$};

			% Right domain nodes
			\node[square_node] (Ap) at (6, 1) {$A'$};
			\node[square_node] (Bp) at (5, 2) {$B'$};
			\node[square_node] (Cp) at (5, 0) {$C'$};

			% Domain boundaries
			\node[left_domain, fit=(A) (B) (C)] (left_box) {};
			\node[right_domain, fit=(Ap) (Bp) (Cp)] (right_box) {};

			% Connections within left domain
			\draw[thick, gray!60,-latex'] (A.70) -- (B.-70) node[right, font=\small, pos=.5] {$f$};
			\draw[thick, gray!60,-latex'] (B.-120) -- (A.120) node[left, font=\small, pos=.5] {$g$};
			\draw[thick, gray!60,-latex'] (A) -- (C) node[below, font=\small, pos=.6] {$h$};

			% Mappings between domains
			\draw[mapping] (B) -- (Bp);
			\draw[mapping] (C) -- (Ap);
			\draw[mapping] (A) -- (Cp);

			% Connection within right domain
			\draw[thick, blue!60,-latex'] (Bp) -- (Cp) node[pos=.3,left, font=\small] {$v$};
			\draw[thick, blue!60,-latex'] (Cp.east) -| (Ap.south) node[pos=.5,right, font=\small] {$u$};

		\end{tikzpicture}
	\end{center}
	\caption{La teoria delle categorie come teoria di \emph{sistemi e processi}: le relazioni all'interno di un dato ambito linguistico si rappresentano mediante lati di un grafo diretto, e un \emph{funtore} preserva (la direzione e) la \emph{componibilità in serie} di questi lati.}
\end{figure}
Le componenti \(A,B,C,\dots\) però ora non sono parte del discorso in un linguaggio naturale; sono `elementi' di una collezione --spesso gigantesca: \emph{tutti} i gruppi, \emph{tutte} le varietà differenziabili, senza eccezione-- \(\mathcal{L}\), che si chiamerà una `categoria'.

Carnap definisce un funtore come una relazione funzionale tra linguaggi, che `preserva l'analisi logica'; per noi, un funtore è una relazione funzionale tra linguaggi, che preserva, in qualche modo da determinare, le relazioni che sussistono tra le loro diverse parti. Queste relazioni tra membri di una classe sono gli \emph{omomorfismi} del dato tipo di struttura in studio, e la possibilità di rendere esplicita la natura di queste relazioni, così come la possibilità di trasformare relazioni in relazioni preservando la connessione tra enti relati, è un'idea al cuore della teoria delle categorie.

La nozione data da Mac Lane ed Eilenberg in \cite{gtone} stava quindi ponendo le basi per un approccio \emph{relazionale}, dinamico, alla costruzione e alla comprensione degli oggetti matematici: le strutture che la matematica studia, già secondo l'idea che fu di Poincaré mezzo secolo prima,\footnote{H. Poincaré, il nonno della teoria delle categorie, disse che `la matematica non è solo un insieme di teoremi, così come una casa non è solo un mucchio di mattoni'. In maniera un po' più poetica (si veda il testo di Cook \cite{Cook1977-ry}) nella mitologia vedica, quando Indra immagina il mondo lo costruisce come una ragnatela o rete, in ciascuno dei cui nodi viene incastonato un gioiello; ogni \emph{dharma}, cioè ogni concetto sensibile o immaginabile, passato presente o futuro, è un nodo in questa rete, e la superficie di ciascun gioiello riflette ogni altro, cosicché ogni cosa che esiste implica tutte le altre (secondo il principio detto \emph{pratītyasamutpāda}, letteralmente traducibile come \emph{mutua produzione condizionata}).} non esistono in un vuoto, ma immerse in una rete di relazioni, separate dalle quali esse sono incomprensibili, o comprensibili con maggiore fatica intellettuale.

\medskip
Lo scopo di questa introduzione non è però delineare una storia completa della teoria delle categorie; chi legge troverà nel libro di Kr\"omer \cite{kromer} on nel lavoro di Marquis \cite{marquis} esempi già insuperabili, che sono pienamente testi \emph{di matematica}. Chi legge apprenderà lì che la teoria delle categorie ha molti precursori, che si possono far risalire molto più indietro nel tempo del 1945, anno della pubblicazione di \cite{gtone}. Un esempio su tutti, il cosiddetto \emph{programma di Erlangen} proposto da Klein in una famosa prolusione del 1872, \cite{Klein1893}.

Nel momento in cui questo testo viene scritto, la teoria delle categorie è uno strumento ubiquitario per comprendere la logica, l'algebra, la geometria in senso moderno, e ciò che è stata a partire dal secondo dopoguerra è in larga parte una \emph{conseguenza} dell'adozione di questo linguaggio. Questa transizione, ed evoluzione, è inestricabile da una storia della matematica che è vasta e ancora in via di definizione; delinearla non compete certamente a chi la sta vivendo.

Il punto che vogliamo rendere esplicito qui è solo che parlare della teoria delle categorie come una disciplina eminentemente astratta, barricata in un formalismo astruso, che mal tollera motivazioni concrete tradisce \emph{completamente} la verità storica --tanto più che tali motivazioni sono sempre radicate nell'esperienza sensibile, solo talvolta un po' meno direttamente. Quando non occupata da faccende estremamente più prosaiche, la prima metà del ventesimo secolo ha intuito una unità fondamentale dietro le strutture della matematica, e cioè che ragionare per analogia animati dal desiderio di rivelare le dinamiche generali di una teoria, è una inesauribile fonte di intuizione. Si pensi, a mero titolo di esempio, alle congetture di Weil \cite{PMIHES_1974__43__273_0,PMIHES_1980__52__137_0}, considerate l'apice del sincretismo matematico, ponendo problemi che uniscono teoria dei numeri, geometria, analisi complessa e topologia; una visione unificata, e la conquista di queste profonde congetture, è stata possibile \emph{anche} grazie al linguaggio delle categorie.

Perciò, all'esatto contrario del pregiudizio comune, lo spirito che ha animato la teoria delle categorie è quello di una comunità di matematici che hanno tentato indefessamente di spiegare in termini di pochi concetti essenziali la natura e il comportamento di costruzioni matematiche che a prima vista erano del tutto scorrelate tra loro: è stata edificata secondo scelte stilistiche che nella loro apparente ingenuità (`non importano gli enti in studio, bensì le maniere in cui essi si relazionano'; `alcuni oggetti matematici sono determinati da una proprietà che li rende unici'; `la trasformazione più importante di tutte è quella che lascia tutto al suo posto') hanno dato numerosi frutti.

\medskip
Senza nessuna pretesa di completezza o di autorità, proviamo a descrivere quali sono queste idee.
\begin{itemize}
	\item Gli assiomi che fondano una teoria devono essere pochi e ben motivati, vuoi dall'esperienza sensibile, vuoi dal numero elevato di esempi vantaggiosi che \emph{quegli} assiomi, e non altri, riescono a descrivere. Al di fuori della teoria delle categorie, la teoria della misura è un esempio relativamente buono di questo tipo di ragione.\footnote{Un famoso teorico delle categorie, J. Bénabou, produce un `cattivo esempio' che riportiamo senza pretesa di essere letterali (e soprattutto senza voler offendere nessuno): per costruire l'insieme delle coppie ordinate \((a,b)\) a partire da due insiemi \(A,B\) è necessario, formalmente parlando, considerare l'insieme \(2^{2^{A\cup B}}\), per poi restringere il discorso alle coppie di una certa forma ben precisa: la coppia ordinata \((a,b)\) consta dell'insieme \(\{\{a\},\{a,b\}\}\). In particolare, per considerare il prodotto cartesiano \(\mathbf{N} \times \mathbf{N}\) di due copie dei naturali (insieme di cui \emph{certamente} vogliamo essere in grado di parlare), va considerato l'insieme \(2^{2^{\mathbf{N} \cup \mathbf{N}}}\): insieme che è gigantesco, e Bénabou definisce `aberrante' la pratica di doverlo considerare per parlare di un oggetto tanto semplice quanto quello che contiene le coppie \((15,18), (3,7), (12, 259)\dots\) In teoria delle categorie, invece, il prodotto cartesiano \(A\times B\) è definito in maniera molto più snella, e non meno rigorosa.}
	\item La matematica deve essere ispirata a un principio di ergonomia e modularità; deve essere relativamente semplice e intuitivo maneggiare gli enti che compongono una teoria, e deve essere chiaro come poter esportare alcuni suoi frammenti a un contesto diverso: che differenza c'è, in ultima istanza, tra un monoide, un anello, e un gruppo dove le operazioni di moltiplicazione e inversione sono continue rispetto a una topologia? In questo senso, la matematica deve essere ispirata a un canone simile a quello che orienta la scrittura di `buon' codice sorgente quando si programma. Tutti sono capaci di scrivere una funzione che calcola un fattoriale; già meno persone sono capaci di farlo tenendo d'occhio il costo computazionale, la leggibilità, la mantenibilità della teoria/libreria dove quel teorema/codice è immerso.
	\item Quando teorie diverse possiedono dei tratti comuni, esiste una spiegazione profonda, non accidentale, per questo. Lo scopo --o l'effetto-- di una parte piuttosto vasta di teoria delle categorie è stato di trovare questa spiegazione profonda, renderla evidente e cercare di portarla alle sue estreme conseguenze. Le applicazioni maggiori di questo principio si possono apprezzare in `discipline dalla natura altamente dialettica' (una locuzione rubata a \cite{lawvere1999profilo}, una lettura squisita che invitiamo chi legge a reperire ad ogni costo), come la geometria e la logica; ma abbondano anche gli esempi in algebra astratta (vedremo questa idea in azione proprio quando cercheremo di `spiegare' il motivo per cui tutti i teoremi di isomorfismo si somigliano tra loro).
	      % \item La matematica possiede un certo grado di auto-referenzialità: le teorie matematiche si possono apprezzare come un certo tipo particolare di oggetto matematico, che può essere compreso mediante il linguaggio matematico. Lungi dall'essere una fumosa affermazione filosofica, questa idea viene sostanziata mediante il linguaggio delle categorie.
\end{itemize}
L'opinione di noi che scriviamo è che questo punto di vista, al di là dei suoi meriti concreti, misurabili, confermi come lo scibile matematico sia alla portata di chiunque lo voglia cogliere, quando esso sia espresso in termini di pochi concetti fondamentali, con lo scopo di delineare cosa, all'interno delle verità del linguaggio, è una tautologia, e concentrare le proprie energie sul comprendere appieno ciò che non lo è.

Chi studia paga un prezzo all'ingresso, perché non deve solo imparare definizioni e teoremi e tecniche di calcolo, ma soprattutto un modo di pensare diversamente (e ripensare, e ripensare ancora, e continuamente mettere in discussione) cosa è la matematica nella sua totalità, e al contempo \emph{disimparare} alcune cattive abitudini impartite dall'insegnamento. Paradossalmente, la teoria delle categorie è in un senso preciso `elementare', si fa a partire da pochissime idee, ma suona incomprensibile all'inizio: pare completamente tautologica, o completamente oscura, senza vie di mezzo. Dopo un po' di ostinazione (che non si può trasmettere all'apprendista, ma solo coltivare) ci si accorge però che i rudimenti del linguaggio categoriale rendono più semplice, veloce ed efficiente apprezzare la nascosta somiglianza tra definizioni molto diverse tra loro, nate per risolvere problemi diversi, e sviluppate in dialetti diversi. In una stanza occupata da dieci matematici, e nove parlano di cose in apparenza tra loro sconnesse, se chi rimane sa la teoria delle categorie può \emph{forse} mostrare un filo conduttore comune ai discorsi degli altri.

Frequentare la matematica con questa ambizione ha un costo cognitivo non indifferente: una indole poliedrica, e poco affine alla specializzazione, aiutano ad apprendere lo spirito dietro molte definizioni, perché esse vengono dall'intuizione topologica, dall'algebra astratta, dalla logica; ma questa indole, senza l'abilità di risolvere problemi, rende impossibile \emph{calcolare} adoperando le definizioni apprese. Per imparare la teoria delle categorie si deve perciò fare un po' della matematica \emph{di tutti gli altri tipi} e trovare, in quest'ultima, le analogie che gli altri non hanno tempo, sensibilità o vocazione per disvelare.

\medskip
Lo scopo del libro che avete in mano è anche colmare l'abisso che c'è tra idee così profonde da un lato, che hanno ispirazione filosofica, evocative, seducenti per la loro generalità, e dall'altro la natura eminentemente pratica (pedissequa, e quasi noiosa) del ragionamento matematico che procede a passi minuscoli tautologia dopo tautologia. La teoria delle categorie si fa con la matematica; perciò è fatta di teoremi, con ipotesi e un ambito di validità precisi, che si dimostrano attingendo a un corpo di conoscenze `esatto'.

Ben più di certa cattiva divulgazione, che pretende di semplificare ciò che semplice non è, per mostrare che è al livello di chiunque, noi crediamo che si debba nobilitare chi ascolta, non semplificare la verità. Pensiamo questo approccio riesca a mostrare che la matematica è `alla portata di tutti', e secondo noi colpendo certamente più al cuore di altri tentativi.

\medskip
Il libro nasce però anche con un fine molto più immediato: accompagnare un corso di teoria delle categorie che, durante l'anno accademico 2021-22 si è svolto online, su youtube, tenuto dal gruppo ItaCa (\url{https://progetto-itaca.github.io/pages/course.html}). Gli autori del libro sono i docenti di quel corso a cui abbiamo deciso di dare il nome collettivo di \emph{Outis} (\(O \acute\upsilon\tau\iota\varsigma\), Nessuno, è il nome con cui Ulisse ingannò Polifemo; ci piaceva, o meglio piaceva a uno degli autori, che questo libro fosse stato scritto da Nessuno).\footnote{Non menzionare nemmeno una volta i loro nomi sarebbe però irrispettoso; appaiono qui, quasi invisibili, una volta sola:
	Greta Coraglia, Jacopo Emmenegger, Enrico Ghiorzi, Francesca Guffanti, Fosco Loregian, Beppe Metere, Daniele Palombi, Paolo Perrone, Enrico Vitale. Un gran numero di persone ha poi collaborato in maniera infinitesimale, ma non meno importante, alla qualità del testo, proponendo correzioni, suggerendo idee, leggendo il numero infinito di bozze, costruendo l'infrastruttura per versionare il testo\dots{} Li ringraziamo tutti e tutte.}

Ci ha spinto a impegnarci in questa ulteriore fatica il fatto che, con poche eccezioni, agli atenei italiani manca un corso il cui argomento centrale sia la teoria delle categorie; abbondano i corsi che la derubricano a strumento, spesso della geometria o dell'algebra; ma ne danno una visione parziale introducendo in tutta fretta le costruzioni categoriali necessarie alla vorace algebra e geometria moderna.

Manca, per contro, un corso che `riordini' le nozioni matematiche di chi studia, unificandole sotto i pochi e fondamentali concetti della matematica strutturale. Questo è uno degli scopi concreto del libro che avete in mano.

Esso si rivolge primariamente a quattro tipi di persone:
\begin{itemize}
	\item \`E pensato in primo luogo per chi studia in una laurea in matematica e ha sentito parlare vagamente della teoria delle categorie; per svariati motivi, queste persone sono curiose di saperne un po' di più. Per loro questo libro deve essere di lettura facile e allo stesso tempo motivante riguardo alle idee fondamentali, affinche dia un'immagine fedele e accessibile della teoria, e invogli ad approfondire.
	\item In secondo luogo, qualcuno ad un livello anche più avanzato e con interessi diversi (altri rami della matematica o discipline `consumatrici' di matematica) che hanno sentito dire che la teoria delle categorie pu\`o aiutare nelle loro discipline di predilezione. Chi sta iniziando un dottorato in geometria algebrica userà, molte volte senza nemmeno saperlo, un piccolo risultato detto `lemma di Yoneda'\dots
	\item Esiste poi un pubblico pi\`u specializzato, desideroso di trovare degli esempi raffinati o di rinfrescarsi la memoria su qualche nozione di base, o ancora di leggere qualcosa di coerente perché ha informazioni parziali, mai organizzate sistematicamente, non soddisfacenti, su certe nozioni di teoria delle categorie che gli sono servite e a cui ha `fatto l'abitudine' (nel senso precisato dalla famosa citazione di Von Neumann: in matematica non si capiscono le cose\dots).
	\item Da ultimo, questo libro è scritto pensando alla comunità italiana di teoria delle categorie, che la usa nella `vita (matematica) di tutti i giorni', e che vuole adottare un testo per insegnare la materia, o per approfondirla nelle parti meno note. La teoria delle categorie usa infatti idee simili quando applicata in logica, in algebra, in geometria; è però raro che chi fa matematica per professione conosca le ramificazioni del linguaggio in discipline lontane dalla sua: il tempo basta a malapena per diventare esperti di una di queste. Questo libro è anche per loro, ed è un nostro desiderio renderlo un testo con cui si può, se serve, \emph{imparare e insegnare}. Un libro su cui magari anche sbattere la testa, soffrire (il meno possibile), per capire qualcosa di nuovo, e su cui tornare, da matematici maturi, continuando ogni volta a notare qualcosa che prima era sfuggito.
\end{itemize}
\Todo{}

\medskip
Come è strutturato, quindi, questo libro? E' innanzitutto suddiviso in capitoli, che rispecchiano la struttura del corso di YouTube ma lo espandono largamente (ad esempio, si insiste sul dare tantissimi esempi di categorie rappresentando quante più aree della matematica è possibile; vi sono costruzioni nel libro che non troverete nelle videolezioni; il \autoref{cap_fattorizzazione} non è presente nel corso, eccetera); le sezioni di ogni capitolo si concludono con degli esercizi per chi legge, solitamente a gruppi di 3-5. Gli esercizi sono risolti alla fine del libro, in una appendice; la ragione per questa insistenza, che sarà una lettura pesante per chiunque abbia esperienza, è che dopo i primi anni di studio della matematica, raramente si impiega un po' di tempo per mostrare agli studenti come si fanno gli esercizi, quali siano le tecniche che si sanno essere valide per fare una dimostrazione. Il linguaggio delle categorie è totalmente nuovo per tante persone, e non avere questo tipo di conferma della correttezza dei propri argomenti è assai frustrante. Come posso fare a sapere che il mio esercizio è giusto, senza un termine di paragone con la soluzione di un altra persona?

Chi legge si accorgerà vedendo le soluzioni degli esercizi (ma dovrebbe farlo solo dopo aver provato a risolverli!) che esistono alcune tattiche di dimostrazione stabilite, che possono essere descritte in poche parole e che hanno un ampio campo di validità. Queste tecniche spesso si aiutano l'un l'altra.

Alcuni paragrafi lungo il testo sono decorati da delle `manine':
\begin{itemize}
	\item \hands{fund} (\emph{fermarsi a pensare}) indica un esempio, osservazione o nozione fondamentale, che non va evitata a una prima lettura, pena il non capire qualcosa di lì a poco;
	\item \hands{skip} (\emph{tagliare corto qui}) indica un paragrafo che invece è possibile saltare senza inficiare la comprensione del testo circostante. Un esempio molto settoriale, una digressione terminologica (solitamente intitolata `A proposito del nome\dots');
	\item \hands{tech} (\emph{lunga vita e prosperità a chi si ferma a leggere}) può essere un esempio abbastanza tecnico, inusuale, che va considerato come una curiosità o una piccola sfida per l'immaginazione di chi legge. Non è essenziale ad altro che ad ampliare il panorama di chi legge.
\end{itemize}
Una prima appendice si occupa di fondamenti, cioè della (meta)teoria degli insiemi e delle classi in cui fare teoria delle categorie. Viene introdotto il linguaggio di classi e insiemi, e la gerarchia degli universi; vengono descritti quelli che si chiamano `problemi di taglia'.\footnote{Un `problema di taglia' in matematica si può forse apprezzare con una analogia, che lasciamo nella forma di un indovinello irrisolto: si immagini di dover cucinare seguendo una ricetta, ma di non poterla applicare perché nessuna ciotola che si ha in cucina basta a contenere gli ingredienti, che sono tutti troppo grandi. Come risolvere questo problema?} Quasi nessuno si preoccupa di queste sottigliezze, considerando la teoria degli insiemi e le sue patologie solo un intralcio a fare matematica `vera'; ci sono però alcuni casi in cui le patologie non possono essere evitate, ed è dovere di chi vuole fare il proprio lavoro correttamente conoscere quei pochi casi.

Una seconda appendice raccoglie le soluzioni degli esercizi; vi sono infiniti modi di risolvere un esercizio (e ancora più modi di sbagliarlo), per cui queste soluzioni non vogliono avere alcun valore prescrittivo. Manca però al panorama della matematica italiana corrente un riferimento che insegni a chi studia a \emph{risolvere problemi di teoria delle categorie}: per farlo, è essenziale vedere come si usano le definizioni e i teoremi, perché chi potrebbe mai comprendere il calcolo differenziale senza mai aver visto la maniera in cui si calcola un'area?

Una terza appendice è fatta di sezioni, ciascuna delle quali presenta dei complementi che sono stati tralasciati nei vari capitoli, perché troppo tecnici o settoriali.\Todo{}

\medskip
Concludiamo questa introduzione parlando, finalmente, di matematica in senso stretto.

Lo scopo delle brevi sezioni che seguono e che chiudono il capitolo è di presentare quattro `preludi' categoriali, raccolti dai vari àmbiti del bagaglio culturale di uno studente che è alla fine di una laurea triennale in una generica università italiana. Si parla di semplici proprietà dei gruppi abeliani, della costruzione dei numeri reali, di topologia. Procedendo nella lettura, vi accorgerete che la teoria delle categorie `salta' con facilità da una parte della matematica all'altra, tentando di trovare dei temi e delle strategie di computo comuni alle diverse discipline.

L'accento, per ora, non è sul rigore, ma sul giusto grado di espressività e generalità.
\section*{Abelianizzazione, o `naturalità'}
Consideriamo un gruppo \(G\), la cui operazione è denotata moltiplicativamente; in esso, consideriamo il sottogruppo generato dagli elementi della forma
\[[x,y]:= xyx^{-1}y^{-1}\]
al variare di \(x,y\in G\). Si tratta quindi del sottogruppo i cui elementi sono generati dai `commutatori' della forma
\[t = [x_1,y_1][x_2,y_2]\cdots[x_n,y_n]\]
al variare di \(x_i,y_i\in G\) e \(n\ge 0\) in \(\mathbf{N}\) (se \(n=0\), il prodotto è vuoto e quindi l'elemento in questione è uguale all'identità di \(G\)): è infatti facile vedere che l'inverso \([x,y]^{-1}\) di un commutatore è a sua volta un commutatore.

\`E altresì facile mostrare che il sottogruppo \([G,G]\) è normale in \(G\), e che è il sottogruppo minimale con la proprietà che il quoziente \(G/[G,G]\) è abeliano (cioè, se \(N\) è normale e \(G/N\) è abeliano, allora \(N\) contiene \([G,G]\)).

Il gruppo \(G/[G,G]\) prende il nome di \emph{abelianizzato} di \(G\), si denota a volte con \(G^\text{a}\), e soddisfa la seguente proprietà di \emph{universalità}:
\begin{quote}
	Esiste un unico omomorfismo \(\alpha : G \to G^\text{a}\) con la seguente `proprietà universale': per ogni omomorfismo di gruppi \(f : G \to A\), di codominio un gruppo abeliano \(A\), esiste un unico omomorfismo di gruppi \(\bar f : G^\text{a} \to A\) con la proprietà che \(\bar f \cmp \alpha = f\).
\end{quote}
\begin{remark}
	Si può pensare a \(G\mapsto G^\text{a}\) come ad una `costruzione' che ad un gruppo \(G\) associa un altro gruppo \(G^\text{a}\), definito da una certa proprietà; questa associazione è poi responsiva al fatto che tra gruppi distinti \(G,H\) possono esistere degli omomorfismi \(f : G \to H\), nel senso che segue:
	\begin{quote}
		Dato un omomorfismo di gruppi \(f : G \to H\) esiste un omomorfismo \(f^\text{a} : G^\text{a} \to H^\text{a}\) tra gli abelianizzati di \(G,H\), definito mandando una classe di equivalenza \(g\cdot [G,G]\) in \(f(g)\cdot [H,H]\).
	\end{quote}
	L'unica cosa da verificare è che \(f^\text{a}\) così definita sia veramente una funzione; del resto, \(f\) discende al quoziente \(G^\text{a}\) partendo da una funzione \(\tilde f : G\to H^\text{a}\), cosa che segue immediatamente dal fatto che \(f\) manda \([G,G]\) in \([H,H]\) (perché \(f[x,y]=[fx, fy]\in [H,H]\)).
\end{remark}
Vi sono due proprietà che ora la corrispondenza \(f\mapsto f^\text{a}\) soddisfa: la loro verifica è a dir poco immediata.
\begin{enumtag}{fu}
	\item \label{fct_1} Se \(1_G\) è l'omomorfismo identico di un gruppo \(G\), allora \(1_G^\text{a}\) è l'omomorfismo identico di \(G^\text{a}\).
	\item \label{fct_2} Considerando due omomorfismi di gruppi componibili \(u : G\to H, v : H\to K\), si ha che
	\[(v\cmp u)^\text{a} = v^\text{a} \cmp u^\text{a}\]
	(l'uguaglianza di funzioni è una uguaglianza \emph{estensionale}, cioè valida elemento per elemento).
\end{enumtag}
Nelle stesse notazioni, si osservi anche che la definizione di \(f^\text{a} : G^\text{a} \to H^\text{a}\) è l'unica possibile qualora si chieda che \(f^\text{a}(\pi^G(x)) = \pi^H(f(x))\) per ogni \(x\in G\), ossia che il diagramma di omomorfismi
\[\xymatrix{
	G \ar[r]^-{\pi^G} \ar[d]_f & G^\text{a}\ar[d]^{f^\text{a}} \\
	H \ar[r]_-{\pi^H} & H^\text{a}
	}\]
sia commutativo.

La famiglia di omomorfismi \(\{\pi^G : G\to G^\text{a}\}\) è quindi specificata in maniera `uniforme' nel parametro da cui dipende, ossia è determinata, per un qualsiasi gruppo \(G\), dalla proiezione al quoziente \(\pi^G : G\to G/[G,G] : x\mapsto x\cdot[G,G]\) che manda un dato elemento nella sua classe laterale. Per \emph{ogni} gruppo \(G\) è possibile costruire una funzione \(\pi^G : G\to G/[G,G]\), definita sempre alla stessa maniera trattando \(G\) come un parametro, e tale che il quadrato precedente sia commutativo. Questa costruzione rende \(\pi\) quella che si chiama una \emph{famiglia polimorfa}: per ogni gruppo \(G\), si dà una funzione \(\pi^G : G\to G/[G,G]\).

In situazioni simili, diciamo che l'assegnazione \(G\mapsto (\pi^G : G\to G^\text{a})\) è \emph{naturale}, o più precisamente che \(\pi^G\) è (la componente di) una \emph{trasformazione naturale}, rappresentata come una freccia dove il parametro \(G\) è stato astratto:
\[\text{da}\quad\forall G : \pi^G : G\to G^\text{a} \quad \text{ si passa a }
	\quad\pi : \id \nat (\_)^\text{a}.\]
% \[\xymatrix{\boldsymbol\pi  : (\_) \ar@{=>}[r] & (\_)^\text{a}.}\]
Il dominio e il codominio di una trasformazione naturale sono dei \emph{funtori}: non daremo qui la definizione (che seguirà nella sezione \ref{sec_funtori}), ma l'idea che chi legge dovrebbe trattenere è che sappiamo trasformare un gruppo \(G\) in un altro gruppo \(F(G)\), e ogni omomorfismo di gruppi \(f : G\to H\) in un omomorfismo \(F(f) : F(G)\to F(H)\), in modo tale che le condizioni \ref{fct_1} e \ref{fct_2} siano verificate: quindi,
\begin{enumerate}
	\item Se \(1_G\) è l'omomorfismo identico, allora \(F(1_G)\) è l'omomorfismo identico di \(F(G)\).
	\item Considerando due omomorfismi componibili \(u : G\to H, v : H\to K\), si ha che
	      \[F(v\cmp u) = Fv \cmp Fu.\]
\end{enumerate}
In questo caso specifico, il dominio di \(\boldsymbol\pi\) è il funtore \emph{identico} definito in modo tautologico; il codominio è il funtore di abelianizzazione, la cui funtorialità è chiara grazie a \ref{fct_1} e \ref{fct_2}.
% \begin{remark}
% 	L'idea intuitiva che questa sezione vuole comunicare è che tutte le proprietà matematiche di una certa importanza si possono rifrasare nello stesso modo; la costruzione che produce \(G^\text{a}\) è un esempio particolare di una pratica generale, quella di definire un oggetto matematico mediante una certa \emph{proprietà universale}, che cioè somigli alla proprietà di \(G^\text{a}\).

% 	Siamo posti di fronte al problema di costruire un oggetto con certe proprietà. Se (è possibile trovarlo e) una volta che lo si è costruito esso soddisfa un requisito di universalità (che superficialmente cambia di volta in volta, ma che a conti fatti chiede sempre la stessa cosa: per ogni `diagramma' della tal forma, esiste un \emph{unico} omomorfismo della tale altra forma, per cui\dots), esso è \emph{univocamente determinato} da questa proprietà.
% \end{remark}
% \`E ora conveniente pensare agli omomorfismi di gruppo --e in effetti, relativi a qualsiasi altra struttura: i gruppi non hanno niente di speciale qui-- come `deformazioni' di una struttura di tipo \(G\) in una di tipo \(H\). In questo senso, l'assegnazione \(G\mapsto G^\text{a}\) è speciale in due sensi: prima di tutto, ogni gruppo \(G\) ha un omomorfismo canonico \(\pi_G : G \to G^\text{a}\) di proiezione al quoziente; secondo, per ogni \(f : G \to H\) le condizioni sopra sono verificate.
\section*{Teoremi di isomorfismo, o `universalità'}
I teoremi di isomorfismo nelle diverse strutture dicono tutti la stessa cosa
\section*{Completamenti, o `aggiunzioni'}
Dato uno spazio metrico \((X,d)\) una \emph{successione di Cauchy} è una successione \(a : \bbN \to X\) tale che per ogni \(\epsilon>0\) esiste un \(N\gg 1\) per cui
\[\forall n,m> N\; : \; d(a_n,a_m) < \epsilon.\]
Dato \((X,d)\) è sempre possibile costruire uno spazio metrico `completo' \(\bar X\) che contiene \(X\) come un sottospazio isometrico e tale che ogni successione \(a : \bbN \to \bar X\) che sia di Cauchy è convergente.

Questo spazio soddisfa la seguente proprietà: comunque sia dato uno spazio metrico completo \((Y,d_Y)\) e una mappa nonespansiva \(f : X\to Y\), esiste un'unico modo di estendere \(f\) a una mappa nonespansiva \(f^* : \bar X \to Y\) che coincide con \(f\) su \(X\le \bar X\).

Per costruire \(\bar X\), consideriamo lo spazio \((C^0(X,\bbR),d_\infty)\) delle funzioni continue \(X\to\bbR\), dotato della metrica uniforme \((f,g)\mapsto \sup_x |fx-gx|\), che è facile mostrare è uno spazio metrico completo. Allora l'embedding \(j : X\mono \bar X\) è dato dalla funzione che manda \(x\) in \(\bar x = d(x,-) : X\to \bbR\), la quale è continua (e iniettiva, dato che la metrica è non degenere); dalla disuguaglianza triangolare segue che \(\sup_z|\bar{x}(z)-\bar{y}(z)|\le d(x,y)\) e in effetti vale l'uguaglianza, ponendo \(z=x\); ma allora, \(j : X\mono \bar X\) è una isometria.

Lo spazio così costruito
\section*{Il teorema di Brouwer, o `funtorialità'}
Dimostrare una cosa difficile `muovendo le mani'

