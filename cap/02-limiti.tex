\chapter{Limiti e colimiti}\label{chap_limiti_colimiti}
% Limiti e colimiti.
\section{La nozione di proprietà universale}
\begin{esercizi}
	\item \item \item \item \item
\end{esercizi}
\subsection{Coni e coconi}
\subsection{Forme di co/limiti}
\section{Esempi di colimiti nella pratica matematica}
\begin{esercizi}
	\item \item \item \item \item
\end{esercizi}
\section{Teoria dei co/limiti}
\begin{esercizi}
	\item \item \item \item \item
\end{esercizi}
\subsection{Esistenza di co/limiti}
\subsection{Ostruzione all'esistenza di co/limiti}
\section{Preservazione, riflessione, creazione di co/limiti}
\begin{esercizi}
	\item \item \item \item \item
\end{esercizi}
% Una categoria è completa se ha prodotti ed equalizzatori.
% Una categoria è completa se ha terminale e wide pullbacks.
% Esempi (ad esempio i prodotti in un poset, l’assenza di prodotti in Fld, oggetto
% dei numeri naturali di Fosco).
% Funtori che preservano/riflettono/creano i limiti. Una categoria
% supermegacompleta è un poset (v. Freyd).
