\section{Trasformazioni naturali}\label{sec_tnat}
\fosco{L'esempio che, storicamente, ha motivato la definizione di `trasformazione naturale' tra funtori ha ben poco di pedagogico o elementare, ma vale la pena ricordarlo lo stesso; la domanda è: qui o altrove?}

\paolo{Sarebbe bello connettere questa sezione all'introduzione spiegando in che senso formalizza l'idea di naturalità (almeno in certi casi).}
In \ref{tante_cose_sono_diag} abbiamo osservato che le spanne e cospanne in \(\ctC\) si possono vedere come funtori; del resto, in \ref{def_span_e_cospan} abbiamo costruito una \emph{categoria} di spanne in \(\ctC\). Un omomorfismo di spanne consiste, secondo quella definizione, di una famiglia di tre morfismi di \(\ctC\) tra i vertici omonimi delle due spanne \((X_1\overset l\leftarrow X_0\overset r\to X_2)\) e \((Y_1\overset{l'}\leftarrow Y_0\overset{r'}\to Y_2)\):
\[\alpha_0 : X_0\to Y_0,\quad \alpha_1 : X_1\to Y_1,\quad \alpha_2 : X_2\to Y_2\]
che rendono commutativi i diagrammi in \ref{mor_spanne}. Riguardando le spanne \(X_\bullet,Y_\bullet\) come diagrammi di forma \(\Lambda^2_0 \to\ctC\), la terna \(\underline\alpha = (\alpha_0,\alpha_1,\alpha_2)\) si può vedere come una freccia \(X_\bullet \nat Y_\bullet\), nel senso seguente:
\begin{itemize}
	\item per ogni spanna \((X_1\overset l\leftarrow X_0\overset r\to X_2)\) possiamo considerare la terna \(\underline{\id}=(\id_0 : X_0\to X_0,\id_1 : X_1\to X_1,\id_2 : X_2\to X_2)\);
	\item date \(\underline\alpha=(\alpha_0,\alpha_1,\alpha_2),\underline\beta=(\beta_0,\beta_1,\beta_2)\) che formano due distinti morfismi tra spanne, ne esiste una composta \(\underline\beta\cmp\underline\alpha=(\beta_0\cmp\alpha_0,\beta_1\cmp\alpha_1,\beta_2\cmp\alpha_2)\) che è a sua volta un omomorfismo di spanne, dato che l'esterno del diagramma
	      % https://q.uiver.app/#q=WzAsMTIsWzEsMCwiWF8xIl0sWzIsMCwiWF8wIl0sWzMsMCwiWF8yIl0sWzEsMSwiWV8xIl0sWzIsMSwiWV8wIl0sWzMsMSwiWV8yIl0sWzEsMiwiWl8xIl0sWzIsMiwiWl8wIl0sWzMsMiwiWl8yIl0sWzAsMCwiWDoiXSxbMCwxLCJZOiJdLFswLDIsIlo6Il0sWzEsMCwibCIsMl0sWzEsMiwiciJdLFs0LDUsInInIiwyXSxbNCwzLCJsJyJdLFs3LDYsImwnJyJdLFs3LDgsInInJyIsMl0sWzMsNiwiXFxiZXRhXzEiLDJdLFs0LDcsIlxcYmV0YV8wIiwyXSxbNSw4LCJcXGJldGFfMiJdLFswLDMsIlxcYWxwaGFfMSIsMl0sWzEsNCwiXFxhbHBoYV8wIiwyXSxbMiw1LCJcXGFscGhhXzIiXSxbOSwxMCwiXFx1bmRlcmxpbmVcXGFscGhhIiwyLHsib2Zmc2V0IjoxLCJsZXZlbCI6Mn1dLFsxMCwxMSwiXFx1bmRlcmxpbmVcXGJldGEiLDIseyJvZmZzZXQiOjEsImxldmVsIjoyfV1d
	      \[\begin{tikzcd}
			      {X:} & {X_1} & {X_0} & {X_2} \\
			      {Y:} & {Y_1} & {Y_0} & {Y_2} \\
			      {Z:} & {Z_1} & {Z_0} & {Z_2}
			      \arrow["{\underline\alpha}"', shift right, Rightarrow, from=1-1, to=2-1]
			      \arrow["{\alpha_1}"', from=1-2, to=2-2]
			      \arrow["l"', from=1-3, to=1-2]
			      \arrow["r", from=1-3, to=1-4]
			      \arrow["{\alpha_0}"', from=1-3, to=2-3]
			      \arrow["{\alpha_2}", from=1-4, to=2-4]
			      \arrow["{\underline\beta}"', shift right, Rightarrow, from=2-1, to=3-1]
			      \arrow["{\beta_1}"', from=2-2, to=3-2]
			      \arrow["{l'}", from=2-3, to=2-2]
			      \arrow["{r'}"', from=2-3, to=2-4]
			      \arrow["{\beta_0}"', from=2-3, to=3-3]
			      \arrow["{\beta_2}", from=2-4, to=3-4]
			      \arrow["{l''}", from=3-3, to=3-2]
			      \arrow["{r''}"', from=3-3, to=3-4]
		      \end{tikzcd}\]
	      è commutativo.
\end{itemize}
Ciò significa che, per ogni categoria \(\ctC\), esiste sulla classe dei funtori \(\Lambda^2_0\to\ctC\) una struttura di categoria.

La nozione di \emph{trasformazione naturale} tra funtori rende precisa questa idea.
\begin{definition}[Trasformazione naturale]\label{def_nat}\index{Trasformazione naturale}\index{Naturale!trasformazione}\index{Naturalità}
	Date due categorie \(\ctC\) e \(\ctD\) e dati due funtori \(F,G:\ctC\to\ctD\), una \emph{trasformazione naturale} \(\alpha:F\Rightarrow G\) consiste dei seguenti dati:
	\begin{itemize}
		\item Per ogni oggetto \(X\) di \(\ctC\), una freccia \(\alpha_X:FX\to GX\) in \(\ctD\) detto la \emph{componente di \(\alpha\) ad \(X\)};
		\item Per ogni morfismo \(f:X\to Y\) di \(\ctD\), il seguente diagramma deve commutare (\emph{condizione di naturalità}):
		      \[
			      \begin{tikzcd}
				      FX \ar{d}{Ff} \ar{r}{\alpha_X} & GX \ar{d}{Gf} \\
				      FY \ar{r}{\alpha_Y} & GY.
			      \end{tikzcd}
		      \]
	\end{itemize}
\end{definition}
\begin{definition}[Composizione verticale di trasformazioni naturali]\label{def_vcomp}\index{Composizione verticale!di trasformazioni naturali}
	Consideriamo tre funtori \(F,G,H:\ctC\to\ctD\) e due trasformazioni naturali \(\alpha:F\Rightarrow G\) e \(\beta:G\Rightarrow H\).
	La \emph{composizione verticale} \(\beta\cmp\alpha:F\Rightarrow H\) delle transformazioni naturali \(\beta,\alpha\), rappresentate avendo il bordo \(G\) in comune,
	\[\xymatrix{
		\ctC \ar[rr]|G\ar@/^1.5pc/[rr]^F\ar@/_1.5pc/[rr]_H
		\rruppertwocell<\omit>{<-2>\alpha}
		\rrlowertwocell<\omit>{<2>\beta}
		&& \ctD
		}
	\]
	è definita dalla composizione \(\beta_X\cmp\alpha_X : FX\to GX\to HX\).
\end{definition}
L'operazione di composizione verticale definisce una categoria \(\ctCat(\ctC,\ctD)\) la cui classe degli oggetti è fatta dai funtori \(\ctC\fun\ctD\), e dove i morfismi \(F\nat G\) sono le trasformazioni naturali come in \ref{def_nat}; evidentemente, la trasformazione naturale identica \(\id_F : F\nat F\) ha per componenti \(\id_{FX} : FX\to FX\).
\begin{example}\index{Categoria!---e di funtori}
	Molti degli esempi di categorie fatti lungo la sezione \ref{sec_esempi_cats} si descrivono come categorie di funtori:
	\begin{itemize}
		\item La categoria delle spanne in \(\ctC\) di \ref{def_span_e_cospan} è la categoria dei funtori \(\ctCat(\Lambda^2_0,\ctC)\) di dominio la spanna generica di \ref{ex_spancospan} e codominio \(\ctC\); dualmente, la categoria delle cospanne in \(\ctC\) è la categoria dei funtori \(\ctCat(\Lambda^2_2,\ctC)\).
		\item La categoria dei multidigrafi di \ref{ex_cat_grafi} è la categoria dei funtori \(\ctCat((0\toto 1)^\op,\ctSet)\); ciò rende precisa l'osservazione subito successiva alla definizione di \ref{ex_cat_doppiafreccia}. Lasciamo per esercizio a chi legge di mostrare che una trasformazione naturale tra funtori \(F,G\) che assumono rispettivamente per valori delle coppie di funzioni
		      \[\xymatrix{
				      F_1\ar@<.25em>[r]^{Fs}\ar@<-.25em>[r]_{Ft} & F_0 & G_1\ar@<.25em>[r]^{Gs}\ar@<-.25em>[r]_{Gt} & G_0
			      }\]
		      riduce a un omomorfismo di multidigrafi come in \ref{ex_cat_grafi}.
		\item La categoria delle rappresentazioni (in \(\ctSet\)) di un monoide di \ref{ex_cat_g_insiemi} è la categoria dei funtori \(\susp M \fun\ctSet\); una trasformazione naturale tra rappresentazioni viste come funtori è precisamente un omomorfismo equivariante. Un fatto analogo è vero per le rappresentazioni di un monoide in \(\ctD\), come in \ref{es_fun_repre} (scrivere i dettagli per esercizio).
	\end{itemize}
\end{example}
Alcuni esempi di categorie di funtori sono ricchi di applicazioni alla topologia, data la presenza di funtori che le legano alle categorie di oggetti geometrici; raccogliamo alcuni esempi tra i più noti.
\begin{hExample}[Trasformazione di Hurewicz in topologia]{tech}\index{Hurewicz!mappa di ---}
	\index{Witold Hurewicz}
	Questo esempio fa uso di alcuni concetti elementari della teoria dell'omologia singolare, e delle definizioni \ref{fun_ex_omoto_omolo} dei funtori di omotopia e omologia:
	\begin{itemize}
		\item se \(X\) è uno spazio topologico, abbiamo definito in \ref{} il gruppo \(\pi_n(X)\) che ha per elementi le classi di omotopia di funzioni continue \(S^n \to X\);
		\item quando \(X=S^n\), il gruppo \(H_n(S^n)\) è ciclico infinito; fissiamo un suo generatore \(\upsilon\) (cosicché per ogni altra classe di omologia \(\kappa\in H_n(S^n)\) esiste un unico intero \(k\in\bbZ\) tale che \(\kappa = k\cdot \upsilon\)).
	\end{itemize}
	Nel 1935, il topologo polacco Witold Hurewicz ha definito un esempio di trasformazione naturale tra il funtore `\(n\)-esimo gruppo di omotopia' ed `\(n\)-esimo gruppo di omologia singolare', \(h : \pi_n \nat H_n\),\footnote{Ha definito, pochi anni prima, anche i funtori \(\pi_n\) di \ref{fun_ex_omoto_omolo}. Hurewicz è un pensatore di fondamentale importanza per la teoria delle categorie, perché è il primo a introdurre la notazione che rappresenta gli omomorfismi come frecce \(f : X \to Y\), come viene ricordato anche nelle Note Storiche del capitolo 1 di \cite{working-categories}.} assegnando a un elemento di \(\pi_n(X) = [S^n,X]\), cioè alla classe di omotopia di una certa \(f : S^n \to X\), l'elemento \(f_*(\upsilon)\) immagine del generatore sopra fissato mediante \(f_* : H_n(S^n) \to H_n(X)\).

	Questa trasformazione è naturale: data una funzione continua \(g : X\to Y\) va mostrato che per ogni \(n\ge 0\) il quadrato
	% https://q.uiver.app/#q=WzAsNCxbMCwwLCJcXHBpX24oWCkiXSxbMCwxLCJcXHBpX24oWSkiXSxbMSwwLCJIX24oWCxcXGJiWikiXSxbMSwxLCJIX24oWSxcXGJiWikiXSxbMCwxXSxbMCwyXSxbMiwzXSxbMSwzXV0=
	\[\begin{tikzcd}[ampersand replacement=\&,cramped]
			{\pi_n(X)} \& {H_n(X,\bbZ)} \\
			{\pi_n(Y)} \& {H_n(Y,\bbZ)}
			\arrow[from=1-1, to=1-2]
			\arrow[from=1-1, to=2-1, "g_*"]
			\arrow[from=1-2, to=2-2, "H_n(g)"]
			\arrow[from=2-1, to=2-2]
		\end{tikzcd}\]
	è commutativo (nella categoria dei gruppi, dato che per \(n=1\) non c'è motivo \(\pi_1(X),\pi_1(Y)\) siano abeliani). Questo segue dalla definizione delle mappe coinvolte: come detto, in orizzontale, una classe di omotopia \([f : S^n \to X]\) viene mandata in \(H_n(f)(\upsilon)\); in verticale, \([f]\) viene mandata in \([g\cmp f : S^n \to X\to Y]\), cosicché la funtorialità di \(H_n\) (e la sua invarianza per omotopia), permette di concludere.
	\fosco{in attesa di una collocazione migliore}
\end{hExample}
\begin{example}\label{ex_cat_sset}\index{Insieme simpliciale}
	La categoria \(\ctsSet=\presh{\bsDelta}\) degli insiemi simpliciali ha per oggetti i funtori \(\bsDelta^\op\fun\ctSet\) definiti in \ref{def_insieme_simpliciale}, dove \(\bsDelta\) è la categoria dei simplessi di \ref{rmk_delta_e_deltaPlus}, e per morfismi le trasformazioni naturali tra questi funtori.

	Una trasformazione naturale \(f : X_*\nat Y_*\) è detta una \emph{mappa simpliciale}, e consiste di una famiglia di funzioni \(f_n : X_n \to Y_n\) tra gli insiemi di \(n\)-simplessi, che sono compatibili con le mappe di faccia e di degenerazione: questo significa che ogni diagramma
	% https://q.uiver.app/#q=WzAsOCxbMCwwLCJYX24iXSxbMCwxLCJYX3tuLTF9Il0sWzEsMCwiWV9uIl0sWzEsMSwiWV97bi0xfSJdLFsyLDAsIlhfbiJdLFszLDAsIllfbiJdLFsyLDEsIlhfe24rMX0iXSxbMywxLCJZX3tuKzF9Il0sWzAsMSwiZF9pIiwyXSxbMiwzLCJkX2knIl0sWzAsMiwiZl9uIl0sWzEsMywiZl97bi0xfSIsMl0sWzQsNiwic19pIiwyXSxbNiw3LCJmX3tuKzF9IiwyXSxbNSw3LCJzX2knIl0sWzQsNSwiZl9uIl1d
	\[\begin{tikzcd}
			{X_n} & {Y_n} & {X_n} & {Y_n} \\
			{X_{n-1}} & {Y_{n-1}} & {X_{n+1}} & {Y_{n+1}}
			\arrow["{f_n}", from=1-1, to=1-2]
			\arrow["{d_i}"', from=1-1, to=2-1]
			\arrow["{d_i'}", from=1-2, to=2-2]
			\arrow["{f_n}", from=1-3, to=1-4]
			\arrow["{s_i}"', from=1-3, to=2-3]
			\arrow["{s_i'}", from=1-4, to=2-4]
			\arrow["{f_{n-1}}"', from=2-1, to=2-2]
			\arrow["{f_{n+1}}"', from=2-3, to=2-4]
		\end{tikzcd}\]
	è commutativo, scegliendo come frecce verticali le facce \(d_i,d_i'\) o le degenerazioni \(s_i,s_i'\) di \(X_*\) e di \(Y_*\) rispettivamente.

	La categoria degli insiemi simpliciali è di grande rilevanza in topologia algebrica (si veda \cite{goerss-jardine,May1999,strom2011modern,DK1}) dato che \(\ctsSet\) ammette un funtore di \emph{realizzazione}
	\[\dmFun{r}{\ctsSet}{\ctTop}\]
	verso la categoria \(\ctTop\) degli spazi topologici, che porta un oggetto la cui struttura è puramente combinatoria\fshyp{}algebrica (il funtore \(X : \bsDelta^\op\fun\ctSet\)) in uno spazio topologico che può essere costruito iterativamente attaccando dischi di dimensione via via più alta (quello che in \cite{Whitehead1978,May1999} e molti altri è chiamato un CW-complesso; si veda \cite{fritsch1990cellular} per una introduzione elementare o \cite{goerss-jardine,riehl2014categorical} per una più indirizzata alle proprietà categoriali di \(\ctsSet\) e del funtore \(r\) di realizzazione topologica).
\end{example}

\begin{remark}[Perché si chiama `globo'?]\label{globi_come_sonfatti}\index{Globo}
	Ragionamenti simili si possono fare anche per la categoria degli insiemi \(n\)-globulari, cioè per la categoria \(\presh{\ctGl(n)}\) di funtori e trasformazioni naturali, contemporaneamente anche spiegando l'origine del termine `globo' per il suo oggetto generico.

	Si osservi che \(\ctGl(1) = \genAArrow\) è la doppia freccia generica di \ref{ex_cat_doppiafreccia}. Da ciò deduciamo che `rappresentare' \(\ctGl(1)\) nella categoria \(\ctSet\) di insiemi e funzioni consiste precisamente nel dare un multidigrafo, con funzioni di dominio e codominio. I globi generici sono una generalizzazione dei multidigrafi, che permettono la rappresentazione di `celle' di dimensione maggiore di 1. Sarà la \autoref{def_insieme_globulare} a rendere precisa questa idea (a sua volta, mediante la nozione di funtore controvariante in \ref{def_funtore_contro}), ma per il momento, è già possibile osservare che una terna di insiemi
	\[\xymatrix{
		C \ar@<.25em>[r]^-{s_2}\ar@<-.25em>[r]_-{t_2}& E \ar@<.25em>[r]^-{s_1}\ar@<-.25em>[r]_-{t_1}& V
		}\]
	di celle \(C\), lati \(E\) e vertici \(V\), con funzioni \(s_1,t_1,s_2,t_2\) consta di elementi \(c\in C\), ciascuno dei quali ha un dominio \(s_2(c) = F\) e un codominio \(t_2(c)=G\), ciascuno dei quali ha un dominio e un codominio,
	\begin{gather*}
		s_1s_2(c) = s_1 F = X\\
		t_1s_2(c) = t_1 F = Y\\
		s_1t_2(c) = s_1 G = Z\\
		t_1t_2(c) = t_1 G = W
	\end{gather*}
	Del resto valgono le relazioni in \ref{ex_globo}, di modo che \(X=Z,	Y=W\), cosicché la cella globulare \(c\in C\) si disegna come
	\[\xymatrix{
		X \rtwocell^F_G{c} & Y
		}\]
\end{remark}
La categoria degli insiemi simpliciali è però decisamente più studiata di quella dei globi.
\begin{remark}[Nervo di una categoria piccola]\label{def_nervo_di_una_cat}\index{Nervo!--- di una categoria}
	Il maggiore interesse per \(\ctsSet\) risiede principalmente nel fatto che ogni categoria piccola \(\ctC\in\ctCat\) induce un insieme simpliciale \(N\ctC\) detto il \emph{nervo} di \(\ctC\), definito nel modo che segue:
	\begin{itemize}
		\item Gli 0-simplessi di \(N\ctC\) sono gli oggetti di \(\ctC\); gli 1-simplessi	sono le frecce di \(\ctC\), e per \(n\ge 2\) gli \(n\)-simplessi di \(N\ctC\) consistono delle \(n\)-uple di frecce componibili di \(\ctC\), cioè
		      \[(N\ctC)_n := \{C_0 \xto{f_1} C_1 \xto{f_2}C_2\xto{f_3}\cdots \xto{f_{n-1}} C_{n-1}\xto{f_n} C_n\mid \tup[0]Cn,\in\ctC_0\}.\]
		      Perciò possiamo denotare il generico \(n\)-simplesso di \(N\ctC\) come una \(n\)-upla ordinata \([\tup fn,]\).
		\item Le mappe di faccia \(d_i : (N\ctC)_n \to (N\ctC)_{n-1}\) sono definite, a seconda che \(i=0\) o \(i=n\), come l'omissione del primo o dell'ultimo termine della sequenza, e altrimenti come la composizione di due frecce adiacenti:
		      \[\begin{cases}
				      d_0[\tup fn,] := [\tup[2]fn,]                                           \\
				      d_n[\tup fn,] := [\tup f{n-1},]                                         \\
				      d_i[\tup fn,] := [\tup f{i-1}, ,f_{i+1}\cmp f_i, \tup[i+2]fn,] & 0<i<n.
			      \end{cases}\]
		\item Le mappe di degenerazione \(s_i : (N\ctC)_n \to (N\ctC)_{n+1}\)	sono definite come l'inserimento di una copia dell'\(i\)-esimo termine (cioè della sua freccia identica) nella sequenza:
		      \[\begin{cases}
				      s_0[\tup fn,] := [\Id[C_0],\tup fn,]                         \\
				      s_n[\tup fn,] := [\tup fn,,\Id[C_n]]                         \\
				      s_i[\tup fn,] := [\tup fi, ,\Id[C_i], \tup[i+1]fn,] & 0<i<n.
			      \end{cases}\]
	\end{itemize}
\end{remark}
\begin{proposition}\index{Nervo!--- di una categoria}
	Il nervo di \(\ctC\) è un insieme simpliciale, ovvero valgono le relazioni \eqref{idsim_1}--\eqref{idsim_3} di \ref{def_insieme_simpliciale}. In più, il funtore nervo è pienamente fedele nel senso di \ref{fun_pienfed}, cosicché la categoria \(\ctCat\) si può identificare ad una sottocategoria piena di \(\ctsSet\).
\end{proposition}
\begin{proof}
	Vanno verificate le identità simpliciali: partiamo dalla mappe di faccia;
	\begin{itemize}
		\item per \(i=0<j<n\) si ha
		      \begin{align*}
			      d_0 d_j[\tup fn,]    & = d_0[\tup f{j-1}, ,f_{j+1}\cmp f_j, \tup[j+2]fn,] \\
			                           & = [\tup[2]f{j-1}, ,f_{j+1}\cmp f_j, \tup[j+2]fn,]  \\
			      d_{j-1}d_0[\tup fn,] & = d_{j-1}[\tup[2]fn,]                              \\
			                           & =[\tup[2]f{j-1}, ,f_{j+1}\cmp f_j, \tup[j+2]fn,];
		      \end{align*}
		\item per \(0<i<j=n\) si ha
		      \begin{align*}
			      d_id_n[\tup fn,]     & = d_i[\tup f{n-1},]                                        \\
			                           & =[\tup f{i-1}, ,f_{i+1}\cmp f_i, \tup[i+2]f{n-1},]         \\
			      d_{n-1}d_i[\tup fn,] & = d_{n-1}[\tup f{i-1}, ,f_{i+1}\cmp f_i, \tup[i+2]f{n-1},] \\
			                           & =[\tup f{i-1}, ,f_{i+1}\cmp f_i, \tup[i+2]f{n-1},];
		      \end{align*}
		\item per \(0<i<n\), e negli altri casi, si ragiona in maniera del tutto simile; si noti che l'equazione \eqref{idsim_1} quando \(j=i+1\) è vera per l'assioma di associatività della categoria \(\ctC\); per esempio, per \(n=5\) si ha
		      \begin{align*}
			      d_1d_2[\tup f5,] & = d_1[f_1,f_3\cmp f_2,f_4,f_5]   \\
			                       & =[(f_3\cmp f_2)\cmp f_1,f_4,f_5] \\
			      d_1d_1[\tup f5,] & = d_1[f_2\cmp f_1,f_3,f_4,f_5]   \\
			                       & =[f_3\cmp (f_2\cmp f_1),f_4,f_5]
		      \end{align*}
		      Questo dimostra tutte le equazioni di \eqref{idsim_1}.
	\end{itemize}
	Per le degenerazioni si hanno simili risultati:
	\begin{itemize}
		\item per \(i=0\) si ha
		      \begin{align*}
			      s_0 s_j[\tup fn,]    & = s_0[\tup fj, ,\Id[C_j], \tup[j+1]fn,]       \\
			                           & =[\Id[C_0],\tup fj, ,\Id[C_j], \tup[j+1]fn,]  \\
			      s_{j+1}s_0[\tup fn,] & = s_{j+1}[\Id[C_0],\tup fn,]                  \\
			                           & =[\Id[C_0],\tup fj, ,\Id[C_j], \tup[j+1]fn,];
		      \end{align*}
		\item le identità \eqref{idsim_2} per \(j=n\) e \(0<i<n\) si dimostrano in maniera del tutto simile.
	\end{itemize}
	L'interazione tra facce e degenerazioni usa gli assiomi di identità per la categoria \(\ctC\); in particolare,
	\begin{align*}
		d_js_j[\tup fn,]     & = d_j[\tup fj, ,\Id[C_j], \tup[j+1]fn,]         \\
		                     & =[\tup f{j-1}, ,\Id[C_j]\cmp f_j, \tup[j+1]fn,] \\
		                     & =[\tup f{j-1}, , f_j, \tup[j+1]fn,]             \\
		d_{j+1}s_j[\tup fn,] & = d_{j+1}[\tup fj, ,\Id[C_j], \tup[j+1]fn,]     \\
		                     & =[\tup fj, ,f_{j+1}\cmp\Id[C_j], \tup[j+2]fn,]  \\
		                     & =[\tup fj, ,f_{j+1} , \tup[j+2]fn,]
	\end{align*}
	Se \(i > j+1\) si apprezza meglio l'equazione che va dimostrata in un suo caso particolare: poniamo quindi, ad esempio, \(n=5\), e \((i,j)=(4,2)\); si ha quindi
	\begin{align*}
		d_4s_2[\tup f5,] & = [f_1,f_2,\Id[C_2],f_3,f_4,f_5]     \\
		                 & =[f_1,f_2,\Id[C_2],f_4\cmp f_3,f_5]  \\
		s_2d_3[\tup f5,] & = s_2[f_1,f_2,f_4\cmp f_3,f_5]       \\
		                 & =[f_1,f_2,\Id[C_2],f_4\cmp f_3,f_5].
	\end{align*}
	Lasciamo a chi legge l'onere di scrivere il caso generale (che non cambia in nulla la struttura di questo argomento), così come l'identità \(d_i\cmp s_j=s_{j-1}\cmp d_i\) quando \(i<j\), che si dimostra con una analoga quantità di tedio; questo conclude la prima parte della dimostrazione.

	\medskip
	Va ora dimostrato che la corrispondenza \(N : \ctCat \fun \ctsSet\) è un funtore, ovvero che un funtore tra categorie piccole \(F : \ctC\fun\ctD\) induce una trasformazione naturale \(NF: N\ctC\nat N\ctD\) tra i rispettivi nervi, e in più che \emph{ogni} mappa simpliciale \(f : N\ctC\nat N\ctD\) origina da un (unico) funtore.

	Del resto, una tale \(f\) è definita da delle componenti \((f_0,f_1,f_2,\dots)\), e dalla definizione sappiamo che \((N\ctC)_0,(N\ctC)_1,(N\ctC)_2\) sono rispettivamente gli oggetti, le frecce e le coppie di frecce componibili di \(\ctC\); altrettanto è vero per \(\ctD\), e allora \(f_0 : \ctC_0\to\ctD_0\) è la candidata funzione sugli oggetti di un funtore, \(f_1 : \ctC_1\to\ctD_1\) la funzione sulle frecce. La compatibilità di \(f_0,f_1\) con l'unica degenerazione \(s : (N\ctC)_0\to(N\ctC)_1\), definita da \(s[C] = [\Id[C]]\) afferma che \(f_1(\Id[C]) = \Id[f_0C]\). Similmente, la compatibilità di \(f_2\) con la faccia \(d_1\), che consiste della mappa di composizione \(N\ctC_2\to N\ctC_1\), afferma che \(f_1\) preserva anche le composizioni.
\end{proof}
\begin{definition}\label{def_natutra}\index{Trasformazione naturale}\index{homCDFG@\(\Hom{\ctCat(\ctC,\ctD)}(F,G)\)}
	Come visto in \ref{ex_cat_cat} i funtori sono i morfismi di una categoria \(\ctCat\) delle categorie (piccole); ma abbiamo appena trovato che anche le trasformazioni naturali tra funtori rendono la classe \(\ctCat(\ctC,\ctD)\) dei funtori \(F : \ctC\funto\ctD\) a sua volta una categoria; in più la legge di composizione orizzontale \(\blank\horcomp\blank :\ctCat(\ctD,\ctE)\times\ctCat(\ctC,\ctD) \to \ctCat(\ctC,\ctE)\) di \ref{} è un bifuntore (la regola di interscambio di \ref{} equivale ad affermare questo).

	Allora le trasformazioni naturali \(\alpha : F\nat G\) si pensano come ``frecce tra frecce'', o ``frecce di dimensione 2'', nel senso che andiamo a spiegare. La definizione \ref{def_nat} ha introdotto una notazione per le trasformazioni naturali: è comune rappresentarle con delle frecce del tipo \(\To\), a riempire un diagramma di funtori come segue:
	\[
		\xymatrix{\ctC \rrtwocell^F_G{\alpha}&& \ctD.}
	\]
	In questa rappresentazione, \(\alpha\) è una freccia di dominio \(F\) e codominio \(G\), ed \(F\) e \(G\) hanno a loro volta un dominio \(\ctC\) e un codominio \(\ctD\).

	\medskip
	La struttura che ha per oggetti le categorie piccole, per frecce i funtori tra esse, e per frecce tra frecce le trasformazioni naturali è un esempio di \emph{2-categoria} (di cui non diamo la definizione qui, si veda \cite{Kelly2005b}).
\end{definition}
\begin{notation}\index{Trasformazione naturale}
	Coerentemente con la notazione usata finora, dovremmo denotare \(\Hom{\ctCat(\ctC,\ctD)}(F,G)\) la collezione delle trasformazioni naturali tra due funtori, cioè dei morfismi nella categoria \(\ctCat(\ctC,\ctD)\). \`E però conveniente usare una notazione più compatta, specie quando \(\ctC,\ctD\) sono determinate dal contesto: una scelta comune per riferirsi a \(\Hom{\ctCat(\ctC,\ctD)}(F,G)\) è \(\Nat(F,G)\) (ma alcuni autori usano convenzioni ancora più compatte). \`E conveniente denotare esplicitamente dominio e codominio di \(F,G\), perciò noi introduciamo la notazione
	\[\natHom FG :=\Hom{\ctCat(\ctC,\ctD)}(F,G)\]
\end{notation}
\begin{remark}\index{Categoria!--- di funtori}
	Sarà spesso utile considerare esempi in cui \(\ctD\) non è una categoria piccola (l'esempio più illustre e pervasivo è \ref{def_yoneda_embeddu}), e sarà altrettanto utile considerare esempi di funtori tra categorie entrambe larghe; però la definizione della `categoria' \(\ctCat(\ctC,\ctD)\) pone un problema di teoria degli insiemi quando \(\ctC\) non è piccola; infatti, in quel caso la classe delle trasformazioni naturali tra due funtori \(F,G:\ctC\funto\ctD\) è una sottoclasse di \(\prod_{C\in\ctC_0}\ctD(FC,GC)\), che in generale può non essere un insieme (per esempio, può coincidere con \(\prod_{C\in\ctC_0}\ctD(FC,GC)\), quando?).
\end{remark}
Passiamo ora alla definizione di isomorfismo nella categoria formata da funtori e trasformazioni naturali.
\begin{hDefinition}[Isomorfismo naturale]{fund}\label{def_isonat}\index{Isomorfismo!--- naturale}\index{Trasformazione naturale!isomorfismo naturale}
	Dati due funtori \(F,G : \ctC\fun\ctD\) tra categorie, un \emph{isomorfismo naturale} consiste di una trasformazione naturale \(\alpha : F\nat G\) che possieda una inversa per l'operazione di composizione verticale di \ref{def_vcomp}; in altre parole, \(\alpha\) è un isomorfismo naturale se esiste una trasformazione naturale \(\bar\alpha : G\nat F\) tale che \(\alpha\vcmp\bar\alpha = 1_G\) e \(\bar\alpha\vcmp\alpha =1_F\). Evidentemente tale \(\bar\alpha\) è unica e si può denotare con \(\alpha^{-1}\).
\end{hDefinition}
Inoltre, la condizione di naturalità per \(\alpha^{-1}\) può essere omessa:
\begin{lemma}\index{Isomorfismo!--- naturale}
	Una trasformazione naturale \(\alpha : F\nat G\) è invertibile se e solo se ciascuna delle sue componenti è un isomorfismo di \(\ctD\); infatti, la condizione di naturalità per \(\alpha\) è il fatto che per ogni \(u : X\to Y\) si abbia
	\[\alpha_Y \cmp Fu = Gu\cmp \alpha_X\]
	da cui segue la condizione di naturalità per \(\alpha^{-1}\) componendo da entrambi i lati questa equazione:
	\[Fu\cmp \alpha_X^{-1}=\alpha_Y^{-1}\cmp \alpha_Y \cmp Fu\cmp \alpha_X^{-1} = \alpha_Y^{-1}\cmp Gu\cmp \alpha_X\cmp \alpha_X^{-1}=\alpha_Y^{-1}\cmp Gu.\]
\end{lemma}
\begin{hLemma}[Presentazione equivalente di trasformazioni naturali]{fund}\index{Categoria!--- freccia generica}\index{Freccia generica}
	Si considerino due categorie \(\ctC,\ctD\). \`E equivalente dare:
	\begin{enumtag}{tn}
		\item \label{tn_1} una trasformazione naturale \(\alpha : F\nat G\), per due funtori \(F,G : \ctC\toto\ctD\) paralleli;
		\item \label{tn_2} un funtore \(A : \ctC\times \genArrow \fun \ctD\), dove \(\ctC\times \genArrow\) è il prodotto di categorie (si veda \ref{def_cat_prodotto}) tra \(\ctC\) e la freccia generica (si veda \ref{ex_cat_freccia}).
	\end{enumtag}
	Più in particolare, è equivalente dare:
	\begin{enumtag}{in}
		\item \label{in_1} un isomorfismo naturale \(\alpha : F\nat G\), per due funtori \(F,G : \ctC\toto\ctD\) paralleli;
		\item \label{in_2} un funtore \(J : \ctC\times \ctIso \fun \ctD\), dove \(\ctC\times \ctIso\) è il prodotto di categorie (si veda \ref{def_cat_prodotto}) tra \(\ctC\) e l'isomorfismo generico (si veda \ref{ex_cat_iso}).
	\end{enumtag}
	\begin{proof}
		Dato un funtore \(A\) come in \ref{tn_2}, la restrizione \(A|_{\{0\}}\) di \(A\) alla sottocategoria generata da \(\{0\}\subset\genArrow\) è un funtore \(\{0\}\times\ctC\cong \ctC \fun\ctD\); chiamiamolo \(F\); altrettanto è vero per il funtore \(G = A|_{\{1\}} : \{1\}\times\ctC\cong \ctC \fun\ctD\). Sia poi \(u : 0\to 1\) l'unica freccia non identica; allora \(A(\id_C,u) : A(C,0) \to A(C,1)\) è una freccia di \(\ctD\) della forma \(FC \to GC\) per la definizione appena data. Del resto, in \(\ctC\times \genArrow\) esiste un ovvio quadrato commutativo,
		% https://q.uiver.app/#q=WzAsNCxbMCwwLCIoQywwKSJdLFswLDEsIihDLDEpIl0sWzEsMCwiKEQsMCkiXSxbMSwxLCIoRCwxKSJdLFswLDIsIihmLFxcaWRfMCkiXSxbMiwzLCIoXFxpZF9ELHUpIl0sWzAsMSwiKFxcaWRfQyx1KSIsMl0sWzEsMywiKGYsXFxpZF8xKSIsMl1d
		\[\begin{tikzcd}[ampersand replacement=\&,cramped]
				{(C,0)} \& {(C',0)} \\
				{(C,1)} \& {(C',1)}
				\arrow["{(f,\id_0)}", from=1-1, to=1-2]
				\arrow["{(\id_C,u)}"', from=1-1, to=2-1]
				\arrow["{(\id_{C'},u)}", from=1-2, to=2-2]
				\arrow["{(f,\id_1)}"', from=2-1, to=2-2]
			\end{tikzcd}\]
		per ogni \(f : C \to C'\) (la cui diagonale è \((f,u) : (C,0) \to (C',1)\)); la funtorialità di \(A\) manda questo quadrato in un quadrato commutativo della forma
		% https://q.uiver.app/#q=WzAsNCxbMCwwLCJGQyJdLFswLDEsIkdDIl0sWzEsMCwiRkMnIl0sWzEsMSwiR0MnIl0sWzAsMiwiRmYiXSxbMiwzLCIoXFxpZF9ELHUpIl0sWzAsMSwiKFxcaWRfQyx1KSIsMl0sWzEsMywiR2YiLDJdXQ==
		\[\begin{tikzcd}[ampersand replacement=\&,cramped]
				FC \& {FC'} \\
				GC \& {GC'}
				\arrow["Ff", from=1-1, to=1-2]
				\arrow["{(\id_C,u)}"', from=1-1, to=2-1]
				\arrow["{(\id_D,u)}", from=1-2, to=2-2]
				\arrow["Gf"', from=2-1, to=2-2]
			\end{tikzcd}\]
		Cosicché le componenti \(\alpha_{A,C} := A(\id_C,u)\) definiscono una trasformazione naturale \(\alpha_A : F\nat G\).

		Vice versa, una trasformazione naturale \(\alpha : F \nat G\) definisce un funtore \(A_\alpha\) ponendo
		\begin{itemize}
			\item \(A_\alpha(X,0) := FX\) e \(A_\alpha(Y,1)=GY\) per ogni \(X,Y\in\ctC_0\);
			\item \(A_\alpha(f,\id_0) := Ff\) e \(A_\alpha(g,\id_1)=Gg\) per ogni \(f,g\in\ctC_1\);
			\item \(A_\alpha(C,u) := \alpha_C : FC \to GC\) per la definizione appena data di \(A_\alpha\) sugli oggetti.
		\end{itemize}
		La naturalità di \(\alpha\) implica che \(A\) sia un bifuntore nel senso di \ref{def_bifuntore}. Infine, è facile verificare che \(\alpha_{A_\alpha}\) coincide, componente per componente, con \(\alpha\), e che \(A_{\alpha_A}\) coincide, sugli oggetti e sulle frecce di \(\ctC\times\genArrow\), con \(A\).

		La dimostrazione dell'equivalenza tra \ref{in_1} e \ref{in_2}, quando a \(\genArrow\) si sostituisca l'isomorfismo generico è del tutto analoga; semplicemente, ora le componenti di \(\alpha_A\) dovranno essere invertibili.
	\end{proof}
\end{hLemma}
\begin{example}[Naturali tra mappe monotòne]\label{nat_tra_posets}\index{Trasformazione naturale!--- tra ordini}
	Abbiamo visto in \ref{exa_monotone_funtori} che una funzione monotòna \(f:(X,\le)\to(Y,\le)\) si può vedere come un funtore. In questo contesto, date due funzioni monotone \(f,g:(X,\le)\to(Y,\le)\), si ha una (unica) trasformazione naturale \(f\Rightarrow g\), o \(f\le g\), se e solo se per ogni \(x\in X\), \(f(x)\le g(x)\). (Quest'ultima disuguaglianza rappresenta la componente della trasformazione nell'oggetto \(x\).)

	Una trasformazione naturale tra funtori che sono insiemi ordinati è perciò una \emph{proposizione}: quella che asserisce che \(f\) e \(g\) sono minori una dell'altra, punto per punto.
\end{example}

% \begin{example}\index{Equivarianza}\index{Trasformazione naturale!--- tra monoidi}
% 	Abbiamo visto nell'Esempio~\ref{exa_azioni_funtori} che dato un gruppo \(G\), un funtore \(\susp G\to\ctSet\) è un insieme \(X\) con un'azione di \(G\) su \(X\). Dati due funtori \(\susp G\to\ctSet\), i.e.\ due insiemi \(X\) e \(Y\) con un'azione di \(G\), una trasformazione naturale tra questi due funtori è una funzione \(f:X\to Y\) che commuta con l'azione di \(G\), nel senso che per ogni \(g\in G\), il seguente diagramma commuta,
% 	\[
% 		\begin{tikzcd}
% 			X \ar[d, "f"'] \ar{r}{g\cdot\blank} & X \ar{d}{f} \\
% 			Y \ar[r, "g\cdot\blank"'] & Y
% 		\end{tikzcd}
% 	\]
% 	(abbiamo indicato con \(g\cdot\blank\) la biiezione di \(X\) che corrisponde all'elemento \(g\) nella rappresentazione).

% 	In teoria delle rappresentazioni, la condizione su \(f\)
% 	\[\text{per ogni } g\in G,\, f(g\cdot x)=g\cdot f(x)\]
% 	si chiama \emph{equivarianza}, ed è esattamente la condizione di naturalità istanziata in questo caso. Una conseguenza immediata della condizione è che \(f\) si restringe all'insieme \(X^G:=\{x\in X\mid \forall g\in G . g\cdot x = x\}\) dei punti fissi dell'azione di \(G\) su \(X\); infatti, per ogni \(x\in X^G\) si ha \(g\cdot f(x) = f(g\cdot x) = f(x)\).
% \end{example}
\begin{example}\label{doppio_duale}\index{Trasformazione naturale!--- sul biduale}
	Dato uno spazio vettoriale \(V\) di dimensione finita su un campo \(k\), ogni primo corso di algebra lineare mostra che è possibile definire un isomorfismo
	% https://q.uiver.app/#q=WzAsMixbMCwwLCJcXGVwc2lsb24gOiBWIl0sWzEsMCwiXFxkdWFse1xcZHVhbCBWfSJdLFswLDFdXQ==
	\[\begin{tikzcd}
			{\epsilon : V} & {\ddual V}
			\arrow[from=1-1, to=1-2]
		\end{tikzcd}\]
	dove per ogni \(W\), \(\dual W :=\Hom{\ctVect}(W,k)\) è lo spazio vettoriale delle mappe lineari \(W\to k\); \(\epsilon_V(v)\) è la mappa lineare di \emph{valutazione} \(\dual V\to k\) che manda \(\alpha : V\to k\) in \(\alpha(v)\). Questo isomorfismo è naturale, dato che una mappa lineare \(f : V\to W\) fa commutare il quadrato
	% https://q.uiver.app/#q=WzAsNCxbMCwwLCJWIl0sWzAsMSwiVyJdLFsxLDAsIlxcZGR1YWwgViJdLFsxLDEsIlxcZGR1YWwgVyJdLFsyLDMsIlxcZGR1YWwgZiJdLFsxLDMsIlxcZXBzaWxvbl9XIiwyXSxbMCwxLCJmIiwyXSxbMCwyLCJcXGVwc2lsb25fViJdXQ==
	\[\begin{tikzcd}
			V & {\ddual V} \\
			W & {\ddual W}
			\arrow["{\epsilon_V}", from=1-1, to=1-2]
			\arrow["f"', from=1-1, to=2-1]
			\arrow["{\ddual f}", from=1-2, to=2-2]
			\arrow["{\epsilon_W}"', from=2-1, to=2-2]
		\end{tikzcd}\]
	(la freccia verticale destra è definita mandando \(\Phi : \dual V \to k\) nella mappa \(\Phi \cmp \dual f\), che manda \(\alpha : V\to k\) in \(\Phi(\alpha\cmp f)\in k\)).
\end{example}
\begin{warning}\label{il_duale_singolo_no}\index{Spazio duale}
	\`E altrettanto noto che si può definire un isomorfismo \(\theta_V : V\cong \dual V\), ma per farlo è essenziale \emph{scegliere} una base \(\mathcal{B}\) di \(V\) (ossia, l'isomorfismo \(\theta_V^{\mathcal{B}}\) non può essere costruito senza questa scelta, e ne dipende, laddove invece l'isomorfismo di \ref{doppio_duale} è  \emph{polimorfo} in \(V\), cioè è definito sempre alla stessa maniera, uniformemente nel `parametro' \(V\)).

	Non è possibile scegliere una base \(\mathcal{B}=\mathcal{B}_V\) per ogni spazio vettoriale \(V\), con degli isomorfismi \(\theta^{\mathcal{B}}_V\), che si assemblano in un isomorfismo naturale `sghembo'
	% https://q.uiver.app/#q=WzAsNCxbMCwwLCJWIl0sWzAsMSwiVyJdLFsxLDEsIlxcZHVhbCBXIl0sWzEsMCwiXFxkdWFsIFYiXSxbMCwzLCJcXHRoZXRhX1YiXSxbMCwxLCJmIiwyXSxbMSwyLCJcXHRoZXRhX1ciLDJdLFsyLDMsIlxcZHVhbCBmIiwyXV0=
	\[\begin{tikzcd}
			V & {\dual V} \\
			W & {\dual W}
			\arrow["{\theta_V}", from=1-1, to=1-2]
			\arrow["f"', from=1-1, to=2-1]
			\arrow["{\theta_W}"', from=2-1, to=2-2]
			\arrow["{\dual f}"', from=2-2, to=1-2]
		\end{tikzcd}\]
	Il problema è più profondo del fatto che, strettamente parlando, le trasformazioni naturali sono state definite tra funtori della stessa varianza;\footnote{Questa restrizione si può aggirare definendo una trasformazione naturale sghemba tra un funtore covariante \(F : \ctC\fun\ctD\) e un funtore controvariante \(G : \ctC^\op\fun\ctD\) come una famiglia \(\alpha_C : FC\to GC\) di morfismi di \(\ctD\) con la proprietà che per ogni \(f\in \Hom{\ctC}(X,Y)\) si abbia \(\alpha_X = Gf\cmp\alpha_Y\cmp Ff\). Questi esempi sono relativamente comuni, ma abbiamo mostrato che in questo contesto nemmeno una trasformazione naturale sghemba può esistere.} infatti, se una famiglia \(\theta^{\mathcal{B}}_V\) esistesse, per ogni mappa lineare \(f : V\to W\) sarebbe vero che \(\theta_V(v)(v') = \theta_W(fv)(fv')\), cosicché (scegliendo \(f=0\)) \(\theta_V(v)(v')=0\), che significa \(\theta_V=0\), il che è assurdo: abbiamo supposto \(\theta_V\) invertibile.
\end{warning}
\begin{example}\label{il_determinante}\index{Trasformazione naturale!determinante come ---}
	Sia \(n\ge 1\) un numero intero; si può definire un funtore
	\[\dmFun{\text{GL}_n}{\ctcRing}{\ctGrp}\]
	come in \ref{ex_fun_GLn}, e un funtore
	\[\dmFun{(-)^\times}{\ctcRing}{\ctGrp}\]
	che manda un anello commutativo nel suo gruppo delle unità (La notazione in tal senso è stata fissata da \ref{varie_categorie_nella_pratica}).

	Il determinante fornisce un esempio di trasformazione naturale \(\det : \text{GL}_n\natto (-)^\times\) di componenti
	\[\dmFun{\det_R}{\text{GL}_n(R)}{R^\times}\]
\end{example}
Un funtore \(F : \ctC\fun\ctC\) si dice \emph{puntato} (risp., \emph{copuntato}) da una trasformazione naturale \(\eta : \id_\ctC \nat F\) (risp., \(\epsilon : F \nat \id_\ctC\)).
\begin{example}[Co/puntature e co/moltiplicazioni]\index{Funtore!--- co/puntato}
	Raccogliamo alcuni esempi di funtori puntati e copuntati:
	\begin{itemize}
		\item Nella categoria degli insiemi, per un \(A\in\ctSet_0\) fissato, il funtore \(SX := X^A\times A\) è copuntato dalla mappa di riduzione o \emph{valutazione}
		      \[\dmFun{\epsilon_X}{X^A\times A}X\]
		      che manda una coppia \(\pair fa\) in \(\epsilon(f,a) = f(a)\); il fatto che le funzioni \(\epsilon_X\) formino una famiglia naturale in \(X\) (ma non in \(A\)) si riduce al fatto che \((u\cmp f)(a) = u\big(f(a)\big)\) per ogni \(u : X\to Y\) (oppure usando un formalismo diverso, nel fatto che coincidano le due assegnazioni
		      % https://q.uiver.app/#q=WzAsNSxbMCwwLCIoe1xcbGFtYmRhIHQuZnR9XntBXFx0byBYfSxhKSJdLFsyLDAsIihcXGxhbWJkYSB0LmZ0KVthL3RdIl0sWzAsMiwie1xcbGFtYmRhIHQudWZ0fV57QVxcdG8gWX0iXSxbMiwyLCIoe1xcbGFtYmRhIHQudWZ0fV57QVxcdG8gWX0pW2EvdF0iXSxbMiwxLCJcXGxhbWJkYSBwLnVwW2YoYSkvcF0iXSxbMCwxXSxbMSw0XSxbMCwyXSxbMiwzXSxbNCwzLCIiLDAseyJsZXZlbCI6Miwic3R5bGUiOnsiaGVhZCI6eyJuYW1lIjoibm9uZSJ9fX1dXQ==
		      \[\begin{tikzcd}[ampersand replacement=\&]
				      {({\lambda t.ft}^{A\to X},a)} \&\& {(\lambda t.ft)[a/t]} \\
				      \&\& {\lambda p.up[f(a)/p]} \\
				      {{\lambda t.uft}^{A\to Y}} \&\& {({\lambda t.uft}^{A\to Y})[a/t]}
				      \arrow[from=1-1, to=1-3]
				      \arrow[from=1-1, to=3-1]
				      \arrow[from=1-3, to=2-3]
				      \arrow[equals, from=2-3, to=3-3]
				      \arrow[from=3-1, to=3-3]
			      \end{tikzcd}\]
		\item il funtore potenza covariante \(\ctP : \ctSet\fun\ctSet\) di \ref{ex_fun_parti} è puntato dalla famiglia di mappe \(\eta_X : X\to \ctP X\) date da \(x\mapsto \{x\}\); la naturalità si traduce nel fatto (ovvio) che \(\ctP_*(f)(\{x\}) = \{f(x)\}\) per ogni \(f : X\to Y\). Similmente, assegnare a un insieme \(X\) l'insieme delle liste in \(X\), come in \ref{mongruppi_liberi}, è un funtore puntato \((\blank)^* : \ctSet\fun\ctSet\), dalla famiglia \(\theta_X : X\to X^* : x\mapsto x\cons \emptyList = [x]\), e la naturalità di \(\theta\) si traduce nel fatto (ovvio) che \(f^*[x] = [f(x)]\) per ogni \(f : X\to Y\);
		\item mandare una lista \emph{non vuota} nella sua testa \(h_X(x\cons xs) =x\) è una trasformazione naturale dal funtore \((\blank)^+\) delle liste \emph{non vuote} all'identità.
	\end{itemize}
	Oltre a essere puntati o copuntati, i funtori \(S\) e \(\ctP\) di prima hanno associate delle trasformazioni naturali, rispettivamente, di tipo
	\[\xymatrix{
			S \ar@{=>}[r]^-\sigma & S\cmp S & \ctP\cmp\ctP \ar@{=>}[r]^-\bigcup & \ctP
		}\] definite come segue:
	\begin{itemize}
		\item \(\sigma_X\) è una mappa di `comoltiplicazione' \(\dmFun{\sigma_X}{X^A\times A}{(X^A\times A)^A\times A}\)
		      definita mandando \(\pair{\lambda t.ft}a\) in \(\pair{\lambda a'.\pair{\lambda t.ft}{a'}}a\).
		\item \(\bigcup_X\) è la mappa di `unione' \(\dmFun{\bigcup_X}{\pow{(\pow X)}}{\pow X}\)
		      definita mandando \(\fkU = (U_\alpha\mid\alpha\in A) \subseteq \ctP(X)\) in \(\bigcup_{\alpha\in A} U_\alpha \subseteq X\).
		\item Da \ref{nat_tra_posets} sappiamo che una trasformazione naturale \(\alpha : f\nat g\) tra mappe monotòne consiste della disuguaglianza \(f(x)\le g(x)\) vera punto per punto nel loro codominio. Allora, una mappa monotòna \(f : X\to X\) di un insieme ordinato \((X,\le)\) è un funtore puntato se e solo se è una \emph{espansione}:
		      \[\forall x\in X.x\le f(x);\]
		      Dualmente, \(g : Y\to Y\) di un insieme ordinato \((Y,\le)\) è un funtore copuntato se e solo se è una \emph{contrazione}:
		      \[\forall x\in X.g(x)\le x.\]
	\end{itemize}
\end{example}
\begin{definition}[Legge distributiva]\index{Legge distributiva}\index{Funtore!Legge distributiva tra ---i}
	Sia \(\ctC\) una categoria, e \(F,G : \ctC\fun\ctC\) due endofuntori; una \emph{legge distributiva} di \(F\) su \(G\) consta di una trasformazione naturale \(\alpha : FG\nat GF\).
\end{definition}
\begin{example}\index{Funtore!--- delle liste}\index{Legge distributiva}
	Esistono due funtori \(F,G : \ctSet\fun\ctSet\) definiti come segue:
	\begin{itemize}
		\item \(F\) manda un insieme \(X\) nell'insieme delle liste non vuote \(X^+ = \sum_{n\ge 1} X^n = X + X\times X + \dots + X^n + \dots\);
		\item \(G\) manda un insieme \(A\) nell'insieme \(1+A\), dove \(1 = \{\bullet\}\) è un singoletto.
	\end{itemize}
	Esiste una legge distributiva \(\alpha : FG\nat GF\) con componenti
	\[\dmFun{\alpha_A}{\sum_{n\ge 1} (1+A)^n}{1+\sum_{n\ge 1} A^n}\]
	date dall'elidere tutte le occorrenze di \(\bullet\) nell'argomento (se la lista non vuota \((a\cons as)\) è fatta unicamente di \(\bullet\), \(\alpha_A(a\cons as) = \bullet\) nel codominio).
\end{example}
\begin{hExample}[Un esempio di isomorfismo naturale]{tech}\label{puniv_prodotto_naturale}\index{Trasformazione naturale!isomorfismo naturale}
	(La natura di questo esempio sarà chiarita dal capitolo \ref{chap_limiti_colimiti}, sulle proprietà universali, e dal capitolo \ref{cap_aggiunti}, sui funtori aggiunti.)

	Dati tre insiemi \(A,B,C\) possiamo formare il prodotto cartesiano \(B\times C = \{\pair bc\mid b\in B, c\in C\}\) e costruire una biiezione
	% https://q.uiver.app/#q=WzAsMixbMCwwLCJcXEhvbXtcXGN0U2V0fShBLEJcXHRpbWVzIEMpIl0sWzEsMCwiXFxIb217XFxjdFNldH0oQSxCKVxcdGltZXMgXFxIb217XFxjdFNldH0oQSxDKSJdLFswLDFdXQ==
	\[\begin{tikzcd}
			{\varpi^A_{BC}:\Hom{\ctSet}(A,B\times C)} & {\Hom{\ctSet}(A,B)\times \Hom{\ctSet}(A,C)}
			\arrow[from=1-1, to=1-2]
		\end{tikzcd}\]
	mandando una funzione \(u : A\to B\times C\) nella coppia \(\pair{\pi_B\cmp u}{\pi_C\cmp u}\), se \(\pi_B : \pair bc\mapsto b\) e \(\pi_C : \pair bc\mapsto c\) sono le proiezioni sui due fattori del prodotto. Si osservi che
	\begin{itemize}
		\item la funzione così definita è biiettiva. La maniera più conveniente di dimostrarlo è esibirne un'inversa: data una coppia di funzioni \(\pair fg \in \Hom{\ctSet}(A,B)\times \Hom{\ctSet}(A,C)\) definiamo la funzione
		      % https://q.uiver.app/#q=WzAsNCxbMCwwLCJBIl0sWzEsMCwiQlxcdGltZXMgQyJdLFswLDEsImEiXSxbMSwxLCJcXHBhaXJ7ZmF9e2dhfSJdLFswLDFdLFsyLDMsIiIsMCx7InN0eWxlIjp7InRhaWwiOnsibmFtZSI6Im1hcHMgdG8ifX19XV0=
		      \[\begin{tikzcd}[row sep=0]
				      f\pmap g : A & {B\times C} \\
				      a & {\pair{fa}{ga}}
				      \arrow[from=1-1, to=1-2]
				      \arrow[maps to, from=2-1, to=2-2]
			      \end{tikzcd}\]
		      Ovviamente \(\pi_B\cmp(f\pmap g)(a)=\pi_B\pair{fa}{ga}=fa\) (cosicché sono uguali le funzioni \(\pi_B\cmp(f\pmap g)=f\)), e similmente \(\pi_C\cmp (f\pmap g) = g\); viceversa, \(\big((\pi_B\cmp u)\pmap(\pi_C\cmp u)\big)(a) = \pair{\pi_B(u(a))}{\pi_C(u(a))} = u(a)\), dato che un elemento \(t=\pair bc\) del prodotto \(B\times C\) soddisfa sempre l'uguaglianza \(t=\pair{\pi_B t}{\pi_C t}\).
		\item la funzione \(\varpi^A_{BC}\) è naturale in tutti i suoi argomenti, ossia \(\varpi^A_{BC}\) sono le componenti di un isomorfismo naturale \emph{di funtori}.

		      Quanto appena detto significa che per ogni terna \(u : A'\to A\), \(v : B\to B'\) e \(w : C\to C'\) di funzioni, il diagramma di funzioni
		      % https://q.uiver.app/#q=WzAsNCxbMCwwLCJcXEhvbXtcXGN0U2V0fShBLEJcXHRpbWVzIEMpIl0sWzEsMCwiXFxIb217XFxjdFNldH0oQSxCKVxcdGltZXMgXFxIb217XFxjdFNldH0oQSxDKSJdLFswLDEsIlxcSG9te1xcY3RTZXR9KEEnLEInXFx0aW1lcyBDJykiXSxbMSwxLCJcXEhvbXtcXGN0U2V0fShBJyxCJylcXHRpbWVzIFxcSG9te1xcY3RTZXR9KEEnLEMnKSJdLFswLDEsIlxcdmFycGlee0F9X3tCQ30iXSxbMiwzLCJcXHZhcnBpXntBJ31fe0InQyd9Il0sWzAsMiwiXFxIb217XFxjdFNldH0odSx2XFx0aW1lcyB3KSIsMl0sWzEsMywiXFxIb217XFxjdFNldH0odSx2KVxcdGltZXMgXFxIb217XFxjdFNldH0odSx3KSJdXQ==
		      \[\begin{tikzcd}
				      {\Hom{\ctSet}(A,B\times C)} & {\Hom{\ctSet}(A,B)\times \Hom{\ctSet}(A,C)} \\
				      {\Hom{\ctSet}(A',B'\times C')} & {\Hom{\ctSet}(A',B')\times \Hom{\ctSet}(A',C')}
				      \arrow["{\varpi^{A}_{BC}}", from=1-1, to=1-2]
				      \arrow["{\Hom{\ctSet}(u,v\times w)}"', from=1-1, to=2-1]
				      \arrow["{\Hom{\ctSet}(u,v)\times \Hom{\ctSet}(u,w)}", from=1-2, to=2-2]
				      \arrow["{\varpi^{A'}_{B'C'}}", from=2-1, to=2-2]
			      \end{tikzcd}\]
		      è commutativo. (Questa è una verifica semplice e tediosa che lasciamo volentieri come esercizio a chi legge. Data una funzione \(\lambda a.\pair{f_Ba}{f_Ca}\), deve essere vero che il termine
		      \[\pair{\lambda a.\pi_{B'}(\pair {vf_Ba}{wf_Ca})}{\lambda a.\pi_{C'}(\pair {vf_Ba}{wf_Ca})} \in \Hom{\ctSet}(A',B')\times \Hom{\ctSet}(A',C')\]
		      è uguale a\dots{})
	\end{itemize}
	Usando il funtore `diagonale' \(\Delta : \ctSet^\op \fun\ctSet^\op\times\ctSet^\op\) di \ref{es_di_funtori} e ricordando da \ref{exam_bifuntori}.\ref{eb_2} che esiste un funtore prodotto \(\blank\times\blank : \ctSet\times\ctSet \fun\ctSet\) definito mandando \(X,Y\mapsto X\times Y\) e \(u : X\to X',v : Y\to Y'\) in \(u\times v : \pair xy\mapsto \pair{ux}{vy}\), \(\varpi\) consta di una trasformazione naturale invertibile che `riempie' il diagramma
	% https://q.uiver.app/#q=WzAsNixbMCwwLCJcXGN0U2V0Xlxcb3BcXHRpbWVzXFxjdFNldFxcdGltZXNcXGN0U2V0Il0sWzAsMiwiXFxjdFNldF5cXG9wXFx0aW1lc1xcY3RTZXQiXSxbMSwyLCJcXGN0U2V0Il0sWzIsMiwiXFxjdFNldFxcdGltZXNcXGN0U2V0Il0sWzIsMSwiXFxjdFNldF5cXG9wXFx0aW1lc1xcY3RTZXRcXHRpbWVzXFxjdFNldF5cXG9wXFx0aW1lc1xcY3RTZXQiXSxbMiwwLCJcXGN0U2V0Xlxcb3BcXHRpbWVzXFxjdFNldF5cXG9wXFx0aW1lc1xcY3RTZXRcXHRpbWVzXFxjdFNldCJdLFswLDEsIlxcY3RTZXReXFxvcFxcdGltZXNcXFBpIiwyXSxbMSwyLCJcXEhvbXtcXGN0U2V0fSIsMl0sWzMsMiwiXFxQaSJdLFs1LDQsIiIsMCx7ImxldmVsIjoyLCJzdHlsZSI6eyJoZWFkIjp7Im5hbWUiOiJub25lIn19fV0sWzQsMywiXFxIb217XFxjdFNldH1cXHRpbWVzXFxIb217XFxjdFNldH0iXSxbMCw1LCJcXERlbHRhXFx0aW1lc1xcY3RTZXRcXHRpbWVzXFxjdFNldCJdLFsxLDUsIlxcdmFycGkiLDAseyJzaG9ydGVuIjp7InNvdXJjZSI6NDAsInRhcmdldCI6NDB9LCJsZXZlbCI6Mn1dXQ==
	\[\begin{tikzcd}
			{\ctSet^\op\times(\ctSet\times\ctSet)} && {(\ctSet^\op\times\ctSet^\op)\times\ctSet\times\ctSet} \\
			&& {(\ctSet^\op\times\ctSet)\times(\ctSet^\op\times\ctSet)} \\
			{\ctSet^\op\times\ctSet} & \ctSet & {\ctSet\times\ctSet}
			\arrow["{\Delta\times\ctSet\times\ctSet}", from=1-1, to=1-3]
			\arrow["{\ctSet^\op\times\Pi}"', from=1-1, to=3-1]
			\arrow[Rightarrow, no head, from=1-3, to=2-3]
			\arrow["{\Hom{\ctSet}(\blank,\blank)\times\Hom{\ctSet}(\blank,\blank)}", from=2-3, to=3-3]
			\arrow["\varpi", shorten <=60pt, shorten >=60pt, Rightarrow, from=3-1, to=1-3]
			\arrow["{\Hom{\ctSet}(\blank,\blank)}"', from=3-1, to=3-2]
			\arrow["\Pi", from=3-3, to=3-2]
		\end{tikzcd}\]
	Chi legge vedrà presto che esempi del genere (solo, quasi sempre più elaborati di questo) pervadono la teoria delle categorie.
\end{hExample}
\begin{remark}[Gli isomorfismi dell'aritmetica sono naturali]\label{iso_aritmetici}
	Anche questo esempio verrà chiarito dalla teoria dei limiti e colimiti, e dei funtori aggiunti; questi isomorfismi sono però di grande aiuto in varie dimostrazioni nel seguito, e quindi li presentiamo qui.
	Se \(A,B,C\) sono insiemi, tutti gli isomorfismi seguenti sono naturali, in tutti i loro argomenti, tra opportuni funtori somma e prodotto:
	\begin{enumtag}{as}
		\item\label{as_1} \(A+\emptyset\cong \emptyset + A \cong A\) (naturale in \(A\), e \(\emptyset\) è l'insieme vuoto);
		\item\label{as_2} \(A+B\cong B+A\) (naturale in \(A,B\));
		\item\label{as_3} \((A+B)+C\cong A+(B+C)\) (naturale in \(A,B,C\));
		%
		\item\label{as_4} \(A\times \ctTerm\cong \ctTerm \times  A \cong A\) (naturale in \(A\), e \(\ctTerm\) è un insieme singoletto);
		\item\label{as_5} \(A\times B\cong B\times A\) (naturale in \(A,B\));
		\item\label{as_6} \((A\times B)\times C\cong A\times (B\times C)\) (naturale in \(A,B,C\));
		%
		\item\label{as_7} \(A\times(B+C)\cong (A\times B) + (A\times C)\) (naturale in \(A,B,C\)).\footnote{Più precisamente, sono i funtori \(\blank_1\times(\blank_2+\blank_3)\) e \((\blank_1\times \blank_2) + (\blank_1\times \blank_3)\) ad essere isomorfi, dove nel lato destro dell'equazione il secondo funtore fa variare \(\blank_1\) contemporaneamente in entrambi i posti.}
	\end{enumtag}
	Se \(\ctA,\ctB,\ctC\) sono categorie, isomorfismi del tutto analoghi valgono, naturali in tutti gli argomenti, in \ref{as_1} per la categoria vuota di \ref{ex_cat_vuota} e in \ref{as_4} per la categoria terminale di \ref{ex_cat_term}.
\end{remark}
\begin{proposition}\index{Gruppo lineare}\index{Yoneda!lemma di ---}
	Si ricordi che in \ref{il_determinante} e \ref{ex_fun_GLn} abbiamo definito un funtore \(\GL_2 : \ctcRing\fun\ctGrp\); proviamo ora che esiste un isomorfismo naturale (precisamente, una biiezione naturale di insiemi, che però può essere promossa a un isomorfismo di gruppi) tra \(\GL_2\) e il funtore \(\Hom{\ctcRing}(S(\GL_2),-)\) per l'anello \(S(\GL_2)\) ottenuto dal quoziente
	\[\bbZ[a,b,c,d,y]/(1-(ad-bc)y)\]
	in poche parole, \(S(\GL_2)\) è ottenuto aggiungendo formalmente all'anello dei polinomi \(\bbZ[a,b,c,d]\) un inverso all'elemento \(y=ad-bc\). Un omomorfismo di anelli \(S(\GL_2) \to R\) è determinato da un omomorfismo \(\varphi : \bbZ[a,b,c,d,y] \to R\) che scende al quoziente; questa è esattamente la scelta di quattro elementi \(\alpha,\beta,\gamma,\delta\in R\) soggetti alla condizione che \(\alpha\delta-\beta\gamma\) sia un'unità (l'immagine di \(y\) mediante \(\varphi\)), e questo fissa un unico elemento \(\left(\begin{smallmatrix}					\alpha	&	\beta \\					\gamma	&	\delta				\end{smallmatrix}\right)\) di \(\GL_2(R)\).

	La biiezione \(\GL_2(R)\to \ctcRing(S(\GL_2),R)\) così determinata è naturale in \(R\), come è facile vedere dal modo in cui un omomorfismo di anelli \(f : R\to R'\) agisce mediante il funtore \(\GL_2\).

	Questo argomento si generalizza anche facilmente al caso del funtore \(\GL_n\): in quel caso, \(S(\GL_n)\) è il quoziente
	\[\bbZ[\{x_{ij}\mid 1\le i,j\le n\}\cup \{y\}]\big/(1-y\cdot\det)\]
	dove \(\det\) è il polinomio \(\sum_{\sigma\in \fkS(n)}\prod_{i=1}^n x_{i,\sigma(i)}\) nelle indeterminate \(x_{ij}\).
\end{proposition}
Si noti che la condizione di naturalità di \(\beta\cmp\alpha\) si ottiene componendo quelle di \(\alpha\) e \(\beta\):
\[
	\begin{tikzcd}
		FX \ar{d}{Ff} \ar{r}{\alpha_X} & GX \ar{d}{Gf} \ar{r}{\beta_X} & HX \ar{d}{Hf} \\
		FY \ar{r}{\alpha_Y} & GY \ar{r}{\beta_Y} & HY.
	\end{tikzcd}
\]
perché evidentemente, si ha
\begin{align*}
	Hf\cmp (\beta_X\cmp \alpha_X) & = (Hf\cmp \beta_X)\cmp \alpha_X \\
	                              & =(\beta_Y\cmp Gf)\cmp \alpha_X  \\
	                              & =\beta_Y\cmp (Gf\cmp \alpha_X)  \\
	                              & =\beta_Y\cmp (\alpha_Y\cmp Ff)  \\
	                              & =(\beta_Y\cmp \alpha_Y)\cmp Ff.
\end{align*}
\begin{definition}[Composizione orizzontale di trasformazioni naturali]\label{def_hcomp}\index{Composizione orizzontale!di trasformazioni naturali}
	Si considerino tre categorie \(\ctC,\ctD,\ctE\), due funtori \(F,G:\ctC\to\ctD\) e due funtori \(F',G':\ctD\to\ctE\). Date due trasformazioni naturali \(\alpha:F\Rightarrow G\) e \(\alpha':F'\Rightarrow G'\), la \emph{trasformazione naturale composta orizzontalmente} \(\alpha'\horcomp\alpha:F'\cmp F\Rightarrow G'\cmp G\), rappresentata come segue,
	\[
		\begin{tikzcd}[sep=large]
			\ctC \ar[bend left, ""{name=F,below}]{r}{F} \ar[bend right, ""{name=G,above}]{r}[swap]{G}
			& \ctD \ar[bend left, ""{name=FP,below}]{r}{F'} \ar[bend right, ""{name=GP,above}]{r}[swap]{G'}
			& \ctE
			\ar[Rightarrow, from=F, to=G, "\alpha"]
			\ar[Rightarrow, from=FP, to=GP, "\alpha'"]
		\end{tikzcd}
	\]
	è definita in componenti dalla diagonale comune del seguente diagramma in \(\ctE\),
	\[
		\begin{tikzcd}
			F'FX \ar{r}{F'\alpha_X} \ar{d}{\alpha'_{FX}} & F'GX \ar{d}{\alpha'_{GX}} \\
			G'FX \ar{r}{G'\alpha_X} & G'GX
		\end{tikzcd}
	\]
	che commuta per via della naturalità di \(\alpha'\).
\end{definition}
\begin{definition}[\Whisk di un funtore e una trasformazione naturale]\label{def_whiskering}\index{Innesto}
	Date tre categorie \(\ctA,\ctB,\ctC\) e un diagramma di funtori e trasformazioni naturali
	% https://q.uiver.app/#q=WzAsNCxbMCwwLCJcXGN0QSJdLFsxLDAsIlxcY3RCIl0sWzMsMCwiXFxjdEMiXSxbNCwwLCJcXGN0RCJdLFswLDEsIksiXSxbMSwyLCIiLDAseyJjdXJ2ZSI6LTJ9XSxbMiwzLCJIIl0sWzEsMiwiIiwxLHsiY3VydmUiOjJ9XSxbNSw3LCJcXGFscGhhIiwwLHsic2hvcnRlbiI6eyJzb3VyY2UiOjIwLCJ0YXJnZXQiOjIwfX1dXQ==
	\[\begin{tikzcd}
			\ctA & \ctB && \ctC & \ctD
			\arrow["K", from=1-1, to=1-2]
			\arrow[""{name=0, anchor=center, inner sep=0}, curve={height=-12pt}, from=1-2, to=1-4, "F"]
			\arrow[""{name=1, anchor=center, inner sep=0}, curve={height=12pt}, from=1-2, to=1-4, "G"']
			\arrow["H", from=1-4, to=1-5]
			\arrow["\alpha", shorten <=3pt, shorten >=3pt, Rightarrow, from=0, to=1]
		\end{tikzcd}\]
	definiamo \emph{\whisk} sinistro tra il funtore \(H\) e la trasformazione naturale \(\alpha\), denotato \(H \whi \alpha\), la trasformazione naturale di componenti
	\[(H\whi\alpha)_B = H(\alpha_B) : H(FB) \to H(GB)\]
	Dualmente, definiamo \emph{\whisk} destro tra il funtore \(K\) e la trasformazione naturale \(\alpha\), denotato \(\alpha \whi K\), la trasformazione naturale di componenti
	\[(\alpha \whi K)_A = \alpha_{KA} : FKA \to GKA.\]
	\`E facile verificare che queste due definizioni dànno trasformazioni naturali (in un caso, \(H\) applicato a ciascun quadrato di naturalità per \(\alpha\) ne preserva la commutatività; nell'altro, stiamo restringendo \(\alpha\) alle sole componenti della forma \(X=KA\)).
\end{definition}
Chi legge può facilmente verificare che (nelle stesse notazioni di sopra) vale l'uguaglianza
\[(H\whi \alpha)\whi K=H\whi (\alpha\whi K)\]
(cioè su ogni componente si ha l'uguaglianza \(((H\whi \alpha)\whi K)_A= H(\alpha_{KA})=(H\whi (\alpha\whi K))_A\)) la quale rende la scrittura \(H \whi \alpha \whi K\) univocamente definita.

Osserviamo che dalla definizione di composizione orizzontale in \ref{def_hcomp} la composizione \(\cmp\) è un bifuntore (nel senso di \ref{def_bifuntore}): da questo segue
\begin{definition}\label{interchangio}\index{Legge di interscambio}\index{Godement!identità di ---}
	Dato un diagramma di trasformazioni naturali
	\[\xymatrix{
		\ctA \ruppertwocell^F{\alpha}
		\rlowertwocell_H{\beta}
		\ar[r]|G & \ctB
		\ruppertwocell^U{\gamma}
		\rlowertwocell_W{\delta}
		\ar[r]|V & \ctC
		}\]
	possiamo ottenere una trasformazione naturale di tipo \(bla\) in due modi:
	\begin{itemize}
		\item Componendo prima \(\beta\cmp\alpha\) e \(\delta\cmp\gamma\), verticalmente, e poi componendo orizzontalmente queste due:
		      \[(\delta\cmp\gamma)\horcomp(\beta\cmp\alpha) : \xymatrix{
				      \ctA \rrtwocell^{UF}_{WH}{} && \ctC;
			      }\]
		\item Componendo prima \(\gamma\horcomp\alpha\) e \(\delta\horcomp\beta\), orizzontalmente, e poi componendo verticalmente queste due:
		      \[(\delta\horcomp\beta)\cmp(\gamma\horcomp\alpha) : \xymatrix{
				      \ctA \rrtwocell^{UF}_{WH}{} && \ctC.
			      }\]
	\end{itemize}
	L'identità di Godement (detta anche \emph{regola di scambio}) afferma che queste due composizioni sono uguali; in effetti, dato che usando la definizione
	\begin{gather*}
		\big((\delta\cmp\gamma)\horcomp(\beta\cmp\alpha)\big)_A = W\beta_A\cmp W\alpha_A\cmp\delta_{FA}\cmp\gamma_{FA} \\
		\big((\delta\horcomp\beta)\cmp(\gamma\horcomp\alpha)\big)_A = \delta_{HA}\cmp V\beta_A \cmp V\alpha_A \cmp\gamma_{FA}
	\end{gather*}
	la regola di scambio discende dalla commutatività di ogni diagramma
	% https://q.uiver.app/#q=WzAsOSxbMCwwLCJVRiJdLFsxLDAsIlVHIl0sWzIsMCwiVUgiXSxbMCwxLCJWRiJdLFsxLDEsIlZHIl0sWzIsMSwiVkgiXSxbMCwyLCJXRiJdLFsxLDIsIldHIl0sWzIsMiwiV0giXSxbMCwzLCJcXGdhbW1hIEYiLDJdLFszLDYsIlxcZGVsdGFfRiIsMl0sWzEsNCwiXFxnYW1tYSBHIl0sWzQsNywiXFxkZWx0YSBHIl0sWzIsNSwiXFxnYW1tYSBIIl0sWzUsOCwiXFxkZWx0YSBIIl0sWzAsMSwiVVxcYWxwaGEiXSxbMSwyLCJVXFxiZXRhIl0sWzMsNCwiVlxcYWxwaGEiLDJdLFs0LDUsIlZcXGJldGEiLDJdLFs2LDcsIldcXGFscGhhIiwyXSxbNyw4LCJXXFxiZXRhIiwyXV0=
	\[\begin{tikzcd}
			UFA & UGA & UHA \\
			VFA & VGA & VHA \\
			WFA & WGA & WHA
			\arrow["{U\alpha_A}", from=1-1, to=1-2]
			\arrow["{\gamma_{FA}}"', from=1-1, to=2-1]
			\arrow["{U\beta_A}", from=1-2, to=1-3]
			\arrow["{\gamma_{GA}}", from=1-2, to=2-2]
			\arrow["{\gamma_{HA}}", from=1-3, to=2-3]
			\arrow["{V\alpha_A}"', from=2-1, to=2-2]
			\arrow["{\delta_{FA}}"', from=2-1, to=3-1]
			\arrow["{V\beta_A}"', from=2-2, to=2-3]
			\arrow["{\delta_{GA}}", from=2-2, to=3-2]
			\arrow["{\delta_{HA}}", from=2-3, to=3-3]
			\arrow["{W\alpha_A}"', from=3-1, to=3-2]
			\arrow["{W\beta_A}"', from=3-2, to=3-3]
		\end{tikzcd}\]
	che esprime la naturalità di \(\gamma\) e \(\delta\) delle componenti di \(\beta\).
\end{definition}
\begin{remark}\index{Trasformazione naturale!composizione orizzontale di ---i}
	L'\whisk tra un funtore e una trasformazione naturale si può definire in termini della composizione orizzontale: infatti, si verifica facilmente che
	\[\alpha \whi K = \alpha\horcomp\id_K\qquad\qquad H \whi \alpha = \id_H \horcomp\alpha\]
	(cosa che spiega la stessa notazione utilizzata per la composizione orizzontale e per l'\whisk).
\end{remark}
\begin{remark}[A proposito di isomorfismi naturali e \whisk]\index{Innesto}
	L'operazione di \whisk destro \(\alpha \whi K\) di un isomorfismo naturale con un funtore \(K\) (notazioni come in \ref{def_whiskering}) restituisce un isomorfismo naturale, ma ovviamente non è vero il viceversa: se esiste una componente \(\alpha_B : FB \to GB\) che è invertibile e \(K : \ctTerm\fun\ctB\) è il funtore che sceglie \(B\), \(\alpha \whi K\) è un isomorfismo naturale, senza che nessuna delle altre componenti di \(\alpha\) sia invertibile.

	L'operazione di \whisk sinistro \(H \whi \alpha\) di un funtore \(H\) con un isomorfismo naturale \(\alpha\) restituisce un isomorfismo naturale, ma ovviamente non è vero che se \(H(\alpha_B)\) è un isomorfismo in \(\ctD\), allora \(\alpha_B : FB\to GB\) è un isomorfismo in \(\ctC\).
\end{remark}
\begin{hDefinition}[Invertitore]{skip}\index{Trasformazione naturale!invertitore di una ---}
	Data una trasformazione naturale \(\alpha : F\nat G\), la sottocategoria piena \(\ctI(\alpha)\) di \(\ctB\) generata da tutti gli oggetti \(B\in\ctB_0\) tali che \(\alpha_B\) è invertibile ha una certa importanza e perciò prende un nome speciale: si chiama l'\emph{invertitore} di \(\alpha\) (più precisamente, l'invertitore di \(\alpha\) è la categoria insieme al funtore pieno di inclusione \(k : \ctI(\alpha) \mono\ctB\)).
\end{hDefinition}
\begin{definition}[Centro di \(\ctC\)]\label{def_centro}\index{Categoria!centro di una ---}\index{Centro}
	Grazie a \ref{lem_end_monoide} sappiamo che, fissata una categoria \(\ctC\), la classe \(\Nat(\id_\ctC,\id_\ctC)\) è un monoide rispetto all'operazione di composizione; esso si dice il \emph{centro} di \(\ctC\) e viene denotato con \(\zentrum(\ctC)\). La motivazione: se \(\ctC\) ha un solo oggetto, ancora grazie a \ref{lem_end_monoide} essa è un monoide, e \(\zentrum(\ctC)\) ne è precisamente il \emph{centro}, ossia il sottomonoide \(\{x\in M\mid \forall a\in M, \, ax=xa\}\).
\end{definition}
\begin{proposition}\index{Centro!--- di un monoide}
	Il fatto che il centro di una categoria con un solo oggetto sia il centro del monoide ad essa associato, secondo \ref{def_centro}, non è casuale. Infatti il centro di una qualsiasi categoria \(\ctC\) è \emph{sempre} un monoide commutativo.
\end{proposition}
\begin{proof}
	Diamo due dimostrazioni di questo risultato; chi legge ed è a suo agio con una, dovrebbe cercare di familiarizzare con l'altra. Chi conosce in che cosa consista il \emph{trucco di Eckmann-Hilton} riconoscerà questo come un suo caso particolare.
	\begin{itemize}
		\item prima dimostrazione: una trasformazione naturale \(\alpha : \id_\ctC\To\id_\ctC\) consiste di una famiglia di mappe \(\alpha_X : X\to X\) indicizzata dagli oggetti di \(\ctC\), con la proprietà che \(\alpha_Y \cmp f = f\cmp \alpha_X\) per ogni morfismo \(f : X\to Y\) in \(\ctC\); del resto, se \(\beta : \id_\ctC\To\id_\ctC\) è un altro elemento di \(\zentrum(\ctC)\), questo vuol dire che \(\alpha_X \cmp \beta_X = \beta_X\cmp \alpha_X\), cosicché \((\alpha\cmp\beta)_X = (\beta\cmp\alpha)_X\), ossia le trasformazioni naturali \(\alpha\cmp\beta\) e \(\beta\cmp\alpha\) coincidono.
		\item seconda dimostrazione: si noti che vale la seguente catena di uguaglianze tra composizioni orizzontali e verticali (invocando \ref{interchangio} ogni volta che sia necessario):
		      \begin{align*}
			      \raisebox{.3ex}{\begin{tikzpicture}[baseline=(current bounding box.center)]
					                      \almond\alpha\beta
				                      \end{tikzpicture}}
			       & =\raisebox{.3ex}{\begin{tikzpicture}[baseline=(current bounding box.center)]
					                          \almonds\id\id\alpha\beta
				                          \end{tikzpicture}} \\
			       & =\raisebox{.3ex}{\begin{tikzpicture}[baseline=(current bounding box.center)]
					                          \almonds\alpha\id\id\beta
				                          \end{tikzpicture}} \\
			       & =\raisebox{.3ex}{\begin{tikzpicture}[baseline=(current bounding box.center)]
					                          \almonds\id\beta\alpha\id
				                          \end{tikzpicture}} \\
			       & =\raisebox{.3ex}{\begin{tikzpicture}[baseline=(current bounding box.center)]
					                          \almonds\beta\id\id\alpha
				                          \end{tikzpicture}} \\
			       & =\raisebox{.3ex}{\begin{tikzpicture}[baseline=(current bounding box.center)]
					                          \almonds\beta\alpha\id\id
				                          \end{tikzpicture}} \\
			       & =\raisebox{.3ex}{\begin{tikzpicture}[baseline=(current bounding box.center)]
					                          \almond\beta\alpha
				                          \end{tikzpicture}}
		      \end{align*}
	\end{itemize}
	Questo conclude la dimostrazione che \(\beta\cmp\alpha = \alpha\cmp\beta\).
\end{proof}
\begin{hRemark}[A proposito dei diagrammi di trasformazioni naturali]{skip}\index{Trasformazione naturale!incollamento di ---i}
	Questa osservazione fa il paio con \ref{cos_diag_comm}.

	Ci sono tre modi di rappresentare un diagramma commutativo di trasformazioni naturali:
	\begin{enumerate}
		\item come un diagramma commutativo in una categoria di funtori \(\Hom{\ctCat}(\ctA,\ctB)\), che ha per oggetti i funtori \(\ctA\fun\ctB\) e per morfismi le trasformazioni naturali;
		\item come \emph{incollamento naturale}, una tassellazione di una parte di piano fatta di bi-agoni, triangoli,\dots{} della forma
		      \[\xymatrix@R=1mm{
			      \bullet \rtwocell^F_G{\alpha} & \bullet & \bullet \ar@/^1pc/[rr]^P\rruppertwocell<\omit>{\beta}\ar[dr]_R & & \bullet && \bullet \ar[dl]\ar[rr]&& \bullet\ar[dl] \\
			      &&&\bullet\ar[ur]_Q && \bullet \ar[rr]&& \bullet \ultwocell<\omit>{\gamma}
			      }\]
		      che hanno la proprietà di condividere in comune il bordo dato da porzioni del loro dominio;
		\item in maniera equazionale, che si ottiene da 1. e 2. nello stesso modo in cui si passa da una presentazione diagrammatica a una equazionale (e viceversa) in una qualsiasi categoria (in questo caso, una opportuna categoria di funtori).
	\end{enumerate}
	Un esempio concreto: consideriamo un arrangiamento di categorie, funtori e trasformazioni naturali e l'uguaglianza
	\[\xymatrix{
		\ctC \ar@/^1pc/[rr]^F\rrtwocell<\omit>{<1>\beta} \drtwocell^U_V{\alpha}&& \ctD \ar[dl]^G\ar@{}[dr]|=& \ctC \drtwocell<\omit>{<-1>\delta}\ar@/_.5pc/[dr]_V\rrtwocell^F_H{\gamma} && \ctD\ar[dl]^G \\
		&\ctE &&& \ctE
		}\]
	Essa si riscrive nella forma di una equazione \(\alpha\cmp\beta=\delta\cmp (G\whi\gamma)\), che a sua volta è un diagramma commutativo
	\[\xymatrix{
			GF\ar@{=>}[r]^\beta\ar@{=>}[d]_{G\whi\gamma} & U \ar@{=>}[d]^\alpha \\
			GH \ar@{=>}[r]_\delta& V
		}\]
	di morfismi nella categoria \(\Hom{\ctCat}(\ctC,\ctE)\) dei funtori di tipo \(\ctC\fun\ctE\).
\end{hRemark}
\begin{example}\label{ex_cat_preschif}\index{Prefascio}
	Un caso particolare di \ref{def_natutra}, quando la categoria \(\ctD\) è la categoria degli insiemi introdotta in \ref{ex_cat_insiemi} è degno di attenzione: fissata \(\ctC\) (piccola), la categoria i cui oggetti sono i funtori di tipo
	\[\dmFun{F}{\ctC^\op}{\ctSet}\]
	e i cui morfismi \(\alpha : F\nat G\) sono le trasformazioni naturali, si chiama la categoria	dei \emph{prefasci} su \(\ctC\). Questo generalizza \ref{exa_funtori_da_poset}, dove la stessa definizione è stata data quando \(\ctC\) è la categoria ottenuta (mediante \ref{ord_sonocat}) dall'insieme parzialmente ordinato \((P,\le)\).

	Varie notazioni sono utilizzate per indicarla: tra le più comuni ci sono \(\widehat{\ctC}\), \(\ctC^\land\) o \(\PSh(\ctC)\): la meno ambigua, che impiegheremo noi, è \(\presh\ctC\).
\end{example}
\begin{definition}\label{def_yoneda_embeddu}\index{Yoneda!immersione di ---}
	L'\emph{immersione di Yoneda} consiste del funtore
	\[\dmFun{\yon_\ctC}{\ctC}{\presh\ctC}\]
	definito come segue.
	\begin{itemize}
		\item Sugli oggetti, manda \(C\in\ctC_0\) nel funtore hom controvariante di \ref{ex_hom_funtore}.
		\item Sui morfismi, manda \(u : C\to C'\) nella trasformazione naturale \(\yon_\ctC u : \yon_\ctC C \nat \yon_\ctC C'\) la cui componente ad \(X\in\ctC_0\) è la funzione
		      % https://q.uiver.app/#q=WzAsMyxbMCwwLCIoXFx5b25fXFxjdEMgdSlfWCA6Il0sWzEsMCwiXFxIb217XFxjdEN9KFgsQykiXSxbMywwLCJcXEhvbXtcXGN0Q30oWCxDJykiXSxbMSwyXV0=
		      \[\begin{tikzcd}[cramped]
				      {(\yon_\ctC u)_X : \Hom{\ctC}(X,C)} && {\Hom{\ctC}(X,C')}
				      \arrow[from=1-1, to=1-3]
			      \end{tikzcd}\]
		      che manda \(X\xto t C\) in \(X\xto t C\xto u C'\).
	\end{itemize}
	La naturalità di \(\yon_\ctC u\), così come la funtorialità (cioè il fatto che \(\yon_\ctC (v\cmp u)\) è la composizione verticale di \(\yon_\ctC v\) e \(\yon_\ctC u\)) si verificano senza difficoltà (chi legge è invitato a ripetere come esercizio \emph{importante} il ragionamento, eventualmente con notazioni proprie): il fatto che \(\yon_\ctC u\) è naturale significa che data una freccia \(g : X\to Y\), il quadrato
	% https://q.uiver.app/#q=WzAsNCxbMCwwLCJcXEhvbXtcXGN0Q30oWCxDKSJdLFsyLDAsIlxcSG9te1xcY3RDfShYLEMnKSJdLFswLDEsIlxcSG9te1xcY3RDfShZLEMpIl0sWzIsMSwiXFxIb217XFxjdEN9KFksQycpIl0sWzIsMCwiXFxfXFxjbXAgZyIsMl0sWzIsMywidVxcY21wXFxfIiwyXSxbMCwxLCJ1XFxjbXBcXF8iXSxbMywxLCJcXF9cXGNtcCBnIiwyXV0=
	\[\begin{tikzcd}[cramped]
			{\Hom{\ctC}(X,C)} && {\Hom{\ctC}(X,C')} \\
			{\Hom{\ctC}(Y,C)} && {\Hom{\ctC}(Y,C')}
			\arrow["{u\cmp\blank}", from=1-1, to=1-3]
			\arrow["{\blank\cmp g}", from=2-1, to=1-1]
			\arrow["{u\cmp\blank}"', from=2-1, to=2-3]
			\arrow["{\blank\cmp g}"', from=2-3, to=1-3]
		\end{tikzcd}\]
	è commutativo. Questo discende dall'associatività della composizione di funzioni. Il fatto che \(u\mapsto \yon_\ctC u\) soddisfi gli assiomi \ref{f_1}, \ref{f_2} di \ref{def_funtore} segue dalla definizione di composizione verticale: dati \(X \xto g Y \xto h Z\),
	% https://q.uiver.app/#q=WzAsNixbMCwwLCJcXEhvbXtcXGN0Q30oWCxDKSJdLFsyLDAsIlxcSG9te1xcY3RDfShYLEMnKSJdLFswLDEsIlxcSG9te1xcY3RDfShZLEMpIl0sWzIsMSwiXFxIb217XFxjdEN9KFksQycpIl0sWzAsMiwiXFxIb217XFxjdEN9KFosQykiXSxbMiwyLCJcXEhvbXtcXGN0Q30oWixDJykiXSxbMiwwLCJcXF9cXGNtcCBnIl0sWzIsMywidVxcY21wXFxfIiwyXSxbMCwxLCJ1XFxjbXBcXF8iXSxbMywxLCJcXF9cXGNtcCBnIiwyXSxbNCwyLCJcXF9cXGNtcCBoIl0sWzUsMywiXFxfXFxjbXAgaCIsMl0sWzQsNSwidVxcY21wXFxfIiwyXV0=
	\[\begin{tikzcd}[cramped]
			{\Hom{\ctC}(X,C)} && {\Hom{\ctC}(X,C')} \\
			{\Hom{\ctC}(Y,C)} && {\Hom{\ctC}(Y,C')} \\
			{\Hom{\ctC}(Z,C)} && {\Hom{\ctC}(Z,C')}
			\arrow["{u\cmp\blank}", from=1-1, to=1-3]
			\arrow["{\blank\cmp g}", from=2-1, to=1-1]
			\arrow["{u\cmp\blank}"', from=2-1, to=2-3]
			\arrow["{\blank\cmp g}"', from=2-3, to=1-3]
			\arrow["{\blank\cmp h}", from=3-1, to=2-1]
			\arrow["{u\cmp\blank}"', from=3-1, to=3-3]
			\arrow["{\blank\cmp h}"', from=3-3, to=2-3]
		\end{tikzcd}\]
	è commutativo.
\end{definition}
\begin{hRemark}[A proposito del simbolo \(\yon_\ctC\)]{skip}\index{Yoneda}\index{aaa_yo@\(\yon_\ctC\)}
	L'immersione di Yoneda è l'unico funtore di questo testo denotato con un carattere non latino o greco. Si legge come lo `\emph{io}' di p\emph{io}ggia, ed è la prima sillaba, nel sistema di trascrizione hiragana degli ideogrammi giapponesi, del nome di famiglia `Yoneda' (\jap{米田 信夫}; in giapponese il cognome precede il nome), di Nobuo Yoneda (\(*\)1930-\(\dag\)1996), matematico e informatico teorico giapponese.
\end{hRemark}
\begin{remark}\index{Prefascio}\index{Funtore!Prefascio}\index{coprefascio|see {Prefascio}}
	In maniera simile a \ref{ex_cat_preschif} è interessante studiare la categoria (denotata \(\check{\ctC}\), \(\ctC^\lor\) o simili) dei funtori
	\[\dmFun{F}{\ctC}{\ctSet}\]
	e delle trasformazioni naturali. Affinché questa si relazioni a \(\presh\ctC\) in modo che
	\[\ctC^\land = \big((\ctC^\op)^\lor\big)^\op\]
	bisogna però scegliere come morfismi \(F\nat G\) tra due funtori \(F,G : \ctC\fun\ctSet\) le trasformazioni naturali \(G\nat F\). Questa scelta permette di dualizzare \ref{def_yoneda_embeddu} definendo l'\emph{immersione di coYoneda}
	\[\dmFun{\coyon_\ctC}{\ctC}{\copresh\ctC}\]
	definita mandando \(C\in\ctC_0\) nel funtore hom covariante di \ref{ex_hom_funtore}, e una freccia \(f : C\to C'\) nella funzione tra hom insiemi (\emph{ibid.}).

	Si osservi che le immersioni di Yoneda e coYoneda si ottengono `saturando' nel primo o nel secondo argomento il bifuntore hom di \ref{ex_hom_funtore}. Questa che sembra una banalità è una importante relazione tra \(\yon_\ctC,\coyon_\ctC\) e \(\hom_\ctC\).
\end{remark}
Un risultato apparentemente semplice da dimostrare, ma dalle conseguenze molto profonde, la cui dimostrazione riposa sul risultato centrale del capitolo \ref{cap_yoneda} (a cui rimandiamo chi legge), è il seguente:
\begin{proposition}\index{Yoneda!lemma di ---}
	Per ogni categoria piccola \(\ctC\), l'immersione di Yoneda \(\yon_\ctC\) di \ref{def_yoneda_embeddu} e di coYoneda \(\coyon_\ctC\) sono funtori pienamente fedeli.
\end{proposition}
\begin{definition}[Il nervo di un funtore]\label{def_nervo_funtore}\index{Funtore!nervo di un ---}\index{Nervo!---	di un funtore}
	Ogni funtore \(F : \ctC \fun\ctD\) induce un bifuntore
	\[\dmFun{\N F}{\ctC^\op\times\ctD}\ctSet\]
	detto il \emph{nervo} di \(F\), e definito da \((C,D)\mapsto \Hom\ctD(FC,D)\) (e similmente sulle frecce). Se \(\ctC\) è piccola, cosicché la categoria \(\presh\ctC\) sia legittima, possiamo `trasporre' il nervo \(\N F\) ad un funtore
	\[\dmFun{\N F}\ctD{\presh\ctC}\]
	(indicato con lo stesso nome) che manda \(D\) nel funtore \(\Hom\ctD(F\blank,D) : C\mapsto \Hom\ctD(FC,D)\)
\end{definition}
Si noti che il nervo di una categoria \(\ctC\), come definito in \ref{def_nervo_di_una_cat}, non è altro che il nervo del funtore canonico
\[\dmFun{\bsDelta}{\ctFOrd}{\ctCat}\]
che riguarda un insieme totalmente ordinato \([n]\in\ctFOrd\) come la categoria \(\bsDelta[n] :=\{0\to 1\to\cdots\to n\}\) (a sua volta, il funtore \(\bsDelta\) è la restrizione \(t|_{\ctFOrd}\), nella terminologia di \ref{ord_come_cat}).
\begin{esercizi}
	\item Mostrare che la regola di scambio di \ref{interchangio} è equivalente al fatto che la composizione
	\[\dmFun{\cmp}{\ctCat(\ctB,\ctC)\times\ctCat(\ctA,\ctB)}{\ctCat(\ctA,\ctC)}\]
	sia un bifuntore (nel senso di \ref{def_bifuntore}) quando le tre categorie di funtori hanno per morfismi le trasformazioni naturali.
	\item Adottando la stessa strategia di \ref{puniv_prodotto_naturale}, costruire un isomorfismo
	\[\xymatrix{
			(A\times B)+(A\times C)\ar[r] & A\times(B+C)
		}\]
	dove \(\times\) indica il prodotto cartesiano, e \(+\) la somma (unione disgiunta) di insiemi. Costruire una biiezione esplicita, e dimostrare la naturalità in \(A,B,C\) separatamente (cosicché a essere isomorfi siano i `trifuntori' \(({\blank}_1\times {\blank}_2)+({\blank}_1\times {\blank}_3)\) e \({\blank}_1\times ({\blank}_2+{\blank}_3)\)).
	\item Per ogni insieme \(X\) definire la funzione \texttt{inits}
	\[\xymatrix{\mathtt{inits}_X : X^* \ar[r] & (X^*)^*}\]
	dove \(MX = X^*\) è il monoide libero delle liste di elementi di \(X\), come in \ref{mongruppi_liberi}, e
	\[\mathtt{inits}_X\emptyList = [\emptyList] \qquad \mathtt{inits}_X [\tup xn,] = [[x_1],[x_1,x_2],\dots, [\tup xn,]]\]
	Mostrare che \(\mathtt{inits}_X\) sono le componenti di una trasformazione naturale \(M \nat M\cmp M\); mostrare, o trovare un controesempio, alla validità della seguente equazione:
	\[\mathtt{inits}_{MX}\cmp\mathtt{inits}_X = M(\mathtt{inits}_X)\cmp \mathtt{inits}_X.\]
	Domande analoghe (per chi sa interpretare questa notazione) per la trasformazione naturale \(\theta : M\nat M \cmp M\) definita da
	\[\theta_X = \begin{cases}
			\emptyList  & \mapsto [\emptyList]                                                 \\
			(x\cons xs) & \mapsto \texttt{map}\kern.5em (x\cons\blank)\kern.5em \theta_X (xs).
		\end{cases}
	\]
	% si controlla che un lato e l'altro dell'uguaglianza devono essere risp.
	% [[[1],[1]],[[1],[1,2],[1,2]],[[1],[1,2],[1,2]]]
	% [[[1]],[[1],[1,2]],[[1],[1,2],[1,2]],[[1],[1,2],[1,2]]] 
	% la seconda è vera ma riduce a un'espressione molto semplice, [1,2,3] |-> [[1,2,3]]
	\item Scrivere in forma equazionale le condizioni di commutatività seguenti, date mediante diagrammi di incollamento naturale:
	% https://q.uiver.app/#q=WzAsMjIsWzAsMSwiXFxidWxsZXQiXSxbMCwyLCJcXGJ1bGxldCJdLFsxLDEsIlxcYnVsbGV0Il0sWzEsMiwiXFxidWxsZXQiXSxbMSwwLCJcXGJ1bGxldCJdLFsyLDAsIlxcYnVsbGV0Il0sWzIsMSwiXFxidWxsZXQiXSxbMywxLCJcXGJ1bGxldCJdLFs0LDAsIlxcYnVsbGV0Il0sWzUsMCwiXFxidWxsZXQiXSxbNSwxLCJcXGJ1bGxldCJdLFs0LDIsIlxcYnVsbGV0Il0sWzMsMiwiXFxidWxsZXQiXSxbNCwxLCJcXGJ1bGxldCJdLFs3LDEsIlxcYnVsbGV0Il0sWzcsMiwiXFxidWxsZXQiXSxbOCwxLCJcXGJ1bGxldCJdLFs4LDIsIlxcYnVsbGV0Il0sWzksMSwiXFxidWxsZXQiXSxbOSwyLCJcXGJ1bGxldCJdLFsxMCwxLCJcXGJ1bGxldCJdLFsxMCwyLCJcXGJ1bGxldCJdLFswLDRdLFsyLDVdLFs0LDVdLFs1LDZdLFsyLDNdLFszLDZdLFswLDJdLFswLDFdLFsxLDNdLFs4LDEzXSxbMTMsMTBdLFs4LDldLFs5LDEwXSxbNyw4XSxbNywxMl0sWzEyLDEzXSxbMTEsMTBdLFsxMiwxMV0sWzE0LDE2XSxbMTYsMTddLFsxNCwxNV0sWzE1LDE3XSxbMTgsMTldLFsxOCwyMF0sWzIwLDIxXSxbMTksMjFdXQ==
	\[\xymatrix@R=5mm{
		& \bullet \ar[r]& \bullet \dlltwocell<\omit>{\beta}\ar[dd]&& \bullet\ar[r] \ar[dd]& \bullet \ar[dd]&& \bullet \ar[r]\ar[dd]\ddrtwocell<\omit>{\alpha}& \bullet \ar@{}[ddr]|=\ddtwocell^{}_{}{\beta}& \bullet\ddrtwocell<\omit>{\delta}\ddtwocell^{}_{}{\gamma} \ar[r]& \bullet\ar[dd]\\
		\bullet \ar[ur]\ar[dd]\ar[r]& \bullet \ar[ur]\ar[dd]&  & \bullet \ar[dd]\ar[ur]& &  && && \\
		&  & \bullet \ar@{}[ur]|=\ultwocell<\omit>{\gamma} &  &\bullet\ar[r]\ultwocell<\omit>{\delta} & \bullet \uultwocell<\omit>{\epsilon}\dlltwocell<\omit>{\theta} && \bullet \ar[r]& \bullet & \bullet\ar[r] & \bullet\\
		\bullet \ar[r]& \bullet\uultwocell<\omit>{\alpha}\ar[ur] && \bullet \ar[r]\ar[ur]& \bullet\ar[ur]
		}\]
	(Iniziare dando dei nomi opportuni ai funtori in ciascun diagramma, ed usare la composizione orizzontale, verticale, e la regola di scambio.)
	\item Si denoti con \(\{s,t\}\) un insieme con due elementi, ed \(M =\{s,t\}^*\) il monoide libero su un insieme con due elementi, seguendo \ref{mongruppi_liberi}; la categoria degli \emph{autogràfi} consiste della categoria delle rappresentazioni di \(M\), nel senso di \ref{es_fun_repre}.
	\begin{itemize}
		\item Spiegare in che senso un autogràfo si descrive come un (multidi)grafo tale che `l'insieme dei lati coincide con l'insieme dei vertici';
		\item Studiare la struttura di autogràfo \(X^\delta\) corrisponde all'azione banale \(s.x=x=t.x\) su un insieme \(X\) di vertici (che sono anche lati, cosicché l'autogràfo \(X^\delta\) ha per lati degli \(x\) della forma \(\xymatrix@C=4mm{x \ar@(ur,dr)[]^x}\)\dots).
	\end{itemize}
\end{esercizi}
