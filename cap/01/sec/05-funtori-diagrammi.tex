\section{Funtori e diagrammi}\label{sec_funtori}
\subsection{Funtori come frecce}
Dato che i monoidi e i poset si possono guardare come categorie, è una buona domanda chiedere a che tipo di trasformazioni corrispondono gli omomorfismi di monoide di \ref{ex_cat_monoidi} e le funzioni monotone di \ref{ex_cat_ordini}. Questo porta alla definizione di \emph{funtore} tra categorie.
\begin{definition}[Funtore]\label{def_funtore}\index{Funtore}
	Date due categorie \(\ctC\) e \(\ctD\), un \emph{funtore} \(F:\ctC\fun\ctD\) consiste dei seguenti dati:
	\begin{enumtag}{f}
		\item \label{f_1} Una funzione tra classi \(F_0 : \ctC_0\to\ctD_0\) che assegna ad ogni oggetto \(X\) di \(\ctC\) un oggetto \(FX\) di \(\ctD\);
		\item \label{f_2} Una funzione tra classi \(F_1 : \ctC_1\to\ctD_1\) che assegna ad ogni morfismo \(X \xrightarrow{f} Y\) di \(\ctC\) una freccia \(FX \xrightarrow{Ff} FY\) di \(\ctD\). Si noti che da ciò segue che \(F(\dom{f})=\dom{Ff}\) e \(F(\cod{f})=\cod{Ff}\), ovvero che i due quadrati
		% https://q.uiver.app/#q=WzAsOCxbMCwwLCJcXGN0Q18xIl0sWzEsMCwiXFxjdENfMCJdLFszLDAsIlxcY3RDXzEiXSxbNCwwLCJcXGN0Q18wIl0sWzAsMSwiXFxjdERfMSJdLFsxLDEsIlxcY3REXzAiXSxbMywxLCJcXGN0RF8xIl0sWzQsMSwiXFxjdERfMCJdLFswLDEsImQiXSxbMCw0LCJ7Rl8xfSIsMl0sWzEsNSwie0ZfMH0iXSxbNCw1LCJkIiwyXSxbMiw2LCJGXzEiLDJdLFs2LDcsImMiLDJdLFsyLDMsImMiXSxbMyw3LCJGXzAiXV0=
		\[\begin{tikzcd}[cramped]
				{\ctC_1} & {\ctC_0} && {\ctC_1} & {\ctC_0} \\
				{\ctD_1} & {\ctD_0} && {\ctD_1} & {\ctD_0}
				\arrow["d", from=1-1, to=1-2]
				\arrow["{{F_1}}"', from=1-1, to=2-1]
				\arrow["{{F_0}}", from=1-2, to=2-2]
				\arrow["c", from=1-4, to=1-5]
				\arrow["{F_1}"', from=1-4, to=2-4]
				\arrow["{F_0}", from=1-5, to=2-5]
				\arrow["d"', from=2-1, to=2-2]
				\arrow["c"', from=2-4, to=2-5]
			\end{tikzcd}\]
		di funzioni di classe sono commutativi.
	\end{enumtag}
	Le funzioni di classe soddisfano le seguenti proprietà:
	\begin{enumtag}{p}
		\item \label{p_1} \emph{Preservazione delle identità}: per ogni oggetto \(X\) di \(\ctC\), \(F(\id_X)=\id_{FX}\);
		\item \label{p_2} \emph{Preservazione delle composizioni}: per ogni coppia di morfismi componibili \(f\) e \(g\) di \(\ctC\), \(F(g\cmp f)=Fg\cmp Ff\).
	\end{enumtag}
\end{definition}
\begin{remark}\label{def_alternativa_funtore}\index{Funtore}
	Una maniera alternativa di dare la definizione di funtore, che spesso torna utile nella pratica, è la seguente: un funtore \(F : \ctC\fun\ctD\) consiste di una funzione tra classi \(F_0 : \ctC_0\to \ctD_0\) come sopra, e per ogni \(X,Y\in\ctC_0\), una funzione di insiemi
	\[\dmFun{F_{XY}}{\Hom{\ctC}(X,Y)}{\Hom{\ctD}(F_0X,F_0Y)}
	\]
	tale che \(F_{XX}(\id_X) = \id_{F_0X}\) e \(F_{XZ}(g\cmp f) = F_{YZ}(g)\cmp F_{XY}(f)\). Dalle relazioni \(F(\dom{f})=\dom{Ff}\) e \(F(\cod{f})=\cod{Ff}\) e dal fatto che \(\ctC_1 = \bigcup_{XY}\Hom{\ctC}(X,Y)\) discende che \(F_{XY}\) non è altro che \(F_1|_{\Hom{\ctC}(X,Y)}\), e viceversa, la famiglia di funzioni \(F_{XY}\) si `incolla' a un'unica funzione di classe
	\[\dmFun{\bigcup_{XY}F_{XY}}{\bigcup_{XY}\Hom{\ctC}(X,Y)}{\bigcup_{XY}\Hom{\ctD}(F_0X,F_0Y)}\]
	(e il codominio di questa funzione di classe è evidentemente una sottoclasse di \(\ctD_1\)). \`E una prassi quasi universale, però, omettere i pedici ad \(F_{XY}\) e scrivere la definizione nella forma di \ref{p_2}.
\end{remark}
\begin{examples}[Primi esempi di funtori]\label{es_di_funtori}\index{Funtore}
	Raccogliamo alcuni esempi elementari di funtori; la verifica che essi soddisfano le proprietà \ref{p_1} e \ref{p_2} è del tutto elementare e lasciata per esercizio.
	\begin{enumtag}{ef}
		\item \label{exfun_1} Data una categoria \(\ctC\), il \emph{funtore identità} su \(\ctC\) assegna ad ogni oggetto \(X\), \(X\) stesso, e ad ogni morfismo \(f\), \(f\) stesso.
		\item \label{exfun_2} L'esempio più semplice è il funtore \emph{costante} \(\const_D : \ctC\fun\ctD\), definito ogni volta che \(\ctD_0\) non è vuota, per ogni scelta di \(D\in\ctD_0\): per ogni \(C\in\ctC_0\), \(\const_D(C)=D\) e per ogni \(f : C\to C'\), \(\const_D(f)=\id_D\).
		\item \label{exfun_3} Il funtore \emph{diagonale} \(\Delta_\ctC : \ctC \fun\ctC\times\ctC\), il cui codominio è la categoria prodotto di \ref{def_cat_prodotto}, manda \(C\in\ctC_0\) in \((C,C)\) e \(f : C\to C'\) in \((f,f)\).
		\item \index{Categoria!--- somma}\label{exfun_4} il funtore di \emph{riduzione} \(\nabla : \ctC+\ctC \fun\ctC\), il cui dominio è la categoria somma di \ref{def_cat_somma}: gli oggetti e i morfismi di \(\ctC+\ctC\) sono della forma \((C,\epsilon)\) e \(f : (C,\epsilon) \to (C',\epsilon)\) per \(\epsilon=l,r\), e la riduzione è definita semplicemente da \(\nabla(C,\epsilon)=C\) e \(\nabla(f)=f : C\to C'\).
		\item \label{exfun_5} Dati due funtori \(F:\ctC\to\ctD\) e \(G:\ctD\to\ctE\), il \emph{funtore composto} \(G\cmp F:\ctC\fun\ctE\) assegna
		\begin{itemize}
			\item Ad ogni oggetto \(X\) di \(\ctC\), l'oggetto \(GFX\) (ossia \(G(F(X))\)) di \(\ctE\);
			\item Ad ogni morfismo \(f:X\to Y\) di \(\ctC\), il morfismo \(GFf\) (ossia \(G(F(f))\)) di \(\ctE\).
		\end{itemize}
		Il fatto che questo dia un funtore (ovvero che preservi identità e composizione) segue dal fatto che \(F\) e \(G\) risultano componendo le funzioni di classe \(G_0\cmp F_0\) e \(G_1\cmp F_1\):
		\[
			G(F(\id_X)) = G(\id_{FX}) = \id_{GFX} ;
		\]
		\[
			G(F(g\cmp f)) = G(Ff\cmp Fg) = GFf\cmp GFg .
		\]
		\item Ogni funtore \(F : \ctC\fun\ctD\) definisce il \emph{funtore opposto}
		\[\dmFun{F^\op}{\ctC^\op}{\ctD^\op}\]
		ponendo \(F^\op C=FC\) e \(F^\op(Y \xot f X) = FY \xot {(Ff)^\op} FX\); in termini della specifica di un funtore come in \ref{def_alternativa_funtore}, la funzione tra classi \(F^\op_0 : \ctC^\op_0\to\ctD^\op_0\) coincide con \(F_0\), e
		\[\dmFun{F_{XY}^\op}{\Hom{\ctC^\op}(Y,X)}{\Hom{\ctD^\op}(FY,FX)}\]
		è la funzione \(F_{YX} : \Hom{\ctC}(X,Y)\to \Hom{\ctD}(FX,FY)\).
	\end{enumtag}
\end{examples}
\begin{definition}[Funtori pieni, fedeli, essenzialmente suriettivi]\label{classi_di_funtori}\index{Funtore!classi di ---i}\index{Funtore!--- pieno}\index{Funtore!--- fedele}\index{Funtore!--- pienamente fedele}
	\`E conveniente dare un nome speciale ai funtori \(F : \ctC\fun\ctD\) tali che, nelle stesse notazioni di sopra, ogni funzione \(F_{XY}\) sia rispettivamente iniettiva, suriettiva, e biiettiva: \(F : \ctC\fun\ctD\) si dirà
	\begin{itemize}
		\item \emph{pieno} se ciascuna \(F_{XY}\) è una funzione suriettiva; questo significa che
		      \begin{quote}
			      per ogni \(g : FX\to FY\) in \(\ctD\) esiste almeno un \(f : X\to Y\) in \(\ctC\) tale che \(Ff=g\).
		      \end{quote}
		\item \emph{fedele} se ciascuna \(F_{XY}\) è una funzione iniettiva; questo significa che
		      \begin{quote}
			      Dati \(u,v : X\to Y\) morfismi paralleli in \(\ctC\), se \(Fu=Fv\) allora \(u=v\).
		      \end{quote}
		\item \emph{pienamente fedele} se ciascuna \(F_{XY}\) è sia iniettiva che suriettiva (quindi, in virtù di \ref{esempi_di_iso}.\ref{ei_2}, è biiettiva); questo significa che
		      \begin{quote}
			      per ogni \(g : FX\to FY\) esiste \emph{uno e un solo} morfismo \(f : X\to Y\) tale che \(Ff=g\).
		      \end{quote}
	\end{itemize}
\end{definition}
\begin{remark}
	Le tre nozioni sono indipendenti: il funtore di inclusione \(\ctC_\cong \subseteq\ctC\) è fedele, ma non pieno se \(\ctC\) non è un gruppoide. Il funtore \(S : \{0\toto 1\} \fun \{0\to 1\}\) che identifica tra loro le due frecce di \(\{0\toto 1\}\) è pieno, ma non fedele.
\end{remark}
\begin{example}\label{exa_monotone_funtori}\index{Funtore!mappa monotòna come ---}
	Dati due insiemi parzialmente ordinati \((X,\le)\) e \((Y,\le)\) guardati come categorie nel senso di \ref{ord_sonocat}, un funtore \((X,\le)\fun(Y,\le)\) è precisamente una funzione monotòna nel senso di \ref{ex_cat_ordini}.

	Innanzitutto, la funzione \(f:X\to Y\) definisce una corrispondenza sugli oggetti.

	Poi, l'unica freccia \([x\le x']\) in \(X\) determina una freccia \([f(x)\le f(x')]\) se e solo se \(f\) è monotòna.

	L'identità è preservata automaticamente: per ogni \(x\), la riflessività dell'ordine di \(Y\) dà un'unica freccia \([f(x)\le f(x)]\). Con un (innocuo) abuso di notazione, \(f(\id_{x})=\id_{f(x)}\) segue da \(f([x\le x])=[fx\le fx]\).

	Allo stesso modo, anche le composizioni sono preservate automaticamente.
\end{example}

\begin{example}\label{exa_funtori_da_gruppi}\index{Funtore!omomorfismi di monoidi come ---}
	Dati due monoidi \(M\) e \(N\) guardati come categorie nel senso di \ref{mon_sonocat}, un funtore \(\susp M\fun \susp N\) è precisamente un omomorfismo di monoidi nel senso di \ref{ex_cat_monoidi}.

	Innanzitutto, un funtore \(F : \susp M\fun \susp N\) deve mappare l'unico oggetto di \(\susp M\) in quello di \(\susp N\) nell'unico modo possibile (ovviamente questo vale per tutti i funtori il cui codominio è della forma \(\susp N\) per qualche monoide \(N\): la loro funzione \(F_0\) sugli oggetti è costante).

	Sulle frecce, il funtore deve assegnare ad ogni morfismo di \(\susp M\) (ossia, ogni elemento del monoide \(M\)) una freccia di \(\susp N\) (elemento di \(N\)). In altre parole \(F_1\) definisce una funzione \(f:M\to N\).

	Le condizioni \ref{def_funtore}.\ref{p_1} e \ref{def_funtore}.\ref{p_2} sono precisamente le condizioni che affermano che \(F_1=f\) preserva l'identità di \(M\), \(f(1_M)=1_N\), e le composizioni, per ogni \(m,m'\in M\) abbiamo \(f(m\cdot m')=f(m)\cdot f(m')\).
\end{example}
% \paolo{ottima, ma c'è già qualcosa a proposito nell'intro. Meglio tenerne una sola.}
\begin{terminology}[A proposito del nome `funtore']\index{Funtore!etimoogia di `---'}
	\`E sorprendentemente difficile attribuire con precisione la creazione del termine `funtore' a un singolo individuo.

	Mac Lane scrive in \cite[note al cap. 1]{working-categories} che la parola è un prestito da Carnap, e precisamente dalla sua opera `La sintassi logica del linguaggio' \cite{carnappio}. Carnap scrive nel 1934, e fu proprio Mac Lane a recensire il libro in questione (e a distruggere completamente la sua tesi principale con un elementare controesempio preso dall'algebra astratta, che evidentemente Carnap ignorava).

	Nel suo testo \cite{kromer}, Ralf Kr\"omer osserva che la terminologia `corrispondenza funtoriale' è stata, da quel momento in poi, largamente utilizzata senza ulteriore discussione (fino al punto che in \cite{grothendieck1965introduction} la `teoria delle categorie' viene chiamata `linguaggio funtoriale'); del resto a Carnap non va nemmeno il merito di avere coniato per primo la parola.

	Haskell Curry \cite{Curry1961SomeLA} attribuisce più correttamente il termine a Tadeusz Kotarbiński, logico e filosofo polacco, che nel suo testo \emph{Elementy teorji poznania} (scritto cinque anni prima di \cite{carnappio}) introduce i funtori come `connettivi sentenziali' all'inizio del suo capitolo II. In \cite{Curry1961SomeLA} l'autore fa uso del concetto di funtore nel suo senso filosofico (più precisamente, in filosofia del linguaggio, dove un funtore è definito come un operatore che agisce sulle frasi di un dato linguaggio, per formarne altre, `preservando l'analisi logica', ossia le relazioni tra le parti della frase) per costruire il linguaggio giocattolo dei sami, teteli e tanteti.

	Il testo di Kotarbiński ha avuto diverse fortune editoriali: una sua edizione inglese a opera di Olgierd Wojtasiewicz è apparsa solo nel 1966 con una traduzione piuttosto goffa del titolo in \emph{Gnosiology} (sic), che contiene un'appendice scritta da Kazimierz Ajdukiewicz, altro famoso esponente della scuola polacca di logica. Ajdukiewicz usa in maniera sorprendentemente libera il termine `funtore' in frasi come [traduzione degli autori]
	\begin{quote}
		[\dots\unkern] se ciò di cui la matematica parla sono i correlati oggettivi di alcuni funtori che appaiono nei suoi teoremi, correlati che dal canto loro non accettano variabili libere, allora la matematica parla di numeri, perché, ad esempio nell'enunciato aritmetico `\(3 + 2 = 5\)' questi argomenti ultimi sono numeri, e null'altro.
	\end{quote}
	Kotarbiński dal canto suo attribuisce la creazione del termine `funtore' a Jan Łukasiewicz e Stanisław Leśniewski:
	\begin{quotation}
		{\L}ukasiewicz, in his system of the sentential calculus, places the functors directly before the functions to which they pertain. [\dots\unkern]

		other logical types can be formed by sentence-forming or term-forming functors of the various kinds (Leśniewski).
	\end{quotation}
	Ci sono poche evidenze storiche che questa attribuzione sia quella corretta, ma è quasi certo che ad aver coniato il termine per primo sia stato un esponente della scuola polacca di logica: Alan Rose parla in \cite{CM_1968__20__153_0} degli `implication and negation functors of Łukasiewicz' riferendosi ad ancora un altro testo, e qui iniziano a perdersi le tracce storiche. Leśniewski sembra essere quello che per primo utilizza il termine in maniera sistematica, e muore pochi anni dopo la pubblicazione di \cite{carnappio}, ma non sembra nemmeno lui essere il primo a utilizzare la parola.

	\`E ragionevole concludere che il termine sia nato come innovazione lessicale spontanea nella scuola polacca di logica e filosofia del linguaggio, nei primi decenni del ventesimo secolo, e che analogamente ad altri termini nati senza un autore preciso, si sia poi diffuso universalmente.
\end{terminology}
Quando, nel secondo esercizio, prendiamo \(\ctD=\ctSet\) di \ref{ex_cat_insiemi} otteniamo una importante costruzione.
\begin{example}\label{exa_azioni_funtori}\index{Funtore!azioni come ---}
	Dato un gruppo \(G\), un funtore \(\susp G\fun\ctSet\) è precisamente un \emph{\(G\)-insieme}, ovvero un insieme con un'azione di \(G\), nel senso di \ref{ex_cat_g_insiemi}.
	Il funtore mappa l'unico oggetto di \(\susp G\) in un insieme, chiamiamolo \(X\).
	Sulle frecce, un po' come nell'esempio sopra, il funtore assegna ad ogni elemento \(g\) di \(G\) una funzione \(X\to X\), che possiamo scrivere come \(x\mapsto g\cdot x\).
	La preservazione dell'identità e della composizione dicono precisamente che le funzioni \(x\mapsto g\cdot x\) per ogni \(g\) danno un'azione di \(G\) su \(X\).

\end{example}
In maniera simile, un funtore \(\susp G\fun\ctVect[\bbR]\) consiste precisamente di una rappresentazione (lineare) di \(G\) su uno spazio vettoriale nel senso che è noto dall'algebra o dalla fisica matematica; le stesse costruzioni si possono fare con un monoide invece di un gruppo, e ciò giustifica la definizione seguente.
\begin{definition}[Rappresentazione di \(M\)]\label{es_fun_repre}\index{Rappresentazione}
	Sia \(\ctD\) una categoria, ed \((M,\cdot,1)\) un monoide; una \emph{rappresentazione} di \(M\) in \(\ctD\) consiste di un funtore \(\susp M \fun \ctD\).
\end{definition}
La teoria classica delle equazioni differenziali ordinarie dà un esempio importante di rappresentazione di \(G\) in una categoria di spazi topologici, mediante i \emph{gruppi a un parametro} di diffeomorfismi,\footnote{Il nome `gruppo a un parametro' è un relitto ottocentesco dovuto al fatto che si preferiva pensare a un'azione del gruppo additivo \(\bbR\) su una varietà \(X\) come a una famiglia di diffeomorfismi \(\{\varphi_t : X\to X\}_t\) dipendenti --appunto-- da un parametro reale \(t\in\bbR\) (o \(t\in [a,b]\), o \(t\in [0,\infty)\)\dots). Chiaramente, \(\varphi_t = \varphi(t,-)\) per un'azione \(\varphi : \bbR\times X\to X\) mediante diffeomorfismi, cioè per un funtore \(\susp(\bbR,+,0)\fun\cate{Mfd}\) verso una categoria di varietà differenziabili e mappe lisce. La lezione per chi legge è che qualsiasi corso introduttivo ai sistemi dinamici fa uso, più o meno nascostamente, più o meno ammettendolo, del concetto di funtore.} o mediante i campi vettoriali a loro associati.
\begin{example}\label{exa_derivata_funtore}\index{Funtore!differenziale come ---}\index{Differenziale}
	Sia \(\ctEuc\) la categoria degli \emph{spazi reali Euclidei puntati}, definita come segue:
	\begin{itemize}
		\item gli oggetti sono coppie \((n,x)\) con \(n\in\bbN\) e \(x\in\bbR^n\);
		\item i morfismi \((n,x)\to (m,y)\) sono funzioni differenziabili \(f:\bbR^n\to\bbR^m\) tali che \(f(x)=y\).
	\end{itemize}
	Possiamo costruire un funtore \(D:\ctEuc \fun \ctMat[\bbR]\) nel seguente modo:
	\begin{itemize}
		\item sugli oggetti, mappa \((n,x)\) in \(n\);
		\item sui morfismi, assegna alla funzione \(f:(\bbR^n,x)\to (\bbR^m,y)\) il \emph{differenziale} di \(f\) in \(x\), cioè, la mappa lineare \(Df:\bbR^n\to\bbR^m\).
	\end{itemize}
	Questa costruzione è funtoriale perché
	\begin{itemize}
		\item il differenziale della funzione identità \(\id : (n,x) \to (n,x)\) è la matrice identità;
		\item il differenziale della composizione di due funzioni è il prodotto (tra matrici) dei differenziali.
	\end{itemize}
	(Se queste proprietà non fossero chiare, può essere di aiuto scriverle per esteso ricordando \ref{ex_cat_matrici}.)
\end{example}
\begin{definition}[Funtore controvariante]\label{def_funtore_contro}\index{Funtore!--- covariante}\index{Funtore!--- controvariante}
	Definiamo un funtore \emph{controvariante}\footnote{La dicotomia terminologica tra \emph{covarianza} e \emph{controvarianza} viene dalla fisica che, purtroppo, definisce i due concetti in maniera opposta alla convenzione in teoria delle categorie; un funtore controvariante rovescia l'ordine delle frecce a cui viene applicato; ma è un tensore \emph{co}variante a trasformare con la matrice duale del cambio di coordinate.} \(F : \ctC \fun\ctD\) come un funtore \(F : \ctC^\op\fun\ctD\); esso consta di:
	\begin{enumtag}{fc}
		\item \label{fc_1} una funzione tra classi \(F_0 : \ctC_0\to\ctD_0\) che assegna ad ogni oggetto \(X\) di \(\ctC\) un oggetto \(FX\) di \(\ctD\);
		\item \label{fc_2} una funzione tra classi \(F_1 : \ctC_1\to\ctD_1\) che assegna ad ogni morfismo \(Y \xrightarrow{f} X\) di \(\ctC^\op\), cioè ad ogni morfismo \(X\xto f Y\) di \(\ctC\), una freccia \(FX \xrightarrow{Ff} FY\) di \(\ctD\) (si noti che da ciò segue che \(F(\dom{f})=\cod{Ff}\) e \(F(\cod{f})=\dom{Ff}\)): in parole semplici, l'azione di \(F_{XY}\) su una freccia la rovescia,
		\[F(\xymatrix{Y \ar[r]^f & X}) = \xymatrix{FX \ar[r]^{Ff} & FY.}\]
	\end{enumtag}
	La struttura in \ref{fc_1} ed \ref{fc_2} soddisfa le proprietà seguenti:
	\begin{enumtag}{pc}
		\item \label{pc_1} \emph{Preservazione delle identità}: per ogni oggetto \(X\) di \(\ctC\), \(F(\id_X)=\id_{FX}\);
		\item \label{pc_2} \emph{Rovesciamento delle composizioni}: per ogni coppia di morfismi componibili \(f\) e \(g\) di \(\ctC\), \(F(g\cmp f)=Ff\cmp Fg\).
	\end{enumtag}
	\index{Funtore!Prefascio}
\end{definition}
Ciò permette di introdurre una definizione molto importante e pervasiva in teoria delle categorie.
\begin{definition}\label{exa_funtori_da_poset}\index{Prefascio}\index{Funtore!Prefascio}
	Sia \((P,\le)\) un insieme preordinato nel senso di \ref{ex_cat_ordini}; un funtore controvariante
	\[\xymatrix{F : (P,\le)^\op \ar[r] & \ctSet}\]
	si dice un \emph{prefascio} (su \(P\)).\footnote{La parola è un calco strutturale dell'inglese \emph{presheaf}, che è a sua volta un calco dal francese \emph{préfaisceaux}; cf. \cite{Tennison1975,Bredon1997,Godement} e \cite[Cap. VII]{pedicchiofoundations} per una introduzione meno geometrica --ma storicamente molto istruttiva. Per indagare l'etimologia del termine e il suo impiego in matematica, chi legge può consultare l'ottima nota breve di Lovering, \cite{Lovering}.}

	Esplicitamente, un prefascio consiste di
	\begin{itemize}
		\item una corrispondenza sugli oggetti, che assegna a ogni elemento \(x\in P\) un insieme \(Fx\),
		\item una corrispondenza sui morfismi, che assegna a ogni \(x\le y\) una funzione \(Fy\to Fx\),
	\end{itemize}
	che soddisfano \ref{def_funtore}.\ref{p_1}, \ref{def_funtore}.\ref{p_2}.
\end{definition}
L'esempio che motiva la definizione è il seguente: se \((X,\tau_X)\) è uno spazio topologico, \(P=\tau_X\) è l'insieme parzialmente ordinato dei suoi aperti, e un prefascio \(F\) su \(\tau_X\) (o più brevemente, su \(X\)) assegna a ogni aperto \(U\subseteq X\) un insieme \(FU\) e ad ogni inclusione \(U\subseteq V\) una funzione detta \emph{restrizione} \(r_{UV} : FV \to FU\), in modo tale che
\[(U\subseteq V\subseteq W) \Rightarrow r_{UV}\cmp r_{VW} = r_{UW} \qquad\qquad r_{UU} = \id_{FU}\]
Si noti che, anche se si può pensare \(U\subseteq V\subseteq X\) come alle funzioni iniettive che includono \(U\) in \(V\) e \(U,V\) in \(X\), le restrizioni \(r_{UV}, r_{UX}, r_{VX}\) non hanno motivo di essere funzioni iniettive.

Consideriamo uno spazio topologico \(X\) e ad ogni aperto \(U\subseteq X\) associamo l'insieme delle funzioni continue \(U\to \bbR\). Questa associazione definisce un prefascio su \(X\) che soddisfa in più anche le seguenti due proprietà: fissato un aperto \(U\subseteq X\) e un ricoprimento aperto \(\{U_i\}\) di \(U\), vale che
\begin{enumtag}{sh}
	\item \label{sheaf_1} per ogni coppia di elementi \(s,t \in FU\),
	\[(\forall i \in I.r_{U_iU}s = r_{U_iU} t) \,\Rightarrow\, s=t\]
	\item \label{sheaf_2} per ogni famiglia \(s_i \in FU_i\), tale che \(r_{U_{ij}U_i}s_i = r_{U_{ij}U_j}s_j\), esiste un (unico, in virtù della proprietà precedente) \(s\in FU\) tale che \(r_{U_iU}s=s_i\).
\end{enumtag}
\begin{remark}\index{Fascio}\index{Prefascio!fascio}
	Un prefascio che soddisfa le proprietà appena enunciate si dice un \emph{fascio} (\emph{sheaf}: letteralmente, un covone di grano) sullo spazio \(X\). La teoria delle equazioni differenziali ordinarie conduce naturalmente alla nozione di fascio (che storicamente precede la nozione più generale di prefascio; questo spiega la terminologia peculiare), \cite[Cap. II]{mac1992sheaves}, \cite{Tennison1975,Borceux1994a,Borceux1989-gf,pedicchiofoundations}, e la geometria algebrica moderna è edificata su questa nozione (anche quando gli spazi topologici considerati possono avere topologie molto diverse da quella di una varietà reale o complessa).

	Si può parlare più in generale di un fascio (sullo spazio \(X\)) a valori in una categoria \(\ctC\) che sia un costrutto nel senso di \ref{def_costrutto} (cosicché sia possibile parlare degli elementi \(s\in UC\) di un oggetto \(C\in \ctC\)): si tratta di un funtore
	\[\xymatrix{F : (P,\le)^\op \ar[r] & \ctC}\]
	che soddisfa le proprietà \ref{sheaf_1}, \ref{sheaf_2} rispetto agli oggetti di \(\ctC\); tra le scelte più comuni, \(\ctC=\ctAb\) (i gruppi abeliani di \ref{varie_categorie_nella_pratica}), \(\ctC=\ctTop\) (in \ref{ex_cat_top}), \(\ctC=\ctCat\) (in \ref{ex_cat_cat}), \(\ctC = \ctCh(\ctAb)\) (in \ref{ex_cat_chcomples}).
\end{remark}
\begin{definition}[Endofuntore]\label{def_endofuntore}\index{Endofuntore}\index{Funtore!endo---}
	Sia \(\ctC\) una categoria; un \emph{endofuntore} consta di un funtore \(F : \ctC\fun\ctC\).
\end{definition}
La teoria degli endofuntori su una categoria è molto vasta per diversi motivi: uno è che esistono molti esempi, su categorie ricche di struttura (ad esempio, sulla sola categoria degli insiemi, persino la teoria dei funtori `polinomiali', cioè scrivibili come \(FX = \sum_{i\in I} A_i\times X^{E_i}\), per opportune famiglie di insiemi \((A_i,E_i\mid i\in I)\), è molto vasta). Un altro motivo è che la categoria i cui oggetti sono le frecce di \(\ctC\) tra gli oggetti \(FX , X\) è un invariante naturale da assegnare a \(F\):
\begin{definition}[Categoria delle co/algebre per un endofuntore]\index{Categoria!--- delle \(F\)-co/algebre}\index{Endofuntore!co/algebra per un ---}
	Sia \(F : \ctC\fun\ctC\) un endofuntore.

	La categoria delle \emph{algebre per \(F\)}, o brevemente delle \emph{\(F\)-algebre}, ha
	\begin{itemize}
		\item per oggetti le frecce di \(\ctC\) della forma \(a : FA\to A\) (l'oggetto \(A\) è detto \emph{supporto} dell'algebra, e la freccia \(a\) la \emph{mappa strutturale});
		\item per frecce \((A,a)\to (B,b)\) le frecce \(u : A\to B\) di \(\ctC\) tra i supporti, tali che il quadrato
		      \[\xymatrix{
				      FA \ar[r]^{Fu}\ar[d]_a & FB \ar[d]^b\\
				      A \ar[r]_-u & B
			      }\]
		      sia commutativo. La composizione si fa come in \(\ctC\).
	\end{itemize}
	Dualmente, la categoria delle \emph{coalgebre per \(F\)}, o brevemente delle \emph{\(F\)-coalgebre}, ha
	\begin{itemize}
		\item per oggetti le frecce di \(\ctC\) della forma \(a : FA\to A\) (l'oggetto \(A\) è detto \emph{supporto} della coalgebra, e la freccia \(a\) la \emph{mappa strutturale});
		\item per frecce \((A,a)\to (B,b)\) le frecce \(u : A\to B\) di \(\ctC\) tra i supporti, tali che il quadrato
		      \[\xymatrix{
				      A \ar[d]_a\ar[r]^-u & B\ar[d]^b\\
				      FA \ar[r]_{Fu} & FB
			      }\]
		      sia commutativo. La composizione si fa come in \(\ctC\).
	\end{itemize}
\end{definition}
\begin{examples}
	Raccogliamo alcuni esempi di endofuntori interessanti, per lo più sulla categoria degli insiemi: spesso un funtore dalla forma molto semplice ha una interessante categoria delle co/algebre.
	\begin{itemize}
		\item Consideriamo l'endofuntore \(M : \ctSet\fun\ctSet\) che, sugli oggetti, assegna ad \(A\) l'insieme `successore' \(\{\bullet\}+A\), dove \(\bullet\notin A\) così che l'unione sia disgiunta, e sui morfismi agisce concordemente: ad \(f : A\to B\) corrisponde la funzione
		      \[\xymatrix@R=0mm{
			      1+A \ar[r] & 1+B \\
			      \bullet \ar@{|->}[r] & \bullet \\
			      a \ar@{|->}[r] & f(a)
			      }\]
		      Una \(M\)-algebra consiste di un insieme \(A\) e una funzione \(a : 1+A\to A\); evidentemente, \(a(\bullet)\) sceglie un elemento di \(A\), e \(a|_A : A\to A\) è una generica endofunzione, cosicché la categoria delle \(M\)-algebre si identifica alla categoria dei sistemi dinamici di \ref{ex_cat_dyn}. Anche una \(M\)-coalgebra è un oggetto interessante: è una funzione \(f : A\to 1+A\), che assegna a certi elementi di \(A\) il valore \(\bullet\); se interpretiamo \(f\) come un algoritmo e \(f(a_\times)=\bullet\) come `\(f\) ha fallito nel dare un output, a partire da un input \(a_\times\)' (per esempio perché \(f\) non è definita su \(a_\times\)), questo determina una funzione parziale nel senso di \ref{ex_cat_pfun}. Le \(M\)-coalgebre si identificano quindi alla categoria \(\pSet\).
		\item Consideriamo l'endofuntore \(\ctP_\bbN:=\ctP(\bbN\times\blank):\ctSet\fun\ctSet\) che manda un insieme \(X\) nell'insieme dei sottoinsiemi di \(\bbN\times X\), dove \(\bbN\) è l'insieme dei numeri naturali.

		      Una \(\ctP_\bbN\)-coalgebra è una funzione \(\theta : X\to \ctP_\bbN X\), che si identifica a un sottoinsieme \(T\) di \(X\times\bbN\times X\) (grazie a \ref{diff_pres_cat_rel}), e che può venire rappresentato come un \emph{sistema di transizione} indicizzato sui numeri naturali: se denotiamo una terna \((x,a,y)\in T\) come `\(x\xto ay\)', e diciamo che \((x,a,y)\in T\) se `dallo stato \(x\), una transizione di nome \(a\) porta allo stato \(y\)', allora \(\theta(x) = \{(a,y)\mid x\xto ay\}\) è l'insieme degli stati raggiungibili a partire da \(x\). I sistemi di transizione sono un vasto capitolo dell'informatica teorica che si approvigiona continuamente alla teoria delle categorie per i suoi risultati; si veda \cite{sangiorgi2011introduction,sangiorgi2012advanced,Jacobs2016}.

		      Si noti che il funtore \(\ctP_\bbN\) è puntato da una trasformazione naturale \(\eta\) con componenti \(\eta_X : X\to \ctP(\bbN\times X)\), definite da \(x\mapsto\{(0,x)\}\). \`E allora naturale considerare le \(\ctP_\bbN\)-algebre \emph{puntate}, ma descriverle esplicitamente è complicato: se \(U\subseteq \bbN\times X\) e \((x,n)\in U\) diciamo che \(x\) ha peso \(n\); una funzione \(\alpha : \ctP_\bbN X\to X\) consiste di una funzione che ad ogni sottoinsieme di \(\bbN\times X\) associa un elemento di \(X\) che può essere pensato come un insieme di `preferenze' in un insieme \(X\) di candidati, \(\bbN\) è un insieme (infinito, numerabile) di elettori. Ogni sottoinsieme di \(\bbN \times X\) è un insieme di preferenze che gli elettori esprimono (si consentono voti multipli: la relazione \(R\subseteq \bbN\times X\) si legge `\(n\) vota per \(x\)'). Quindi, l'algebra seleziona il candidato in base alle preferenze di voto. Gli omomorfismi di algebre sono quelle funzioni che rispettano le preferenze di voto, e l'elemento \(\alpha(\varnothing)\in X\) è il candidato selezionato se nessuno si presenta alle urne.
		\item Consideriamo l'endofuntore \(S=1+(A\times\blank):\ctSet\fun\ctSet\) che, fissato un insieme \(A\) manda \(X\) in \(1+(A\times X)\).

		      Una \(S\)-coalgebra consiste di una funzione \(\sigma : X\to 1+(A\times X)\) che ad ogni elemento di \(X\) associa o l'unico elemento di \(1=\{\bullet\}\) (in tal caso diciamo che \(\sigma\) termina in \(x\)), o una coppia \((a,x')\), che va ancora interpretata come una transizione \(x\xto ax'\).

		      Una \(S\)-algebra, invece, consiste di una funzione \(\xi : 1+(A\times X)\to X\) che sceglie un elemento \(\xi(\bullet)=x_0\in X\), e tale che \(\xi|_{A\times X}\to A\times X\to X\) sia un \emph{semiautoma deterministico} (con stato iniziale \(x_0\)): l'insieme \(A^*\) delle liste di elementi di \(A\) agisce su \(X\) definendo per induzione
		      \[\bar\xi(\emptyList,x):=x \qquad \bar\xi(a\cons as,x) := \xi(a, \bar\xi(as,x))\]
		\item Quando si considera un preordine \((P,\le)\) come una categoria, alla maniera di \ref{}, un endofuntore consiste di una mappa monotòna \(f : (P,\le) \to (P,\le)\); allora, una \(f\)-algebra consiste di un elemento \(a\in P\) tale che \(f(a) \le a\) (quello che si dice un \emph{punto prefisso} per \(f\)), mentre una \(f\)-coalgebra è un elemento \(a\in P\) tale che \(a \le f(a)\) (quello che si dice un \emph{punto postfisso} per \(f\)). Nella teoria classica degli ordini hanno rilevanza gli elementi \(x_0=\inf\{x\in P \mid f(x) \le x\}\) e \(x_1=\sup\{x\in P \mid x \le f(x)\}\) (quando esistono), che sono rispettivamente la più piccola \(f\)-algebra e la più grande \(f\)-coalgebra. Si noti che in particolare, \(fx_0=x_0\) e \(f(x_1)=x_1\).
	\end{itemize}
\end{examples}
\begin{definition}[Bifuntore]\label{def_bifuntore}\index{Bifuntore}\index{Funtore!bi---}
	Un \emph{bifuntore} è un funtore il cui dominio sia una categoria prodotto come in \ref{def_cat_prodotto}; più precisamente, date tre categorie \(\ctA,\ctB,\ctC\) un bifuntore \(F : \ctA \times \ctB \fun \ctC\) assegna
	\begin{itemize}
		\item a ogni coppia \((A,B)\in \ctA_0\times\ctB_0\) di oggetti un oggetto \(F(A,B) \in\ctC_0\);
		\item a ogni coppia di morfismi \(f : A\to A', g : B\to B'\) una freccia \(F(f,g) : F(A,B)\to F(A',B')\) in \(\ctC\).
	\end{itemize}
	Questi dati soddisfano poi \ref{p_1} e \ref{p_2} in \ref{def_funtore}.
\end{definition}
Si dimostri per facile esercizio che definire \(F(f,\id_B)\) e \(F(\id_A,g)\) è sufficiente a definire \(F(f,g)\) in maniera univoca, dato che il quadrato di morfismi in \(\ctC\)
\[
	\begin{tikzcd}
		F(A,B) \ar[r, "{F(A,g)}"]\ar[d, "{F(f,B)}"'] & F(A,B') \ar[d, "{F(f,B')}"]\\
		F(A',B) \ar[r, "{F(A',g)}"'] & F(A',B')
	\end{tikzcd}
\]
è commutativo, così che \(F(f,g)=F(f,B')\cmp F(A,g) = F(A',g)\cmp F(f,B)\).
\begin{notation}\index{Funtore!--- \(n\)-ario|see {bifuntore}}
	Similmente, possiamo definire un \emph{trifuntore} \(\ctA_1\times\ctA_2\times\ctA_3\fun\ctB\), un funtore \(n\)-ario \(\ctA_1\times\dots\times\ctA_n\fun\ctB\) e in generale, per ogni insieme \(I\) di indici, un funtore di arietà \(I\) come \(\prod_{i\in I}\ctA_i\fun\ctB\) (la definizione di prodotto in \ref{def_cat_prodotto} si generalizza nel modo ovvio, in una maniera che sarà il \autoref{chap_limiti_colimiti} a precisare).
\end{notation}
\begin{hExample}[Funtore hom co- e controvariante, bifuntore hom]{fund}\label{ex_hom_funtore}\index{Funtore!hom --- covariante}\index{Funtore!hom --- controvariante}
	Sia \(\ctC\) una categoria, e \(C\) un suo oggetto. Definiamo un funtore
	\[\dmFun{\Hom{\ctC}(-,C)}{\ctC^\op}{\ctSet}\]
	come segue:
	\begin{itemize}
		\item un oggetto \(Y\) viene mandato nell'insieme \(\Hom{\ctC}(Y,C)\) dei morfismi \(u : Y\to C\);
		\item una freccia \(f : Y\to Y'\) viene mandato nella funzione
		      \[\xymatrix{
				      \Hom{\ctC}(Y',C) \ar[r] & \Hom{\ctC}(Y,C)
			      }\]
		      che manda \(u : Y'\to C\) in \(u\cmp f : Y\to Y' \to C\).
	\end{itemize}
	Le condizioni \ref{p_1}, \ref{p_2} si riducono agli assiomi di categoria: si tratta di
	\begin{itemize}
		\item verificare che \((u\cmp f)\cmp g = u\cmp (f\cmp g)\); vero, perché la composizione è associativa;
		\item verificare che \(u\cmp \id_Y = u\); vero, perché la composizione ha le identità di \(\ctC\) come elementi neutri.
	\end{itemize}
	Dualmente, possiamo definire un funtore hom covariante \(\Hom{\ctC}(C,-)\), fissando un oggetto \(C\in\ctC_0\) e definendo la funzione sugli oggetti come \(\Hom{\ctC}(C,-)_0 : Y\mapsto \Hom{\ctC}(C,Y)\) e \(\Hom{\ctC}(C,f) : (u : C\to Y)\mapsto f\cmp u\) per \(f : Y\to Y'\).

	\index{Funtore!hom ---} In effetti, entrambi i funtori hom covarianti e controvarianti nascono fissando la componente (destra per il funtore hom controvariante, sinistra per il covariante) del \emph{bifuntore hom} di \(\ctC\),
	\[\xymatrix{\Hom\ctC(-,-) : \ctC^\op\times\ctC \ar[r] & \ctSet}\]
	definito come segue:
	\begin{itemize}
		\item Sugli oggetti, \((X,Y)\mapsto \Hom{\ctC}(X,Y)\);
		\item sui morfismi, il morfismo \((u : X'\to X, v : Y\to Y')\) induce la funzione
		      \[\xymatrix{\Hom{\ctC}(X,Y) \ar[r] & \Hom{\ctC}(X',Y')}\]
		      che manda \(h : X\to Y\) in \(v\cmp h\cmp u\).
	\end{itemize}
	Con questa definizione, è evidente che se \(f : A\to B\) è una freccia di \(\ctC\), \(\Hom\ctC(f,X)\) agisce come \(\Hom\ctC(f,\id_X)\) e \(\Hom\ctC(Y,f)\) agisce come \(\Hom\ctC(\id_Y,f)\); ma allora, quando sia definito, \(\Hom\ctC(f,g)\) agisce come \(\Hom\ctC(f,\id_\bullet)\cmp\Hom\ctC(\id_\bullet,g)\), cioè come \(\Hom\ctC(\id_\bullet,g)\cmp\Hom\ctC(f,\id_\bullet)\).
\end{hExample}
\begin{examples}[Alcuni esempi di bifuntori]\label{exam_bifuntori}\index{Bifuntore}\index{Funtore!bi---}
	Raccogliamo alcuni esempi di funtori di due o più variabili, tra le categorie introdotte finora.
	\begin{enumtag}{eb}
		\item\label{eb_1} il bifuntore hom di \ref{ex_hom_funtore}, che esiste per ogni categoria \(\ctC\);
		\item\label{eb_2} Esiste un bifuntore \(\blank\times\blank : \ctSet\times\ctSet \fun\ctSet\) che manda un oggetto \((X,Y)\) del prodotto di categorie \(\ctSet\times\ctSet\) nell'insieme \(X\times Y\) e i morfismi \((f :A \to X, g : B\to Y)\) nel morfismo \(f\times g : A\times B \to X\times Y : (a,b)\mapsto (f(a),g(b))\).
		\item\label{eb_3} Esiste un bifuntore \(\blank\oplus\blank : \ctVect\times\ctVect \fun\ctVect\) che, similmente a prima, manda due spazi vettoriali \(V,W\) nella loro \emph{somma diretta}, e due mappe lineari \(f : V\to V',g : W\to W'\) nella mappa lineare \(\left(\begin{smat} f & 0 \\ 0 & g \end{smat}\right) : V\oplus W \to V'\oplus W'\). Questo e il precedente esempio sono casi particolari del \emph{prodotto} di due oggetti rispettivamente nelle categorie \(\ctSet,\ctVect\).
		\item\label{eb_4} \ref{ex_cat_matrici} (le matrici a coefficienti in un campo) fornisce un esempio di categoria dove i morfismi sono bifuntori: una matrice \(m\times n\) a coefficienti in \(k\) infatti si può vedere come una funzione \(\bkt m\times \bkt n \to \bbF\), e considerare \(\bkt m = \{\iter n\}\) come una categoria discreta (\ref{ex_cat_discreta}) motiva la definizione seguente.
	\end{enumtag}
\end{examples}
\begin{definition}[Categoria delle matrici di insiemi]\index{Categoria!--- delle matrici}
	Definiamo la categoria \(\ctM(\ctSet)\) che ha
	\begin{itemize}
		\item per oggetti gli insiemi \(A,B,C,\dots\);
		\item per morfismi \(A\to B\) i funtori
		      \[\dmFun M{\ctSet/A}{\ctSet/B}\]
		      che mandano un oggetto \(\var EA\) in un oggetto \(\var{M_0A}B\).
	\end{itemize}
	La composizione è la composizione di funtori:
	\[\xymatrix{
			\ctSet/A \ar[r]^M & \ctSet/B \ar[r]^N & \ctSet/C
		}\]
	è la composizione \(N\cmp M\).
\end{definition}
La definizione appena data merita alcune spiegazioni: si può considerare una freccia \(M\in\Hom{\ctM(\ctSet)}(A,B)\) come una `matrice di insiemi' \((\tilde M(a,b)\mid (a,b)\in A\times B)\), dove \(\tilde M(a,b)\) è definito a partire dai morfismi \(\var[\lceil a \rceil]{*}{A}\) che `scelgono' \(a\in A\),
\[\tilde M(a,b) := \{x\in M_0a\mid (M\lceil a \rceil)(x)=b\}\]
(\(M\lceil a \rceil\) è, per definizione, un oggetto \(\var{M_0a}{B}\) dell'incubo sopra \(B\)).
Ora, dare una famiglia di insiemi \(\tilde M(a,b)\) al variare di \(A\times B\) è equivalente a dare un funtore \(\tilde M : A\times B \fun\ctSet\) dal prodotto di due categorie discrete; a sua volta questo è equivalente a dare un funtore \(M : \ctSet/A \fun\ctSet/B\) come sopra, il quale soddisfa una certa proprietà (è un `aggiunto sinistro', si veda il \autoref{cap_aggiunti}) che lo rende univocamente determinato dagli \(\tilde M(a,b)\).
\begin{remark}\index{Funtore!--- controvariante}
	Ogni funtore \(F : \ctC\fun\ctD\) definisce un funtore \(F^\op : \ctC^\op \fun\ctD^\op\) che coincide con \(F\) sugli oggetti, e che è definito sui morfismi mandando \(f^\op\in\ctC^\op\) in \(F(f)^\op\in\ctD^\op\). \`E semplice (e noioso) verificare le proprietà \ref{p_1}, \ref{p_2}.

	Dal momento che \((\ctC^\op)^\op = \ctC\), il dato di un funtore controvariante \(F : \ctC\fun\ctD\), cioè di un funtore \(F : \ctC^\op \fun\ctD\), è esattamente equivalente al dato di un funtore \(F^\op : \ctC \fun \ctD^\op\). \`E a volte più conveniente considerare l'uno piuttosto che all'altro funtore, riferendosi alla sua controvarianza.
\end{remark}
\begin{example}\label{exa_funtori_liberi_forgetti}\index{Funtore!--- libero e dimenticante}
	Se \((\Omega,a)\) è una segnatura algebrica come in \ref{ex_cat_sigma_strutture}, esiste un funtore
	\[
		\begin{tikzcd}
			U : \ctMod(\Omega,a) \ar[r] & \ctSet
		\end{tikzcd}
	\]
	che manda ogni modello in \(\ctMod(\Omega,a)\) nel suo \emph{insieme soggiacente} \(X\).
\end{example}
\begin{example}[Sui funtori liberi]\label{funtori_liberi}\index{Funtore!--- monoide libero}
	Ogni segnatura algebrica \((\Omega,a)\) ammette \emph{modelli liberi}, nel senso che è possibile definire un funtore \(\ctSet\fun\ctMod(\Omega,a)\) che soddisfa una certa proprietà (di `aggiunzione', si vedrà nel capitolo \ref{cap_aggiunti}). Esiste un modo uniforme di costruire questo funtore data la segnatura \((\Omega,a)\), ma questo livello di generalità è prematuro ora. Ci concentriamo su un esempio tra i più elementari, e tuttavia esemplificativo della procedura generale, che è stato già delineato in \ref{mongruppi_liberi}: il \emph{monoide libero} \(E^*\) su un insieme \(E\) definisce un funtore \(\ctSet\fun\ctMon\), dato che una funzione \(f : X\to Y\) ha l'ovvia `estensione' \(f^* : X^*\to Y^*\) che manda una parola \([\tup xn{}]\) nella parola \([\tup {fx}n{}]\). Le proprietà \ref{f_1} ed \ref{f_2} di \ref{def_funtore} sono immediate da verificare.
\end{example}
\begin{definition}[La categoria delle categorie]\label{ex_cat_cat}\index{Categoria!--- delle categorie}
	Come si può verificare, la composizione di funtori come definita sopra è associativa, e con il funtore identità definito sopra, soddisfa l'assioma di identità. Per questo le categorie e i funtori formano una categoria.

	Nel dare questa definizione è però importante essere precisi: la categoria delle categorie così definita non può essere un oggetto di sé stessa, o avremmo appena creato un paradosso molto fastidioso. Quella che chiamiamo `categoria delle categorie' è la categoria larga \(\ctCat\)  formata prendendo
	\begin{itemize}
		\item come oggetti tutte le categorie \emph{piccole};
		\item come morfismi i funtori tra due categorie piccole.
	\end{itemize}
\end{definition}
\begin{remark}[La categoria delle categorie esiste davvero?]\index{Categoria!--- delle categorie}
	La `categoria' \(\cate{CAT}\) di tutte le categorie localmente piccole è `illegittima' perché non è localmente piccola: infatti la classe dei funtori \(\ctC \fun\ctD\) tra due categorie larghe non è un insieme.\footnote{Usando la definizione, è possibile mostrare che c'è una classe propria di funzioni di classe \(F_0 : \ctC_0 \to \ctD_0\) e \(F_1 : \ctC_1 \to \ctD_1\) che soddisfano \ref{p_1},\ref{p_2}.} Questo può sembrare un puntiglio terminologico, ma è causa di diversi problemi (ad esempio: per \(\ctCAT\) non esistono i funtori hom di \ref{ex_hom_funtore}; quando avremo definito una trasformazione naturale in \ref{sec_tnat} vedremo che anche la categoria \([\ctCat,\ctSet]\) dei funtori da \(\ctCat\) a \(\ctSet\) è illegittima, e che questo è un problema --per esempio, per \(\ctCat\) la mappa \(\yon\) di \ref{def_yoneda_embeddu} non esiste, e nessuno dei risultati del capitolo \ref{cap_yoneda} è valido).

	Per rimarcare la distinzione tra categorie legittime e illegittime, vari autori ricorrono a diverse terminologie: Mac Lane in \cite{working-categories} le chiama `metacategorie' e \cite{acc} le chiama `quasicategorie' (termine che, però, ha radicalmente cambiato significato nel corso degli ultimi quindici anni, riferendosi a strutture definibili in topologia algebrica, che somigliano a categorie ma dove gli assiomi \ref{cp_1}, \ref{cp_2} `valgono a meno di omotopia'). Raramente faremo riferimento a \(\ctCAT\), e mai parlando della sua struttura di `categoria'; avremo invece spesso bisogno di formare categorie del tipo \([\ctC,\ctSet]\) (o simili: \([\ctC,\ctAb]\), \([\ctC,\ctVect]\) eccetera, ma \emph{solo quando \(\ctC\) è piccola} (in questo caso queste categorie sono tutte legittime).
\end{remark}
\begin{definition}[La categoria dei gruppoidi]\index{Categoria!--- dei gruppoidi}\label{ex_cat_gruppoidi}
	La categoria \(\ctGpd\) dei gruppoidi (piccoli) è definita analogamente a \(\ctCat\); i suoi oggetti sono i gruppoidi (nel senso di \ref{def_gruppoide}),	e i suoi morfismi sono i funtori tra gruppoidi (la sottocategoria \(\ctGpd\) è piena in \(\ctCat\), dato che i funtori tra gruppoidi sono tutti e soli i funtori  tra le categorie sottostanti; qualcosa di simile accade per la sottocategoria, piena, dei gruppi dentro quella dei monoidi).
\end{definition}
\begin{definition}\label{def_cat_cocomma}\index{Categoria!--- comma}
	Consideriamo una cospanna di funtori
	\[\begin{tikzcd}
			\ctC \ar[r, "F"'] & \ctE &\ar[l, "G"] \ctD
		\end{tikzcd}\]
	e definiamo la \emph{categoria comma}\footnote{`Comma' va qui inteso nella sua letterale accezione latina di `sezione di un verso' o di una frase (per questo quindi il plurale di comma è \emph{commæ}). Invece di tutti i morfismi di \(\ctE\), la comma \((F/G)\) considera unicamente quelli che connettono oggetti nelle rispettive immagini di \(F\) e \(G\).} \((F/G)\) come segue:
	\begin{itemize}
		\item gli oggetti di \((F/G)\) sono terne \((C,D,f)\) dove \(C\in\ctC_0,D\in\ctD_0\) e \(f : FC\to GD\) è una freccia di \(\ctE\);
		\item una freccia \((C,D,f)\to (C',D',g)\) consiste di una coppia \((u : C\to C', v : D\to D')\) tale che il quadrato
		      \[\begin{tikzcd}
				      FC \ar[r, "f"]\ar[d, "Fu"'] & GD\ar[d, "Gv"] \\
				      FC' \ar[r, "g"'] & GD'
			      \end{tikzcd}\]
		      sia commutativo.
	\end{itemize}
	L'identità e la composizione si fanno come in \(\ctD\), cioè l'identità di \((C,D,f)\) è \(\id_C,\id_D\) e la composizione è definita da
	\[\begin{tikzcd}
			(C,D,f) \ar[r, "{(u,v)}"] & (C',D',g) \ar[r, "{(u',v')}"] & (C'',D'',h)
		\end{tikzcd}\]
\end{definition}
\begin{remark}[Alcuni esempi di categorie comma]\index{Categoria!--- comma}
	Importanti esempi di categorie sono particolari categorie comma per opportune scelte di \(F\) o \(G\); l'esercizio \ref{gfdpgubai_5} invita chi legge a esplicitare, tra le altre, \((F/E)\) e \((E/G)\) dove \(E\) è un oggetto guardato come funtore \(E : \ctTerm \fun\ctE\) alla maniera di \ref{tante_cose_sono_diag}.
\end{remark}
\begin{definition}[La categoria degli elementi di un funtore]\label{elts_F}\index{Categoria!--- degli elementi}\index{aaa_EltsF@\(\Elts{}F\)}
	Ad ogni funtore \(F : \ctC\fun\ctSet\) si associa una categoria \(\Elts \ctC F\), detta \emph{categoria degli elementi} di \(F\), che ha
	\begin{itemize}
		\item per oggetti le coppie \((C,x)\) dove \(C\in\ctC_0\) è un oggetto di \(C\), e \(x\) un elemento di \(FC\);
		\item per morfismi \((C,x)\to (C',y)\) i morfismi \(f : C\to C'\) in \(\ctC\) tali che la funzione \(Ff : FC\to FC'\) `preserva il punto base', cioè manda \(x\) in \(y\).
	\end{itemize}
	La composizione si fa come in \(\ctC\), \(\Id[C] : (C,x)\to (C,x)\) è l'identità dell'oggetto \((C,x)\).
\end{definition}
Più in generale, per ogni oggetto \(D\in\ctD_0\) e funtore \(F : \ctC\fun\ctD\), definiamo la categoria degli elementi di \(F\) in \(D\) come la categoria degli elementi \(\Elts\ctC{\Hom\ctD(D,F\blank)}\) dove \(\Hom\ctD(D,F\blank) : C\mapsto\Hom\ctD(D,FC)\); la \(\Elts\ctC F\) precedente si recupera quando \(F\) ha codominio \(\ctSet\), per \(D=\{*\}\) un singoletto.

La costruzione si dà anche per un funtore controvariante \(F : \ctC^\op\fun\ctSet\), la cui categoria degli elementi ha oggetti \((C,x)\) come sopra, e morfismi \(f : (C,x)\to (C',y)\) tali che \(Ff : FC'\to FC\) manda \(y\) in \(x\).
\paolo{Il contenuto è ottimo, ma secondo me è messa troppo presto: dobbiamo prima dare al lettore qualche esempio non banale di funtore, e in particolare l'idea di funtore dimenticante. Quindi muoverei la digressione più avanti (diciamo, almeno dopo la definizione 2.5.10, o più avanti ancora), dando prima esempi di funtori dimenticanti antemponendo l'esempio 2.5.24. Oppure additittura mettendo funtori dimenticanti tra gli esempi della sezione 2.5.2.}
\begin{hDigression}[Costrutti e loro proprietà]{skip}\index{Costrutto}\index{Funtore!--- fedele}\index{Categoria!--- concreta}\label{def_costrutto}
	La nozione di funtore fedele permette di formalizzare l'idea che alcune categorie nascano considerando `insiemi dotati di struttura' e funzioni compatibili con quella struttura. In \cite[I.5.1]{acc} un \emph{costrutto} è definito come una coppia \((\ctC,U)\) dove \(U : \ctC \fun\ctSet\) è un funtore fedele (e più in generale un costrutto \emph{sopra \(\ctX\)} consiste di una categoria \(\ctC\) con un funtore fedele \(U : \ctC\fun\ctX\)), che permette di pensare a ogni oggetto \(C\) di \(\ctC\) come ad un insieme \(UC\), e ad ogni freccia \(f : C\to C'\) come ad una particolare funzione \(Uf : UC\to UC'\). In questo caso, si dice che la categoria \(\ctC\) nel costrutto \((\ctC,U)\) è resa \emph{concreta} da \(U\). Una categoria \(\ctC\) tale che esiste un \(U : \ctC \fun\ctSet\) che la rende un costrutto si dice \emph{concretizzabile}; sorprendentemente (la dimostrazione è tutt'altro che elementare ed è dovuta a Peter Freyd), la categoria delle classi di omotopia di funzioni continue di \ref{ex_cat_hotop} non è concretizzabile \cite{Freydconc}.

	Si può definire, da qui, la categoria \(\cate{Cons}\) dei costrutti che ha per morfismi \(K : (\ctC,U)\fun (\ctD,V)\) quei funtori \(K : \ctC\fun\ctD\) compatibili con \(U,V\) nel senso ovvio che \(V\cmp K=U\). Più formalmente, al netto di un problema di taglia, si può pensare la categoria \(\cate{Cons}\) come la sottocategoria della comma \(\cate{CAT}/\ctSet\) sui soli oggetti \(\ctC\fun\ctSet\) che sono funtori fedeli. Quel che però è più importante è che la nozione di \emph{isomorfismo} in \(\cate{Cons}\) formalizza l'idea che si possano avere presentazioni differenti di uno stesso costrutto. Ad esempio, esiste una categoria degli spazi topologici, definita in \ref{ex_cat_top}, che è un costrutto rispetto all'ovvio funtore \(U : \ctTop\fun\ctSet : (X,\tau)\mapsto X\) che dimentica la topologia \(\tau\) di uno spazio \((X,\tau)\).

	Lo stesso spazio può però essere presentato, invece che mediante una famiglia di `aperti' \(\tau \subseteq \pow X\) chiusa per intersezioni finite e unioni arbitrarie, mediante una famiglia di `chiusi' \(\gamma\subseteq \pow X\) che soddisfa i ben noti assiomi duali (e una funzione sarà una freccia in \(\ctTop_\gamma\) se e solo se la controimmagine di un chiuso è un chiuso), oppure mediante una famiglia di filtri di intorni \((\fkF_x \mid x\in X)\) (e ci sarà una opportuna nozione di freccia nella categoria \(\ctTop_\fkF\) così determinata), o ancora mediante un operatore di chiusura \(k_X : \pow X\to \pow X\) (e ci sarà una opportuna nozione di freccia nella categoria \(\ctTop_k\) così determinata), eccetera.

	Sono, ora, veri i seguenti fatti:
	\begin{itemize}
		\item le categorie \(\ctTop_\fkF,\ctTop_\tau,\ctTop_\gamma,\ctTop_k,\dots\) degli spazi topologici specificati mediante famiglie di filtri, mediante famiglie di aperti, di chiusi, mediante operatori di chiusura, etc., sono tutte isomorfe (ossia esistono isomorfismi tra ogni coppia, nella `categoria' \(\cate{CAT}\));
		\item non solo: sono equivalenti \emph{in quanto costrutti}, ossia mediante dei funtori compatibili con i funtori dimenticanti di ciascuna categoria.
	\end{itemize}
	La nozione di costrutto permette anche di formalizzare una differenza sostanziale tra gli esempi di categoria dati in \ref{varie_categorie_nella_pratica} e gli esempi dati da un lato in \ref{varie_categorie_nella_pratica}, e dall'altro in \ref{po_wo_to} e \ref{ex_cat_top}: i primi sono di natura \emph{algebrica}, e i secondi di natura \emph{topologica}, e questa è una dicotomia fondamentale.

	Dato un costrutto \((\ctC,U)\) la \emph{fibra} sopra un insieme \(X\) è la sottocategoria formata dagli oggetti \(C\) tali che \(UC=X\) (tutti i \(\ctC\)-oggetti che hanno un dato supporto \(X\)), dove è definita una relazione di preordine \(C\le C'\) se e solo se \(\id_X : UC \to UC'\) è immagine di una freccia \(i : C\to C'\) in \(\ctC\). In categorie del tipo definito in \ref{varie_categorie_nella_pratica}, la fibra sopra un oggetto è preordinata in modo banale, nel senso che \(C\le C'\) se e solo se \(C=C'\); per contro, nelle categorie definite in \ref{po_wo_to} e \ref{ex_cat_top} la fibra sopra un insieme ordinato (risp., uno spazio topologico) è fatta dal poset delle relazioni d'ordine (risp., delle topologie) su un dato insieme \(X\), ordinato dalla relazione di finezza (ossia dalla relazione di inclusione in \(\pow {\pow X}\)).
\end{hDigression}
\subsection{Funtori come diagrammi}
In \ref{es_fun_repre} abbiamo visto che una \emph{rappresentazione} di un monoide \(M\) consiste di un funtore \(\susp M \fun\ctSet\), e più in generale un funtore \(\ctC\fun\ctSet\) può essere pensato come una rappresentazione di \(\ctC\); se una categoria \(\ctJ\) è pensata come una \emph{forma} nel senso di \ref{ssec:categorie_forme}, e soprattutto quando è una forma generica nel senso di \ref{ex_cat_freccia}, \ref{ex_spancospan}, \ref{es_catena_trans}, possiamo dare la definizione seguente.
\begin{definition}\label{def_diagramma_comm}\index{Diagramma}
	In teoria delle categorie, la parola \emph{diagramma} (di \emph{forma} \(\ctJ\)) è sinonimo di \emph{funtore} (di \emph{dominio} \(\ctJ\)).

	Più formalmente, se \(\ctJ\) è una categoria \emph{piccola} e \(\ctC\) una categoria, un diagramma in \(\ctC\) di forma \(\ctJ\) è precisamente un funtore
	\[\begin{tikzcd}
			D : \ctJ \ar[r] & \ctC.
		\end{tikzcd}
	\]
\end{definition}
\begin{remark}\label{tante_cose_sono_diag}\index{Diagramma}
	Come è evidente, gli esempi di diagrammi abbondano tra gli esempi fatti in precedenza: fissata una categoria \(\ctC\), una spanna in \(\ctC\) come in \ref{def_span_e_cospan} è un diagramma di forma \(\Lambda^2_0\) e una cospanna è un diagramma di forma \(\Lambda^2_2\). Un quadrato commutativo in \(\ctC\) è un diagramma di forma \(P[2]\), e un cubo \(n\)-dimensionale (le cui facce commutano tutte) uno di forma \(P[n]\). Un singolo oggetto di \(\ctC\) si può vedere come un diagramma di forma \(\Delta[0]=\ctTerm\), e più in generale un insieme di cardinalità \(|A|\) di oggetti di \(\ctC\) è un diagramma di forma \(A^\delta\), come in \ref{ex_cat_discreta}; una singola freccia \(f : X\to Y\) in \(\ctC\) si può vedere come un diagramma di tipo \(\Delta[1] = \genArrow\), e più in generale una catena
	\[\begin{tikzcd}
			A_0 \ar[r, "f_1"] & A_1 \ar[r, "f_2"] & \dots \ar[r] & A_{n-1} \ar[r, "f_n"] & A_n
		\end{tikzcd}
	\]
	di morfismi contigui in \(\ctC\) è un diagramma di tipo \(\Delta[n]\).

	Questi esempi giustificano i nomi che abbiamo dato in \ref{ex_cat_freccia}, \ref{ex_cat_doppiafreccia}, \ref{ex_spancospan}, perché le categorie \(\Lambda^2_0,\Lambda^2_2,\Delta[n], P[n], \dots\)  sono astrazioni o `forme generiche' per diagrammi di un certo tipo: ecco che allora possiamo adottare un lieve abuso di notazione, nel chiamare, per una qualche categoria \(\ctC\) fissata, \emph{cospanna in \(\ctC\)} un diagramma della forma
	\[\dmFun C{\Lambda^2_2}\ctC\]
	(e dualmente una spanna in \(\ctC\)) o catena (infinita, di lunghezza \(\alpha\)) un diagramma della forma
	\[\dmFun K\alpha\ctC\]
	se \(\alpha\) è un numero ordinale, guardato come categoria rispetto alla sua relazione d'ordine, e così via per ogni altro genere di forma. Questo abuso di notazione è, ovviamente, molto conveniente quando ci si debba riferire a insiemi di oggetti e morfismi di \(\ctC\) connessi a una certa maniera.
\end{remark}

\begin{remark}\index{Diagramma}
	Il termine `diagramma' è di solito riservato a funtori il cui dominio è una categoria piccola, e il cui codominio può essere una categoria larga, per suggerire il principio generale cui abbiamo già accennato in \ref{ssec:categorie_forme}, ossia la tensione tra categorie come forme e come universi: un diagramma di forma piccola \(\ctJ\) è `interpretato' in una categoria \(\ctC\), possibilmente larga, come un grafo connesso e diretto alla stessa maniera di \(\ctJ\). Possiamo quindi accedere a una porzione piccola di \(\ctC\) e operare su di essa i ragionamenti che faremmo in \(\ctJ\), \emph{specializzati ai morfismi di \(\ctC\)}.
\end{remark}
La nozione di diagramma permette di formalizzare l'idea di diagramma commutativo (in tutte le sue parti) come un insieme di condizioni di commutatività. L'idea informale è ovviamente sufficiente a operare e ragionale proficuamente con i diagrammi: questa osservazione è pensata per chi desidera una spiegazione più tecnica.
\begin{hRemark}[Cos'è esattamente un diagramma commutativo?]{tech}\label{cos_diag_comm}\index{Diagramma!--- commutativo}
	L'idea intuitiva di diagramma commutativo è semplice da afferrare graficamente, e difficile da formalizzare:
	\begin{itemize}
		\item guardando una categoria come un multidigrafo (\ref{ex_cat_grafi}) su cui è definita una legge di composizione associativa e unitale, un diagramma commutativo in \(\ctC\) consiste di una uguaglianza tra due cammini di frecce componibili di \(\ctC\);
		\item in termini della formulazione `senza oggetti' degli assiomi di categoria (\ref{jgdasoh_1}), un diagramma commutativo in \(\ctC\) consiste di una equazione valida nel magma parziale definito dalla categoria stessa.
	\end{itemize}
	Il secondo punto di vista è però un po' oscuro, e il primo è complicato da formalizzare appieno (scopo che questa osservazione si prefigge).

	Diciamo che un \emph{diagramma libero} in \(\ctC\) è un funtore \(\Phi : \bfF\ctJ \fun\ctC\) (si veda \ref{ex_cat_libera}); un diagramma libero è \emph{commutativo} se \(\Phi\) ammette una fattorizzazione \([\Phi_0,q]\)
	\[\xymatrix{
		\bfF\ctJ \ar[r]^-q & P \ar[r]^-{\Phi_0} & \ctC
		}\]
	dove \(q : \bfF\ctJ \fun P\) è un funtore verso un insieme preordinato.

	Questa definizione è pensata per formalizzare l'idea che un diagramma sia un certo grafo con composizione, e che il diagramma sia `commutativo' se, comunque sia data una coppia di vertici \(X,Y\) in \(\bfF\ctJ\), e due cammini
	\[\xymatrix@R=0cm{
		\vec a : X \ar[r]^-{a_0} & A_1 \ar[r]^-{a_1} & A_2 \ar[r]^-{a_2}  & \dots \ar[r]^-{a_3} & A_n \ar[r]^-{a_n} & Y\\
		\vec b : X \ar[r]_-{b_0} & B_1 \ar[r]_-{b_1} & B_2 \ar[r]_-{b_2} & \dots \ar[r]_-{b_3} & B_m \ar[r]_-{b_m} & Y
		}\]
	di dominio \(X\) e codominio \(Y\) in \(\bfF\ctJ\), applicare \(\Phi\) e comporre il risultato in \(\ctC\) dia lo stesso risultato su \(\vec a\) e su \(\vec b\). Con lieve abuso della notazione, ma in una formulazione facile da ricordare,
	\begin{quote}
		un diagramma commutativo è un funtore \(\Phi : \ctJ \fun\ctD\) tale che per ogni coppia di morfismi paralleli \(\vec a,\vec b\) del suo dominio, \(\Phi a = \Phi b\) in \(\ctC\).
	\end{quote}

	In un insieme preordinato, guardato come categoria al modo di \ref{ord_sonocat}, non esistono due frecce distinte aventi lo stesso dominio e codominio, quindi \emph{tutti} i diagrammi sono commutativi; questa proprietà dei preordini è recuperata per costruzione dalla definizione appena data.
\end{hRemark}
\begin{lemma}\label{lem_funtori_preservano_comm}\index{Diagramma commutativo}
	Con una definizione precisa di diagramma commutativo, si può mostrare formalmente la seguente desiderabile ovvietà: ogni funtore \(F : \ctC \to\ctD\) manda diagrammi commutativi di \(\ctC\) in diagrammi commutativi di \(\ctD\) (della stessa forma, a patto di considerare `degeneri' i lati \(\vec a\) del diagramma \(F(\Phi)\) tali che \(F(\vec a) = a_n\cmp a_{n-1}\cmp\dots\cmp a_0\) sia l'identità di \(\ctD\)); infatti, la fattorizzazione \([\Phi_0,q]\) in
	\[\xymatrix{
		\bfF\ctJ\ar[rr]^\Phi\ar[dr]_q && \ctC \ar[r]^-F & \ctD \\
		& P\ar[ur]_-{\Phi_0}
		}\]
	induce, per composizione, la fattorizzazione \([F\Phi_0,q]\).
\end{lemma}
\begin{remark}[Sull'immagine di un funtore]\label{im_non_subcat}\index{Funtore!immagine di un ---}\index{Immagine}
	L'immagine (stretta) di un funtore \(F : \ctC\fun\ctD\) può non essere una sottocategoria nel senso di \ref{def_subcat}; un controesempio minimo è il seguente: il funtore \(F\) definito da \(Ff=v,Fg=u\) in
	% https://q.uiver.app/#q=WzAsNyxbMCwwLCJYIl0sWzEsMCwiWSJdLFsxLDEsIloiXSxbMiwxLCJXIl0sWzAsNSwiQSJdLFsxLDUsIkIiXSxbMiw1LCJDIl0sWzAsMSwiZiJdLFsyLDMsImciXSxbNCw1LCJ2Il0sWzUsNiwidSJdLFs0LDYsInVcXGNtcCB2IiwyLHsiY3VydmUiOjJ9XSxbMiw1LCJGIiwwLHsic2hvcnRlbiI6eyJzb3VyY2UiOjMwLCJ0YXJnZXQiOjQwfSwic3R5bGUiOnsidGFpbCI6eyJuYW1lIjoibWFwcyB0byJ9fX1dXQ==
	\[\begin{tikzcd}[row sep=tiny]
			X & Y \\
			& Z \ar[dddd, maps to, "F"]& W \\
			\\
			\\
			\\
			A & B & C
			\arrow["f", from=1-1, to=1-2]
			\arrow["g", from=2-2, to=2-3]
			% \arrow["F", shorten <=22pt, shorten >=30pt, maps to, from=2-2, to=6-2]
			\arrow["v", from=6-1, to=6-2]
			\arrow["{u\cmp v}"', curve={height=12pt}, from=6-1, to=6-3]
			\arrow["u", from=6-2, to=6-3]
		\end{tikzcd}\]
	rende componibili morfismi di \(\ctD\) che non hanno una composizione in \(\ctC\), perché non appartengono al dominio della composizione. Questo mostra che per un funtore, essere iniettivo \emph{sui morfismi} senza esserlo sugli oggetti è una condizione insufficiente a renderlo una `immersione' (tale cioè che la sua immagine stretta sia una sottocategoria).
\end{remark}
Per questo si preferisce la nozione di \emph{immagine essenziale} di un funtore.
\begin{definition}[Immagine essenziale di \(F\)]\label{def_fun_imess}\index{Immagine essenziale}
	Sia \(F : \ctC\fun\ctD\) un funtore; l'immagine essenziale di \(F\) consiste della più piccola sottocategoria di \(\ctD\) che contiene l'immagine di \(F\) in quanto omomorfismo di multidigrafi.

	Più formalmente, diciamo che una sottocategoria \(\ctA\subseteq\ctD\) è \emph{repleta}\footnote{Dal verbo latino \emph{rĕplĕo}, \emph{colmare} viene \emph{repletum}; si veda Dante, Purg. XXV, 70--72:
		\begin{verse}
			[\dots\unkern] lo motor primo a lui si volge lieto \\
			sovra tant'arte di natura, e spira \\
			spirto novo, di vertù repleto.
		\end{verse}} se per ogni \(A\in\ctA_0\) e ogni isomorfismo \(f : X\to A\), si ha che \(X\in\ctA_0\), e \(f \in \Hom\ctA(X,A)\). L'intersezione di sottocategorie replete è repleta, e l'\emph{immagine essenziale} \(\essim F\) di \(F : \ctC\to\ctD\) risulta da \(\bigcap_{F(\ctC)\subseteq \ctA \text{ repleta}} \ctA\)
\end{definition}
Nel guardare un funtore \(D : \ctJ \fun\ctC\) come una rappresentazione della categoria dominio \(\ctJ\) nel codominio \(\ctC\), è spesso utile restringere l'attenzione al caso in cui \(\ctJ\) è una forma come in \ref{ssec:categorie_forme}: i casi più interessanti sono quelli in cui
\begin{itemize}
	\item \(D\) è un diagramma \emph{discreto}, ovvero ha per dominio una categoria discreta della forma \(A^\delta\), come in \ref{ex_cat_discreta}; \(D\) allora specifica una famiglia di oggetti \(\{C_a\mid a\in A\}\) indicizzata dagli elementi di \(A\);
	\item un diagramma \emph{connesso} ha per dominio una categoria \(\ctJ\) tale che \(\pi_0^\ctCat(\ctJ)\) (si veda \ref{exe_cpt_conn}) abbia un singolo elemento; per diagrammi del genere vale un risultato di `continuità': se \(D\) è un diagramma connesso, allora la sottocategoria \(\essim D\) di \(\ctC\) è connessa (si confronti questo con l'analogo enunciato in topologia: se \(f : X\to Y\) è una funzione continua, l'immagine \(f(C)\) di qualsiasi sottoinsieme connesso di \(X\) è	connessa in \(Y\));
	\item un diagramma \emph{in catena} ha per dominio un ordinale come in \ref{es_catena_trans}, guardato come categoria; un caso molto importante di diagrammi in catena si incontra in algebra astratta o algebra lineare: sia \(k\) un campo e denotiamo \(k[X]_{\le n}\)	l'insieme dei polinomi di grado al più \(n\) in una variabile \(X\) a coefficienti in \(k\); allora possiamo considerare il diagramma in catena di dominio \(\omega=\{0\le1\le 2\le\dots\le n\le n+1\le\dots\}\) e codominio la categoria \(\ctVect\) di \ref{varie_categorie_nella_pratica}, che ha per immagine
	      \[
		      k = k[X]_{\le 0} \subseteq
		      k[X]_{\le 1} \subseteq
		      k[X]_{\le 2} \subseteq \dots
		      \subseteq k[X]_{\le n} \subseteq k[X]_{\le n+1} \subseteq \dots
	      \]
	      evidentemente l'intero anello dei polinomi \(k[X]\) risulta come \(\bigcup_{n\ge 0} k[X]_{\le n}\) (e le operazioni di somma e prodotto di polinomi sono, in un senso opportuno, compatibili con questa filtrazione).
\end{itemize}
Una classe importante che generalizza i diagrammi in catena è fatta dai diagrammi \emph{filtrati}, che hanno per dominio quella che si dice una \emph{categoria filtrata}:
\begin{definition}[Categoria filtrata]\label{def_cat_filtrata}\index{Categoria!--- filtrata}
	Una categoria non vuota \(\ctJ\) si dice \emph{filtrata} se
	\begin{enumtag}{cf}
		\item \label{cf_1} per ogni \(n\)-upla finita di oggetti \(\tup Jn,\) esiste un oggetto \(J^*\) e frecce \(\tup kn,\), dove \(k_i : J_i\to J^*\);
		\item \label{cf_2} per ogni \(n\)-upla di morfismi paralleli \(\tup un, : J\to J'\), esiste una freccia \(k : J'\to K\) tale che \(k\cmp u_1 = k\cmp u_2 = \dots = k \cmp u_n\): cioè il diagramma
		\[\xymatrix{
			J \ar@<.75pc>[r]^-{u_1} \ar@{}@<.2pc>[r]|\vdots\ar@<-.75pc>[r]_-{u_n} & J' \ar[r]^-k & K
			}\]
		è commutativo.
	\end{enumtag}
	Le frecce \(k_i\) e l'oggetto \(J^*\) in \ref{cf_1} si dicono un \emph{cocono} per \(\tup Jn,\) (\(J^*\) è la `punta' del cocono e le \(k_i\) le sue `componenti'), e in modo simile \(k\) in \ref{cf_2} si dice un cocono per \(\tup un, : J\to J'\). Allora, le proprietà \ref{cf_1}, \ref{cf_2} si possono rifrasare così:
	\begin{quote}
		Una categoria è filtrata se ogni famiglia finita di oggetti \(\tup Jn,\) e ogni famiglia finita di morfismi \(\tup un, : J\to J'\) ammettono un cocono.
	\end{quote}
	Si osservi che, come conseguenza di \ref{cf_2}, una categoria filtrata \(\ctJ\) ha la seguente \emph{proprietà} (si veda \ref{condizione_ore}) \emph{di Ore}:\footnote{Øystein Ore, già menzionato nella nota a pié di pagina in \ref{def_gruppoide}.}
	\begin{quote}
		Per ogni \(J,X,Y\in\ctJ_0\) e ogni \(u : J\to X\), \(v : J\to Y\) esiste un modo di `completare il quadrato'
		\[\xymatrix{
				J\ar[r]^u\ar[d]_v & X \ar@{.>}[d]^{f}\\
				Y \ar@{.>}[r]_{g} & Z
			}\]
		con due frecce \(X \xto fZ \xot g Y\).
	\end{quote}
\end{definition}
Si può dimostrare (per induzione) che più in generale ogni spanna \(n\)-aria (nel senso di \ref{def_span_e_cospan}: una spanna \(n\)-aria è evidentemente una \(\{\iter n\}\)-spanna) può essere completata a un diagramma della forma
\[\xymatrix@R=1mm{
	& X_1\ar@{.>}[ddr]^{d_1}\\
	& X_2\ar@{.>}[dr]\\
	J \ar[uur]^-{v_1}\ar[ur]_{v_2}\ar[dr]_-{v_n} & \vdots & Z\\
	& X_n\ar@{.>}[ur]_{d_n}
	}\]
tale che tutte le composizioni \(d_i\cmp v_i\) siano uguali a una stessa freccia \(v : J\to Z\).
% \paolo{Ci aggiungerei la categoria degli insiemi finiti, (o degli insiemi finiti e delle funzioni iniettive).}
\begin{examples}\label{contro_esempi_filt}\index{Categoria!--- filtrata}\index{Ordine parziale!--- diretto}
	Esempi di categorie che sono filtrate:
	\begin{itemize}
		\item Alcuni esempi elementari sono i seguenti: la categoria codiscreta \(A^\chi\) su un insieme (non vuoto) è filtrata; la categoria terminale \(\ctTerm\) è filtrata; se due categorie \(\ctI,\ctJ\) sono filtrate, è filtrato il loro prodotto (questo permette di suddividere in due problemi indipendenti la verifica che \(\ctI\times\ctJ\) è filtrata); ogni cono destro della forma \(\ctJ^\rhd\) è una categoria filtrata, e più in generale ogni categoria che ammette un oggetto terminale nel senso di \ref{def_cono_su_C} è filtrata.\footnote{In particolare, lo è la categoria degli insiemi finiti, dove ogni singoletto è un oggetto terminale; si veda il capitolo sui limiti e colimiti per maggiori informazioni.}
		\item la categoria-poset \(\omega=\{0\le 1\le 2\le\dots\}\): la condizione \ref{cf_2} è banalmente ovvia (non esistono due frecce \(u,v\) parallele e distinte), e \ref{cf_1} è vera perché ogni insieme finito \(\{\tup pn,\}\) di numeri naturali è limitato superiormente da \(p^*=\max\{\tup pn,\}\); similmente, ogni ordinale (finito, nel senso di \ref{ex_cat_catena}, o infinito, nel senso di \ref{ex_cat_ordinali}) è una categoria filtrata; più in generale, un preordine \((P,\le)\) è una categoria filtrata se e solo se è un \emph{ordine diretto}: ogni insieme finito di elementi \(\tup xn, \in P\) ha un maggiorante \(\bar x\), cioè \(x_i\le \bar x\) per ogni \(i=\iter n\). La categoria \(\bsDelta\) dei simplessi è filtrata, e più in generale ogni categoria di `forme' come in §\ref{ssec:categorie_forme} che contenga \(\ctTerm\) tra i suoi oggetti è filtrata.
	\end{itemize}
	Esempi di categorie che non sono filtrate:
	\begin{itemize}
		\item L'esempio più semplice di categoria non filtrata è dato da un insieme parzialmente ordinato, dove esiste una coppia di elementi \(a,b\) che non possiede un estremo superiore; l'esempio minimo è quindi l'insieme \(\{0,a,b\}\) ordinato da \(0\le a,b\) (ma \(a,b\) non sono confrontabili, e quindi non esiste un cocono per \(\{a,b\}\)).
		\item Se \(A\) è un insieme con almeno due elementi, la categoria discreta \(A^\delta\) di \ref{ex_cat_discreta} non è filtrata.
		\item La categoria \(\bsDelta_\text{inj}\) dei \emph{semisimplessi}, ossia la sottocategoria ottenuta da \(\bsDelta\) prendendo gli stessi oggetti \([n]\) e solo le cofacce \(\delta_n^i\) per morfismi, non è filtrata: infatti, nessuna funzione \emph{iniettiva} (le sole che sono ottenibili componendo mappe di cofaccia) \([1] \to [n]\) coequalizza la coppia di morfismi \(\delta_1^0,\delta_1^1 : [0]\toto [1]\), che quindi non ammette un cocono.
	\end{itemize}
\end{examples}
\begin{theorem}\label{estensione_filtrata}\index{Funtore!--- iniziale}
	Le seguenti condizioni sono equivalenti:
	\begin{itemize}
		\item \(\ctJ\) è una categoria filtrata;
		\item per ogni diagramma \(D : \ctI \fun\ctJ\) dove \(\ctI\) è una categoria finita, esiste una estensione \(\bar D : \ctI^\rhd \fun \ctJ\) di \(D\) al cono destro di \(\ctI\): cioè, esiste una fattorizzazione
		      \[\xymatrix{
			      \ctI \ar[rr]^D \ar[dr]_-{i^\rhd} && \ctJ \\
			      & \ctI^\rhd\ar[ur]_-{\bar D}
			      }\]
	\end{itemize}
\end{theorem}
\begin{proof}
	La seconda condizione implica la prima, perché è sufficiente prendere \(\ctI=\bkt{n}^\delta\) finita e discreta, e \(\ctI=\{0\toto 1\}\) la freccia generica per ritrovare le condizioni \ref{cf_1}, \ref{cf_2}.

	Viceversa, se \(\ctJ\) è filtrata e \(D : \ctI \fun\ctJ\) un diagramma da una categoria finita, l'insieme finito \(\{\tup Jn,\}=\{DI_1,\dots, DI_n\}\) ammette un cocono \(\pair {k_i}{J^*}\); possiamo porre \(\bar D I := DI\), ma definire \(\bar D (\infty)\) e l'azione di \(\bar D\) sulle frecce è più complesso: non esiste garanzia che i triangoli della forma
	\[\xymatrix@R=3mm{
		**[l] DI=J\ar[dr]\ar[dd]_{D(I\to I')} \\
		&J^* \\
		**[l] DI'=J'\ar[ur]
		}\]
	siano commutativi. Si possono però definire gli oggetti \(B_0 := J^*\), e per ogni fissato \(i=\iter n\), e ogni \(f : I_i \to I'\) l'oggetto \(B_i\) come il cocono della famiglia di frecce parallele
	\[\xymatrix@R=1mm{
		&DI'\ar[dr]&\\
		DI_i = J_i\ar[ur]\ar[r]\ar[dr]& DI''\ar[r]& B_0 \ar@{.>}[r]^-{u_i} & C_i\\
		&DI'''\ar[ur]&
		}\]
	(ce ne sono solo un numero finito, dato che \(\ctI\) è finita, \ref{cat_finita_locfinita}). Le \(u_i\), prese come frecce solide nel seguente diagramma, danno una spanna \(n\)-aria, si veda \ref{ex_spancospan}:
	\[\xymatrix@R=0mm{
		& C_1\ar@{.>}[ddr]^{l_1}\\
		& C_2\ar@{.>}[dr]\\
		B_0 \ar[uur]^-{u_1}\ar[ur]_{u_2}\ar[dr]_-{u_n} & \vdots & B_\infty\\
		& C_n\ar@{.>}[ur]_{l_n}
		}\]
	che ammette un cocono \(\pair{B_\infty}{l_i}\) (le frecce tratteggiate). Poniamo \(\bar D(\infty) := B_\infty\), e a questo punto la maniera in cui abbiamo costruito \(B_\infty\) assicura che \(\bar D\) sia un funtore.
\end{proof}
Una classe di categorie più generale, che contiene le categorie filtrate, è quella delle \emph{categorie setacciate}:
\begin{definition}\label{def_cat_setacciata}\index{Categoria!--- setacciata}
	Una categoria \(\ctJ\) si dice \emph{setacciata} se per ogni coppia di oggetti \(X,Y\in\ctJ\) la categoria delle cospanne di \ref{def_span_e_cospan} è connessa nel senso di \ref{ex_fun_cpt_conn}, \ref{exe_cpt_conn}.
\end{definition}
% \Todo{forse qualcos'altro: prodotto di setacciate è setacciato, etc?}
\begin{theorem}\label{filtr_setacciata}\index{Categoria!--- filtrata}
	Ogni categoria filtrata è setacciata; la categoria \Todo{} è setacciata ma non è filtrata.
\end{theorem}
\begin{proof}
	Se \(\ctJ\) è filtrata soddisfa la proprietà di Ore; allora mostriamo che ogni coppia di cospanne della forma
	\[\xymatrix@R=1mm{
		&Z  & \\
		X \ar[dr]_{f'}\ar[ur]^f&& Y\ar[ul]_g\ar[dl]^{g'}\\
		& Z'
		}\]
	è connessa da un cospan di frecce in \(\ctJ\), cioè esiste un diagramma
	\[\xymatrix@R=7mm{
		&Z\ar[d]^z & \\
		X \ar[r]_x \ar[dr]_{f'}\ar[ur]^f& K & Y\ar[ul]_g\ar[dl]^{g'}\ar[l]^y\\
		& Z'\ar[u]_{z'}
		}\]
	Per la proprietà di Ore esistono quadrati commutativi della forma
	\[\xymatrix{
			X \ar[r]^f\ar[d]_{f'}& Z\ar@{.>}[d]^u & Y \ar[r]^g\ar[d]_{g'}& Z\ar@{.>}[d]^v\\
			Z' \ar@{.>}[r]_{u'} & U & Z' \ar@{.>}[r]_{v'}& V
		}\]
	per certi oggetti \(U,V\) e frecce \(u,v,u',v'\); ripetendo questo argomento riusciamo a costruire un diagramma commutativo
	\[
		\xymatrix{
		&  & Z \ar@{->}[ld]_{u} \ar@{->}[rd]^{v} &  &  \\
		X \ar@/^/@{->}[rru]^{f} \ar@/_/@{->}[rrd]_{g} & U \ar@{->}[r]|-{\ell} & K & V \ar@{->}[l]|-{r} & Y \ar@/_/@{->}[llu]_{f'} \ar@/^/@{->}[lld]^{g'} \\
		&  & Z' \ar@{->}[lu]^{u'} \ar@{->}[ru]_{v'} &  &
		}
	\]
	(servono due applicazioni di \ref{cf_2}, e una applicazione della proprietà di Ore, per scegliere \(K\) in modo che \(\ell,r\) siano tali per cui \(\ell\cmp u = r\cmp v\) e \(\ell\cmp u' = r\cmp v'\) allo stesso tempo); ora basta porre \(z = \ell\cmp u \), \(z' =	\ell\cmp u'\) e \(x = \ell\cmp u\cmp f\), \(y = r\cmp v\cmp g\).
\end{proof}
\begin{remark}[Perché categorie `filtrate' e `setacciate'?]\index{Categoria!--- filtrata}
	L'idea è che una categoria setacciata \(\ctJ\) dispone di un filtro che seleziona un cocono per ogni sottocategoria discreta e finita di \(\ctJ\); un filtro è invece capace, in maniera più forte, di selezionare un cocono per ogni sottocategoria finita di \(\ctJ\), anche non discreta, cosicché un filtro è solo un tipo di setaccio particolarmente sottile. La figura \ref{fig_filtro_setaccio} disegna questa situazione.
\end{remark}
\begin{figure}
	\begin{center}
		\begin{tikzpicture}[
				x=4em, y=4em,
				dot/.style={
						circle,
						fill=#1,
						inner sep=0pt,
						outer sep=2pt,
						minimum size=4pt,
						draw=none,
					},
				microdot/.style={
						circle,
						fill=#1,
						inner sep=0pt,
						outer sep=0pt,
						minimum size=1pt,
						draw=none,
					},
				wrap/.style={
						fill=black!5,
						draw=gray,
						rounded corners,
						inner sep=.5em,
					},
			]

			\def\seedX{4}
			\def\sizeX{5}
			\pgfmathsetseed{\seedX}
			\GimmeBounds{\seedX}{\sizeX}
			\begin{scope}[local bounding box=JJ]
				\pgfmathsetseed{\seedX}
				\foreach \i in {1,...,10} {
						\pgfmathsetmacro{\x}{(rnd-\xmid)/ifthenelse(\xmax==\xmin,1,\xmax-\xmin)}
						\pgfmathsetmacro{\y}{(rnd-\ymid)/ifthenelse(\ymax==\ymin,1,\ymax-\ymin)}
						\path node[dot=gray] (J\i) at (\x,\y) {};
					}
			\end{scope}
			\begin{scope}[on background layer]
				\path node[wrap, fit=(JJ)] (ctJ) {\color{gray}$\ctI$};
			\end{scope}
			\begin{scope}[xshift=2cm,yshift=-2cm]
				% Draw the square outline
				\fill[ibmBlue!50] (0,0) rectangle (.95,1.95) node[above left] {\color{black}filtro};
				\node[right] (cocone) at (2.5, 1) {$J^*$};
				\foreach \i in {1,...,10} {
						\draw[ibmBlue!50] (J\i) -- ($(J\i)!.95!(cocone)$);
					}

				\pgfmathsetmacro{\minDist}{0.1}  % Minimum distance between points
				\pgfmathsetmacro{\maxAttempts}{15} % Maximum attempts per point

				\coordinate (p0) at (0.5, 0.5);
				\fill (p0) circle (0.01);

				\pgfmathsetmacro{\activeCount}{1}
				\pgfmathsetmacro{\pointCount}{1}

				\coordinate (active0) at (0.5, 0.5);
				\def\randomNess{15}
				\foreach \i in {0.04, 0.08, ..., 0.92} {
						\foreach \j in {0.04, 0.08, ..., 1.999} {
								\pgfmathsetmacro{\randX}{\i + (random(-\randomNess,\randomNess)/1000)}
								\pgfmathsetmacro{\randY}{\j + (random(-\randomNess,\randomNess)/1000)}

								\pgfmathparse{(\randX > 0.01) && (\randX < 0.999) && (\randY > 0.01) && (\randY < 1.9)}
								\ifnum\pgfmathresult=1
									% Add random rejection to create more natural distribution
									\pgfmathparse{random(0,100)}
									\ifnum\pgfmathresult>1  % 80% chance to place dot
										\fill[ibmBlue!50!black] (\randX, \randY) circle (0.01);
									\fi
								\fi
							}
					}
			\end{scope}
		\end{tikzpicture}
	\end{center}
	\caption{In una categoria filtrata \(\ctJ\), ogni sottocategoria finita \(\ctI\) ammette un cocono; la categoria \(\ctJ\) dispone di un filtro sottile abbastanza da selezionare un cocono \(J^*\) per \(\ctI\), estendendo un diagramma \(D : \ctI \fun\ctJ\) a \(D^* : \ctI^\rhd \fun \ctJ\) come in \ref{estensione_filtrata}. Nella figura, gli oggetti di \(\ctI\) `percolano' attraverso il filtro e `convergono' alla punta \(J^*\) del cocono.}
	\label{fig_filtro_setaccio}
\end{figure}
\begin{esercizi}
	\item Se \(\ctC\) è una categoria con più di un oggetto, scrivere quali proprietà deve avere un funtore \(\ctC \fun \susp N\); farsi degli esempi scegliendo come \(\ctC\) una delle categorie di \ref{sec_esempi_cats}, e con \(N=(\bbN,+,0)\) il monoide dei numeri naturali.
	\item Dualmente, se \(\ctD\) è una categoria con più di un oggetto, scrivere quali proprietà deve avere un funtore \(\susp M \fun \ctD\); farsi degli esempi scegliendo come \(\ctD\) una delle categorie di \ref{sec_esempi_cats}, e con \(M = (\bbN,+,0)\) il monoide dei numeri naturali.
	\item Scrivere in cosa consiste un prefascio, nel senso di \ref{exa_funtori_da_poset}, quando \(P=\pow X\) è l'insieme delle parti di un insieme \(X\), ordinato dall'inclusione. In cosa consiste un \emph{fascio} \(F : (\pow X)^\op\to\ctSet\), cioè un funtore che soddisfa le due condizioni in \ref{cf_1}, \ref{cf_2}? (Suggerimento: \(\pow X\) è generato dai suoi elementi minimali, ossia \(U\subseteq X\) risulta da un ricoprimento fatto di singoletti, \(\bigcup_{x\in U} \{x\}\).)
	\item \label{gfdpgubai_5} Determinare la categoria comma \((F/G)\) nelle varie combinazioni in cui: \(F\) è il funtore identità; \(G\) è il funtore identità; \(F\) è un funtore costante in un oggetto \(E\in\ctE_0\); \(G\) è un funtore costante in un oggetto \(E\in\ctE_0\). Chi è \((E_1/E_2)\) se \(E_1,E_2\) sono due oggetti, guardati come funtori \(\ctTerm \fun\ctE\) (cf. il primo esercizio di questa sezione)?
	\item Descrivere la categoria degli elementi nei seguenti casi particolari:
	\begin{itemize}
		\item il funtore hom controvariante \(\Hom\ctC(-,C)\) di \ref{ex_hom_funtore}, e il \emph{bifuntore} hom \(\Hom\ctC(-,-)\);
		\item Un prefascio come in \ref{exa_funtori_da_poset}, quando \(P\) è l'insieme ordinato degli aperti di uno spazio topologico \((X,\tau_X)\);
		\item il funtore dimenticante \(U : \ctMon \fun\ctSet\) come in \ref{exa_funtori_liberi_forgetti} che ad ogni monoide associa il suo insieme sottostante.
	\end{itemize}
\end{esercizi}
