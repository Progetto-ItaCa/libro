% \iffalse meta-comment
%
% %%%=[ DRIVER ]================================================================
%
%<*driver>
\documentclass[a4paper,full]{l3doc}

\EnableCrossrefs
\CodelineIndex
\RecordChanges

\begin{document}
\DocInput{\jobname.dtx}
\PrintChanges
\PrintIndex
\end{document}
%</driver>
%
% \fi
%
% \changes{0.1.0}{2025/06/11}{initial version}
%
% \GetFileInfo{\jobname.sty}
%
% \title{^^A
%   The \pkg{\jobname} package^^A
%   \fileinfo^^A
% }
%
% \author{^^A
%   Progetto ItaCa\\^^A
%   \texttt{\href{https://progetto-itaca.github.io/}{progetto-itaca.github.io}}^^A
% }
%
% \date{\fileversion~(\filedate)}
%
% \maketitle
%
% ^^A=[ DOCUMENTATION ]=========================================================
%
% \begin{documentation}
%
% \section{Documentation}
%
% \relax
%
% \end{documentation}
%
% ^^A=[ IMPLEMENTATION ]========================================================
%
% \begin{implementation}
%
% \section{Implementation}
%
% Open the \pkg{DocStrip} guards and set the internal namespace prefix (as per \LaTeX3 \pkg{DocStrip} convention).
%    \begin{macrocode}
%<*package>
%<@@=itaca>
%    \end{macrocode}
%

\NeedsTeXFormat{LaTeX2e}[2020/12/01]
\ProvidesPackage{itaca}
  [lo .sty per il libro di ItaCa]

% TODO:
% - funziona tutto ma c'è molto refactoring da fare:
% - [in progress] la parte in LaTeX3 è vecchia e farraginosa
% - [todo] i nomi delle categorie sono un po' random e andrebbero razionalizzati
% - [todo] le macro vanno raggruppate e separate per tipo
% - [todo] alcune vanno ridefinite con meno hacks, codice più elegante
% - [todo] va tutto documentato meglio, con dei commenti che spiegano come si usano le macro
% - [todo] la testa di freccia per gli epi nei diagrammi va migliorata


% questi vanno chiamati per primi
\RequirePackage{ xparse
               , mathrsfs
               }

\ExplSyntaxOn

\clist_new:N \g_itaca_upper_clist
\clist_gset:Nn \g_itaca_upper_clist { A,B,C,D,E,F,G,H,I,J,K,L,M,N,O,P,Q,R,S,T,U,V,W,X,Y,Z }

\clist_new:N \g_itaca_lower_clist
\clist_gset:Nn \g_itaca_lower_clist { a,b,c,d,e,f,g,h,i,j,k,l,m,n,o,p,q,r,s,t,u,v,w,x,y,z }

\clist_new:N \g_itaca_greek_clist
\clist_gset:Nn \g_itaca_greek_clist { alpha,beta,gamma,delta,epsilon,zeta,eta,theta,iota,kappa,lambda,mu,nu,xi,pi,rho, sigma,tau,upsilon,phi,chi,psi,omega,varphi,varpi,varepsilon,varrho,vartheta,Gamma,Delta,Theta,Lambda,Xi,Pi,Sigma,Upsilon,Phi,Chi,Psi,Omega }

\clist_new:N \g_itaca_both_clist
\clist_concat:NNN \g_itaca_both_clist \g_itaca_upper_clist \g_itaca_lower_clist

\clist_new:N \g_itaca_categories_clist
\clist_gset:Nn \g_itaca_categories_clist { Set, Rel, FRel, Fin, Top, Ab, Gph, dGph, Grp, Cat, CAT, Mod, Rng, Ring, cRing, Mon, Pos, Circ, Meas, Flux, POrd, Pre, Span, Cospan, sSet, gSet, Dyn, Ord, FOrd, Lat, HoTop, Fld, DInt, Ch, Gl, Gpd, Alg, Tw }

% \mathbf{#1} -> \bf#1

\clist_new:N \g_itaca_operators_clist
\clist_gset:Nn \g_itaca_operators_clist { GL, lan, Lan, LAN, lft, Lft, LFT, Nat, ran, Aut, Ran, RAN, Rft, Yan, RFT, rft, op, co, coop, colim, coker, eq, coeq, PSh, sk, Sub, Quot, N }

\cs_new_protected:Npn \itaca_makeabbrev:nnN #1 #2 #3
 {
  \clist_map_inline:Nn { #3 }
   {
    \cs_new_protected:cpn { #2 } { #1 { ##1 } }
   }
 }
\itaca_makeabbrev:nnN { \textbf } { bf#1 } \g_itaca_both_clist
\itaca_makeabbrev:nnN { \mathfrak } { fk#1 } \g_itaca_both_clist
\itaca_makeabbrev:nnN { \boldsymbol } { mnd#1 } \g_itaca_both_clist
\itaca_makeabbrev:nnN { \boldsymbol } { bs#1 } \g_itaca_both_clist
\itaca_makeabbrev:nnN { \textsf } { sf#1 } \g_itaca_both_clist
\itaca_makeabbrev:nnN { \underline } { un#1 } \g_itaca_both_clist

\itaca_makeabbrev:nnN { \mathcal } { ct#1 } \g_itaca_upper_clist
\itaca_makeabbrev:nnN { \mathscr } { ect#1 } \g_itaca_upper_clist
\itaca_makeabbrev:nnN { \mathbb } { ict#1 } \g_itaca_upper_clist
\itaca_makeabbrev:nnN { \mathbb } { bb#1 } \g_itaca_upper_clist
\itaca_makeabbrev:nnN { \mathrm } { ob#1 } \g_itaca_upper_clist

\itaca_makeabbrev:nnN { \mathbf } { ct#1 } \g_itaca_categories_clist
\itaca_makeabbrev:nnN { \mathrm } { #1 } \g_itaca_operators_clist


\cs_new_protected:Npn \itaca_make_macro_abbrev:nnN #1 #2 #3
{
	\clist_map_inline:Nn { #3 }
	{
		\cs_new_protected:cpn { #2 } { #1 { \csname ##1\endcsname } }
	}
}

\itaca_make_macro_abbrev:nnN { \boldsymbol } { bs#1 } \g_itaca_greek_clist
\ExplSyntaxOff

\def\pSet{\partial\ctSet}
\def\ctAlgT{\ctT\emdash\ctAlg}
\def\ctCsAlg{\mathrm{C}^*\emdash\ctAlg}
\def\ctSAlg{\Sigma\emdash\ctAlg}
\NewDocumentCommand{\ctSEAlg}{O{E}}{(\Sigma,#1)\emdash\ctAlg}
\NewDocumentCommand{\ctSEEAlg}{O{E}}{\ctSEAlg[E']}

% imprescindibile yoneda in hiragana, va caricato prima di `inputenc'
\RequirePackage{CJKutf8}
\newcommand{\yon}{\text{\begin{CJK}{UTF8}{min}よ\end{CJK}}}
\newcommand{\coyon}{\reflectbox{\yon}}
\newcommand{\jap}[1]{\text{\begin{CJK}{UTF8}{min}#1\end{CJK}}}

\RequirePackage[utf8]{inputenc}
\RequirePackage[T1]{fontenc}

\RequirePackage[svgnames]{xcolor}
% NOTE: this is "Color Blind Safe Palette of IBM Design Language v1 Color"
\definecolor{ibmBlue}{HTML}{648fff}
\definecolor{ibmPurple}{HTML}{785ef0}
\definecolor{ibmMagenta}{HTML}{dc267f}
\definecolor{ibmOrange}{HTML}{fe6100}
\definecolor{ibmYellow}{HTML}{ffb000}

\NewDocumentCommand{\drawDot}{ O{black} }{%
\begin{tikzpicture}
  \node[circle, fill=#1, radius=2pt, inner sep=0pt, minimum size=4pt, draw=none] at (0,0) {};
\end{tikzpicture}
}
\newcommand{\yellowDot}{ \drawDot[ibmYellow] }
\newcommand{\redDot}{ \drawDot[ibmMagenta] }
\newcommand{\blueDot}{ \drawDot[ibmBlue] }
\newcommand{\purpleDot}{ \drawDot[ibmPurple] }
\newcommand{\orangeDot}{ \drawDot[ibmOrange] }

\RequirePackage[pdftex]{graphicx}
\RequirePackage{ amsfonts
               , amsmath
               , amssymb
               , amsthm
               , enumitem
               , epigraph
               , microtype
               , mathtools
               , nextpage
               , hyphenat
               , todonotes
               , stackrel
               , stmaryrd
               , tikz-cd
               , turnstile
               , xspace
               }

\RequirePackage{tikz}
  \usetikzlibrary{decorations.pathmorphing}
\usetikzlibrary{calc,matrix,arrows,positioning,backgrounds,fit}

% TikZ drawing helper for random point clouds
\NewDocumentCommand{\GimmeBounds}{m m}{
  \pgfmathsetmacro{\xmin}{1}
  \pgfmathsetmacro{\xmax}{0}
  \pgfmathsetmacro{\ymin}{1}
  \pgfmathsetmacro{\ymax}{0}
  \pgfmathsetseed{#1}
  \foreach \i in {1,...,#2} {
      \pgfmathsetmacro{\x}{rnd}
      \pgfmathsetmacro{\y}{rnd}
      \pgfmathparse{min(\x,\xmin)}\global\let\xmin\pgfmathresult
      \pgfmathparse{max(\x,\xmax)}\global\let\xmax\pgfmathresult
      \pgfmathparse{min(\y,\ymin)}\global\let\ymin\pgfmathresult
      \pgfmathparse{max(\y,\ymax)}\global\let\ymax\pgfmathresult
    }
  \pgfmathparse{(\xmin + \xmax) / 2}\global\let\xmid\pgfmathresult
  \pgfmathparse{(\ymin + \ymax) / 2}\global\let\ymid\pgfmathresult
}

\NewDocumentCommand{\PlaceholderFigure}{m}{%
  \begin{figure}[h]
		\begin{center}
			\includegraphics[height=0.3\textwidth]{example-image}
			\caption{#1}
		\end{center}
	\end{figure}%
}

\NewDocumentCommand{\ctVect}{O{\bbF}}{{#1}\emdash\cate{Vect}}
\NewDocumentCommand{\ctMat}{O{\bbF}}{{#1}\emdash\cate{Mat}}

\setlength{\epigraphwidth}{0.5\textwidth}
% classi di frecce
\let\mono\hookrightarrow
\let\epi\twoheadrightarrow
\NewDocumentCommand{\Epi}{ O{\ctSet} }{\mathrm{Epi}(#1)}
\NewDocumentCommand{\Mono}{ O{\ctSet} }{\mathrm{Mono}(#1)}
\def\pow#1{2^{#1}}
\def\Elts#1#2{\nabla_{#1}(#2)}
\DeclareMathOperator*{\essim}{ess\,im}

\usetikzlibrary {arrows.meta}
\pgfkeys{
  cm to/.tip={
    Classical TikZ Rightarrow[
      length = +1.05pt 11,
      width' = +0pt 0.65,
      line width = 0pt 0.9 1,
    ]
  },
  cm double to/.tip = {cm to[sep=+0pt -8.9]cm to},
}
% teoremi
\makeatletter
\def\defthm#1#2{%
  \newtheorem{#1}{#2}[section]%
  \expandafter\def\csname #1autorefname\endcsname{#2}%
  \expandafter\let\csname c@#1\endcsname\c@theorem}
\makeatother

\theoremstyle{definition}
% \theoremstyle{plain}
% \theoremstyle{remark}
% ^ commentato perché Ivan si lamenterà

\numberwithin{equation}{section}
\newtheorem{theorem}{Teorema}[section]
 \defthm{corollary}{Corollario}
 \defthm{proposition}{Proposizione}
 \defthm{lemma}{Lemma}
 \defthm{definition}{Definizione}
 \defthm{notation}{Notazione}
 \defthm{remark}{Osservazione}
 \defthm{conjecture}{Congettura}
 \defthm{axiom}{Assioma}
 \defthm{example}{Esempio}
 \defthm{examples}{Esempi}
 \defthm{exercise}{Esercizio}
 \defthm{exercises}{Esercizi}
 \defthm{counterexample}{Controesempio}
 \defthm{construction}{Costruzione}
 \defthm{warning}{Caveat}
 \defthm{digression}{Digressione}
 \defthm{perspective}{Prospettiva}
 \defthm{discussion}{Discussione}
 \defthm{terminology}{Terminologia}
 \defthm{heuristics}{Euristica}

\newtheorem*{proposition*}{Proposizione}

\RequirePackage{fontawesome}
\ExplSyntaxOn
\NewDocumentCommand{\hands}{m}
{
  \str_case:nnTF { #1 }
  {
    { fund } { \faicon{hand-paper-o} }
    { skip } { \faicon{hand-scissors-o} }
    { tech } { \faicon{hand-spock-o} }
  }{}
  {
    \msg_error:nnn { hands } { invalid-argument } { #1 }
    % Error for invalid argument
  }
}

% Custom error message if the argument is not recognized
\msg_new:nnn { hands } { invalid-argument }
  { The~argument~'#1'~is~invalid.~Valid~arguments~are~"fund",~"skip",~or~"tech".
  }
\ExplSyntaxOff

\newenvironment{hCorollary}[2][]{\begin{corollary}[#1\,|\,\hands{#2}]}{\end{corollary}}
\newenvironment{hProposition}[2][]{\begin{proposition}[#1\,|\,\hands{#2}]}{\end{proposition}}
\newenvironment{hLemma}[2][]{\begin{lemma}[#1\,|\,\hands{#2}]}{\end{lemma}}
\newenvironment{hDefinition}[2][]{\begin{definition}[#1\,|\,\hands{#2}]}{\end{definition}}
\newenvironment{hNotation}[2][]{\begin{notation}[#1\,|\,\hands{#2}]}{\end{notation}}
\newenvironment{hRemark}[2][]{\begin{remark}[#1\,|\,\hands{#2}]}{\end{remark}}
\newenvironment{hConjecture}[2][]{\begin{conjecture}[#1\,|\,\hands{#2}]}{\end{conjecture}}
\newenvironment{hAxiom}[2][]{\begin{axiom}[#1\,|\,\hands{#2}]}{\end{axiom}}
\newenvironment{hExample}[2][]{\begin{example}[#1\,|\,\hands{#2}]}{\end{example}}
\newenvironment{hExamples}[2][]{\begin{examples}[#1\,|\,\hands{#2}]}{\end{examples}}
\newenvironment{hCounterexample}[2][]{\begin{counterexample}[#1\,|\,\hands{#2}]}{\end{counterexample}}
\newenvironment{hConstruction}[2][]{\begin{construction}[#1\,|\,\hands{#2}]}{\end{construction}}
\newenvironment{hWarning}[2][]{\begin{warning}[#1\,|\,\hands{#2}]}{\end{warning}}
\newenvironment{hDigression}[2][]{\begin{digression}[#1\,|\,\hands{#2}]}{\end{digression}}
\newenvironment{hPerspective}[2][]{\begin{perspective}[#1\,|\,\hands{#2}]}{\end{perspective}}
\newenvironment{hDiscussion}[2][]{\begin{discussion}[#1\,|\,\hands{#2}]}{\end{discussion}}
\newenvironment{hTerminology}[2][]{\begin{terminology}[#1\,|\,\hands{#2}]}{\end{terminology}}
\newenvironment{hHeuristics}[2][]{\begin{heuristics}[#1\,|\,\hands{#2}]}{\end{heuristics}}
\newenvironment{hTheorem}[2][]{\begin{theorem}[#1\,|\,\hands{#2}]}{\end{theorem}}

\ProcessOptions\relax

\newcommand{\Hom}[1]{#1}
% NO: \ctC(X,Y)
% \Hom{\ctC}(X,Y)
\newcommand{\varHom}{\hom}
\newcommand{\vvarHom}[1]{\mathrm{Hom}_{#1}}

\def\fun{\longrightarrow}
\let\funto\fun

% funtore costante
\def\const{\text{c}}

\def\aa{\text{a}}
\def\nat{\Rightarrow}
\let\natto\nat
\let\To\nat
\def\implies{\Rightarrow}
\def\whisk{innesto\@\xspace}
\def\Whisk{Innesto\@\xspace}
\def\zzg{zig-zag\@\xspace}

\def\xto#1{\xrightarrow{#1}}
\def\xot#1{\xleftarrow{#1}}
\def\xfun#1{\xlongrightarrow{#1}}
\def\xnat#1{\xRightarrow{#1}}

\def\genArrow{\{0\to 1\}}

% == categorie che hanno un nome
\newcommand{\cate}[1]{\mathbf{#1}}
  \NewDocumentCommand{\ctEuc}{O{\bbR}}{\cate{Euc}^{#1}_*}
  \newcommand{\ctInit}{\cate{0}}
  \newcommand{\ctTerm}{\cate{1}}
  \newcommand{\ctIso}{\cate{I}}

\def\singleton{\{\bullet\}}
\newcommand{\cop}{\coprod}

% #1 #2 : gli oggetti che pulli
% #3 : l'oggetto su cui #1 #2 arrivano
\newcommand{\pull}[3]{#1 \times_{#3} #2}
% X x_Z Y <- \pull XYZ oppure \pull{X}{Y}{Z}

\newcommand{\varpull}[3]{#1 \underset{#3}\times #2}
% X x_{f,g} Y <- \pull X{f,g}Z oppure \pull{X}{Y}{Z}

% #1 #2 : gli oggetti che pushi
% #3 : l'oggetto lungo cui incolli
\newcommand{\push}[3]{#1 +_{#3} #2}

\newcommand{\init}{0}
\newcommand{\term}{1}

\let\toto\rightrightarrows
\NewDocumentCommand{\turk}{ O{S} }{\{0 \toto^#1 1\}}
\newcommand{\id}{\mathrm{id}}
\NewDocumentCommand{\Id}{O{}}{\mathrm{id}_{#1}}
\newcommand{\dom}[1]{d(#1)}
\newcommand{\cod}[1]{c(#1)}
\newcommand{\cmp}{\circ}
\let\vcmp\cmp
\newcommand{\horcomp}{*}
\let\whi\horcomp

\def\op{\text{op}}
\newcommand{\opp}[1]{#1^\op}

\newcommand{\End}[2][]{\mathrm{End}_{#1}(#2)}

% strutture monoidali
\newcommand{\monProd}{\otimes}
\newcommand{\monUnit}{I}
\newcommand{\monSymm}{\sigma}
\newcommand{\associator}{\alpha}
\newcommand{\unitorL}{\lambda}
\newcommand{\unitorR}{\varrho}

% aggiunti
\NewDocumentCommand{\adjunct}{O{} O{}}{\nsststile{#2}{#1}}

% isomorfismi / equivalenze
\newcommand{\iso}{\cong}
\newcommand{\eqv}{\simeq}

\newcommand{\defeq}{\coloneqq}
\newcommand{\emptyList}{(\,)}
\newcommand{\cons}{\mathbin{::}}
\NewDocumentCommand{\tup}{O{1} m m m}{
#2_{#1} #4 \dots #4 #2_{#3}
}
\NewDocumentCommand{\optup}{O{1} m m m}{
#2_{#3} #4 \dots #4 #2_{#1}
}
% #1 : l'indice di partenza
% #2 : la variabile indicizzata
% #3 : l'indice di fine
% #4 : il simbolo infisso ("," "\otimes" "+" etc)

\NewDocumentCommand{\iter}{O{1} O{,} m}{
{#1} #2 \dots #2 {#3}
}

% sospensione di una categoria
\def\susp{\mathcal{B}}
\def\pair#1#2{\langle #1,#2\rangle}

% parentesi e notazione per mappe nel prodotto
\def\bkt#1{\langle #1\rangle}
\def\squig{\rightsquigarrow}
\def\dual#1{#1^\ast}
\def\ddual#1{#1^{\ast\ast}}
\let\pmap\divideontimes


\ExplSyntaxOn
\NewExpandableDocumentCommand{\extract}{O{#2}mm}
 {
  \tl_range:nnn { #3 } { #1 } { #2 }
 }
\ExplSyntaxOff

% un generico `todo`
\newcommand{\Todo}[1]{\todo[inline, caption={todo : \extract[1]{30}{#1}},color=green!20, bordercolor=white]{\color{gray} #1}}

% un box dei commenti dedicato a ciascuno di noi
\NewDocumentCommand{\bla}{ O{\textit{anonimo}} m O{red} }{
  \todo[inline, caption={#1 : \extract[1]{30}{#2}}, color=#3!10, bordercolor=white]{ \color{gray} \textsf{#1} ha detto: #2 }
}

\NewDocumentCommand{\fosco}{ m }{\bla[Fosco]{#1}[red]}
\NewDocumentCommand{\paolo}{ m }{\bla[Paolo]{#1}[green]}
\NewDocumentCommand{\beppe}{ m }{\bla[Beppe]{#1}[blue]}
\NewDocumentCommand{\ivan}{ m }{\bla[Ivan]{#1}[yellow]}
\NewDocumentCommand{\enricoV}{ m }{\bla[Enrico V.]{#1}[black]}
\NewDocumentCommand{\enricoG}{ m }{\bla[Enrico G.]{#1}[cyan]}
% ... poi aggiungerò gli altri.

\DeclareDocumentEnvironment{enumtag}{m}{%
	\begin{enumerate}[ label = \textsc{#1}\oldstylenums{\arabic*}), ref = \textsc{#1}\oldstylenums{\arabic*} ] }{ \end{enumerate} }

\DeclareDocumentEnvironment{esercizi}{ O{E.\thesection.\arabic*} }%
  { \subsubsection*{Esercizi della sezione \thesection} \begin{enumerate}[ label = #1 , ref = #1 ] }
  { \end{enumerate} }


\newcommand{\emdash}{\text{-}}
\setlist[itemize,enumerate]{itemsep=0pt}

\def\pto{\rightarrowtriangle}
\DeclareMathOperator{\im}{im}

\def\longmor#1#2{\begin{tikzcd}[ampersand replacement=\&] #1 \ar[r] \& #2 \end{tikzcd}}

\newenvironment{xsmallmatrix}[1]
{\renewcommand\thickspace{\kern#1}\smallmatrix}
{\endsmallmatrix}

\NewDocumentCommand{\prevar}{o m m O{\downarrow}}{
	\IfNoValueTF{#1}{
	  \begin{smallmatrix}
	  	#2 \\ % dom
      #4 \\ %  ↓
      #3    % cod
	  \end{smallmatrix}}
	{
		\begin{xsmallmatrix}{0em}
			   & #2 \\ %  dom
			#1 & #4 \\ % f ↓
			   & #3    %  cod
		\end{xsmallmatrix}
	}
}

\newenvironment{smat}{\left(\begin{smallmatrix}}{\end{smallmatrix}\right)}

\ExplSyntaxOn
\NewDocumentCommand{\var}{ o m O{s} m O{\downarrow} }{
	\str_case:nn { #3 } {
    {r}{\left(       \prevar[#1]{#2}{#4}[#5] \right)}
    {s}{\left[       \prevar[#1]{#2}{#4}[#5] \right]}
    {c}{\left\{      \prevar[#1]{#2}{#4}[#5] \right\}}
    {a}{\left\langle \prevar[#1]{#2}{#4}[#5] \right\rangle}
	}}
	\newcommand{\precop}[2]{
		\begin{smallmatrix}
			#1 \\ #2
		\end{smallmatrix}
	}
	\NewDocumentCommand{\copmap}{O{r} m m}{
		\str_case:nn { #1 } {
			{r}{\left(       \precop{#2}{#3} \right)}
			{s}{\left[       \precop{#2}{#3} \right]}
			{c}{\left\{      \precop{#2}{#3} \right\}}
			{a}{\left\langle \precop{#2}{#3} \right\rangle}
		}}

\ExplSyntaxOff

\RequirePackage{adjustbox}
\newenvironment{adju}[1][0.925]{%
\begin{center}%
	\begin{adjustbox}{max height=0.5\textheight, max width=#1\textwidth}%
}{%
  \end{adjustbox}\end{center}%
}

\NewDocumentCommand{\lort}{ m }{
  {}^\perp #1
}

\NewDocumentCommand{\rort}{ m }{
  #1^\perp
}

\NewDocumentCommand{\act}{ m }{
  #1/\!\!/#1
}

% [d]isplay[m]ath[Fun]ctor
\NewDocumentCommand{\dmFun}{m m m}{
  \begin{tikzcd}[ampersand replacement=\&]
    #1 : #2 \ar[r] \& #3
  \end{tikzcd}
}

\def\norm#1{\|#1\|}

\NewDocumentCommand{\matrici}{ m m m }{\mathrm{M}_{#1\times #2}(#3)}

\RequirePackage{imakeidx}
\RequirePackage[automark]{scrlayer-scrpage}
\RequirePackage[italian]{babel}
\RequirePackage{microtype,multirow,booktabs}
\RequirePackage{hyperref}
\let\second\relax
\let\parto\rightharpoonup
\RequirePackage{mathabx}
\RequirePackage[color]{showkeys}
\RequirePackage{circuitikz}

\ctikzset{tripoles/thickness=1}
\hypersetup{
    colorlinks=true,
    linkcolor=black,
    filecolor=black,
    urlcolor=black,
    citecolor=black,
    }
\pagestyle{scrheadings}

% KOMA-classes have a weird default for footnotes
\deffootnote{1em}{1em}{\textsuperscript{\thefootnotemark}}
% ... and captions
\renewcommand*{\figureformat}{%
  \figurename~\thefigure%
%  \autodot% DELETED
}
\renewcommand*{\tableformat}{%
  \tablename~\thetable%
%  \autodot% DELETED
}
%==
\NewDocumentCommand{\hammock}{ O{\ctSet} m m }{ #2 \leftrightarroweq_{#1} #3 }
\RequirePackage[font={small}]{caption}
\RequirePackage{datetime}

\NewDocumentCommand{\presh}{ m }{ \Hom{\ctCat}({#1}^\op,\ctSet) }
\NewDocumentCommand{\copresh}{ m }{ \Hom{\ctCat}({#1},\ctSet) }
\NewDocumentCommand{\zentrum}{ }{ \fkz }

\NewDocumentCommand{\almond}{ m m O{\bullet} }{
\node[inner sep=1pt] (C0) at (0,0) {$#3$};
\node[inner sep=1pt] (C1) at (1.5,0) {$#3$};
\draw (C0) -- (C1)
  node[font=\tiny,pos=.5,above] {$#1$}
  node[font=\tiny,pos=.5,below] {$#2$};
\draw (C0) to[bend left=50] (C1);
\draw (C0) to[bend right=50] (C1);
}
\NewDocumentCommand{\almonds}{ m m m m O{\bullet} }{
\node[inner sep=1pt] (C0) at (0,0) {$#5$};
\node[inner sep=1pt] (C1) at (1.5,0) {$#5$};
\node[inner sep=1pt] (C2) at (3,0) {$#5$};
\draw (C0) -- (C1)
  node[font=\tiny,pos=.5,above] {$#1$}
  node[font=\tiny,pos=.5,below] {$#2$};
\draw (C0) to[bend left=50] (C1);
\draw (C0) to[bend right=50] (C1);
%
\draw (C1) -- (C2)
  node[font=\tiny,pos=.5,above] {$#3$}
  node[font=\tiny,pos=.5,below] {$#4$};
\draw (C1) to[bend left=50] (C2);
\draw (C1) to[bend right=50] (C2);
}

% \def\radius{3.5pt}
\def\displayRadius{3.5pt}
\def\textstyleRadius{3.0pt}
\def\scriptstyleRadius{2.0pt}
\def\scriptscriptstyleRadius{1.5pt}

\def\displayDotSize{0.375pt}
\def\textstyleDotSize{0.325pt}
\def\scriptstyleDotSize{0.25pt}
\def\scriptscriptstyleDotSize{0.175pt}

\NewDocumentCommand{\doBlankInner}{ O{black} m m }{%
\begin{tikzpicture}[baseline=(current bounding box.south)]%
\foreach \i in {0,...,15}%
  \fill[#1] (\i * 360/16:#2) circle (#3);%
\end{tikzpicture}%
}

\NewDocumentCommand{\doBlank}{ O{black} }{%
\mathchoice
  {\doBlankInner[#1]{\displayRadius}{\displayDotSize}}
  {\doBlankInner[#1]{\textstyleRadius}{\textstyleDotSize}}
  {\doBlankInner[#1]{\scriptstyleRadius}{\scriptstyleDotSize}}
  {\doBlankInner[#1]{\scriptscriptstyleRadius}{\scriptscriptstyleDotSize}}
}

\NewDocumentCommand{\blank}{ O{black} }{\mathop{\doBlank[#1]}}

\NewDocumentCommand{\natHom}{ O{\ctC} O{\ctD} m m }{
  \Hom{[#1,#2](#3,#4)}
}

\NewDocumentCommand{\drawTuple}{m O{}}{
\foreach \i in {0,...,#1} {
    \node[font=\tiny] (\i#2) at (\i,0) {$\i$};
}
}

\NewDocumentCommand{\drawChain}{m O{}}{
\foreach \i in {0,...,#1} {
    \node[font=\tiny] (\i#2) at (\i,0) {$\i$};
    \ifnum\i<#1
        \node[font=\tiny] at (\i+0.5,0) {$<$};
    \fi
}
}
% inutile riga di controllo
% END OF FILE

%
% Close the \pkg{DocStrip} guards and call it a day.
%    \begin{macrocode}
%</package>
%    \end{macrocode}
%
% \end{implementation}
