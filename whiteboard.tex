\documentclass[ paper=a4
              % , pagesize
              , fontsize=12pt
              , twoside=true
              , bibliography=totoc
              , index=totoc
              , version=last
              ]{scrbook}
\usepackage[all,2cell]{xy}\UseAllTwocells
\usepackage{itaca}
\usepackage{quiver}
\usetikzlibrary{intersections}
\recalctypearea
\makeindex
%%% debug area
\tracinglostchars=3  % Increase verbosity
\errorcontextlines=10 % Show more context around errors
%%%

\begin{document}
\begin{center}
 \begin{tikzpicture}[
   x=4em, y=4em,
   dot/.style={
     circle,
     fill=#1,
     inner sep=0pt,
     outer sep=2pt,
     minimum size=4pt,
     draw=none,
    },
   wrap/.style={
     fill=black!5,
     draw=gray,
     rounded corners,
     inner sep=.5em,
    },
  ]

  \def\seedX{4}
  \def\sizeX{5}
  \pgfmathsetseed{\seedX}
  \GimmeBounds{\seedX}{\sizeX}

  % \foreach \i in {1,...,3} \draw[-latex, gray] (J\i) -- (Jinfty);
  % \foreach \i in {1,...,5} \draw[-latex, gray] (J\i') -- (Jinfty');
  % \draw[-latex] (Jinfty) -- (Jinfty');

  \begin{scope}[local bounding box=JJ]
   \pgfmathsetseed{\seedX}
   \foreach \i in {1,...,10} {
     \pgfmathsetmacro{\x}{(rnd-\xmid)/ifthenelse(\xmax==\xmin,1,\xmax-\xmin)}
     \pgfmathsetmacro{\y}{(rnd-\ymid)/ifthenelse(\ymax==\ymin,1,\ymax-\ymin)}
     \path node[dot=gray] (J\i') at (\x,\y) {};
    }
  \end{scope}

  \begin{scope}[on background layer]
   % \path node[wrap, fit=(UP)] (D) {};
   % \path node[wrap, fit=(DN)] (C) {};
   \path node[wrap, fit=(JJ)] (ctJ) {};
   % \node[left=.25em of D] {$\ctI'$};
   % \node[left=.25em of C] {$\ctI''$};
  \end{scope}
  % Vanishing point
  \coordinate (VP) at (5,3);

  % Fraction of the line to draw (0 = start, 1 = full length)
  \def\fraction{0.3}

  % Base points
  \begin{scope}[yshift=-4cm]

   % Vanishing point (arbitrary)
   \coordinate (VP) at (6,3);

   % Base Y range
   \def\ymin{0}
   \def\ymax{6}
   \clip       (2-.15,\ymin)
   -- ++(33:3)
   -- ++(0,.4*\ymax)
   -- (2-.15,.9*\ymax) -- cycle ;
   % Draw vertical red lines
   \foreach \x in {2,2.15,...,4.75} {
     \draw[densely dotted, name path=vert\x] (\x,\ymin) -- (\x,\ymax);
    }

   % Draw blue perspective lines from bottom-left corner toward VP
   % and intersect with verticals
   % \foreach \y in {0,0.5,...,5.5} {
   %     \path[name path=diag\y] (1,\y) -- (VP);

   % Find intersections with vertical lines
   % \foreach \x in {2,2.25,...,4.75} {
   %     \path[name intersections={of=diag\y and vert\x, by=I\x}];
   %     % \fill[blue] (I\x) circle (0pt); % define point I
   % }

   % Now collect intersections into a line (polyline)
   % \draw[blue]
   %     % manually list the intersections to connect
   %     (2,\y) -- ++(4,0); % will be clipped later
   % }
   % \tikzstyle{reverseclip}=[insert path={(current page.north east) --
   %   (current page.south east) --
   %   (current page.south west) --
   %   (current page.north west) --
   %   (current page.north east)}
   % ]
   % Clip the blue lines to the region between the red lines

   \clip (2-.15,\ymin) rectangle (4.75+.15,\ymax);
   % Redraw blue lines with clipping
   \foreach \y in {0,0.15,...,5.5} {
     \draw[densely dotted] (1,\y) -- (VP);
    }

  \end{scope}
 \end{tikzpicture}
\end{center}
\newpage  

\begin{tikzpicture}[
  x=4em, y=4em,
  dot/.style={
    circle,
    fill=#1,
    inner sep=0pt,
    outer sep=2pt,
    minimum size=4pt,
    draw=none,
   },
  microdot/.style={
    circle,
    fill=#1,
    inner sep=0pt,
    outer sep=0pt,
    minimum size=1pt,
    draw=none,
   },
  wrap/.style={
    fill=black!5,
    draw=gray,
    rounded corners,
    inner sep=.5em,
   },
 ]

 \def\seedX{4}
 \def\sizeX{5}
 \pgfmathsetseed{\seedX}
 \GimmeBounds{\seedX}{\sizeX}
 \begin{scope}[local bounding box=JJ]
  \pgfmathsetseed{\seedX}
  \foreach \i in {1,...,10} {
    \pgfmathsetmacro{\x}{(rnd-\xmid)/ifthenelse(\xmax==\xmin,1,\xmax-\xmin)}
    \pgfmathsetmacro{\y}{(rnd-\ymid)/ifthenelse(\ymax==\ymin,1,\ymax-\ymin)}
    \path node[dot=gray] (J\i) at (\x,\y) {};
   }
 \end{scope}
 \begin{scope}[on background layer]
  \path node[wrap, fit=(JJ)] (ctJ) {\color{gray}$\ctJ$};
 \end{scope}
\begin{scope}[xshift=2cm,yshift=-2cm]
% Draw the square outline
\fill[ibmBlue!50] (0,0) rectangle (.95,1.95) node[above left] {\color{black}filtro};
\node[right] (cocone) at (2.5, 1) {$J^*$};
  \foreach \i in {1,...,10} {
    \draw[ibmBlue!50] (J\i) -- ($(J\i)!.95!(cocone)$);
   }

% Poisson disc sampling parameters
\pgfmathsetmacro{\minDist}{0.1}  % Minimum distance between points
\pgfmathsetmacro{\maxAttempts}{15} % Maximum attempts per point

% Initialize with first point
\coordinate (p0) at (0.5, 0.5);
\fill (p0) circle (0.01);

% Define arrays to store active and final points
\pgfmathsetmacro{\activeCount}{1}
\pgfmathsetmacro{\pointCount}{1}

% Store first point as active
\coordinate (active0) at (0.5, 0.5);
\def\randomNess{15}
% Poisson disc sampling algorithm (simplified version)
% Pre-generate a grid of candidate points with minimum distance constraint
\foreach \i in {0.04, 0.08, ..., 0.92} {
    \foreach \j in {0.04, 0.08, ..., 1.999} {
        % Add some randomness to grid positions
        \pgfmathsetmacro{\randX}{\i + (random(-\randomNess,\randomNess)/1000)}
        \pgfmathsetmacro{\randY}{\j + (random(-\randomNess,\randomNess)/1000)}
        
        % Check if point is within bounds
        \pgfmathparse{(\randX > 0.01) && (\randX < 0.999) && (\randY > 0.01) && (\randY < 1.9)}
        \ifnum\pgfmathresult=1
            % Add random rejection to create more natural distribution
            \pgfmathparse{random(0,100)}
            \ifnum\pgfmathresult>1  % 80% chance to place dot
                \fill (\randX, \randY) circle (0.01);
            \fi
        \fi
    }
}
\end{scope}
\end{tikzpicture}

\end{document}
