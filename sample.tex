% questo documento è praticamente solo una sandbox per testare i comandi e le feature senza rompere il main
% potete anche usarlo come file in cui fare modifiche innocue quando volete testare se git funziona. Come quello che sto facendo io adesso
\documentclass[ paper=a4
              , pagesize
              , fontsize=12pt
              , twoside=true
              , BCOR=5mm
              , DIV=calc
              , bibliography=totoc
              , final
              , version=last
              ]{scrbook}

\usepackage{blindtext}

\usepackage{itaca}


\usepackage[automark]{scrlayer-scrpage}
\pagestyle{scrheadings}
\recalctypearea

\begin{document}
\frontmatter

\title{Titolo}
\author{Autori}
\date{Data}
\publishers{Casa editrice}
\uppertitleback{Dettagli pubblicazione I}
\lowertitleback{Dettagli pubblicazione II}
\dedication{Dedica}

\maketitle

\tableofcontents

\mainmatter


\chapter{}
Gli oggetti iniziale e terminale:
\[\init\quad\term\]
Un morfismo
\[f : A \to B\]
Un funtore
\[F : \ctA \fun \ctB\]
Una t. naturale
\[\eta : F \nat G\]

Una pletora di categorie con un nome:
\[
	\ctSet\quad
	\ctTop\quad
	\ctAb\quad
	\ctGrp\quad
	\ctCat\quad
	\ctMod\quad
	\ctRing\quad
	\ctMon\quad
	\ctPos\quad
	\ctLat\quad
	\ctVect\]
Il colimite
\[\lim_{I\in\ctI} \quad \colim_{I\in\ctI}\]


\[A + B, \coprod A_i, \sum A_i\]

\[\pull{X}{Y}{Z} \quad \push{X}{Y}{Z}\]
\[\varpull{X}{Y}{f,g} \quad \push{X}{Y}{Z}\]

\[F\adjunct[a][b] G\]

Every adjunction generates a monad \(\mndT = (T, \eta,\mu)\).

\(\mndT = (T, \eta,\mu)\) Every\(\mndT = (T, \eta,\mu)\) adjunction generates \(\mndT = (T, \eta,\mu)\)a monad \(\mndT = (T, \eta,\mu)\).
\(\mndT = (T, \eta,\mu)\) Every adjunction generates a monad.

\[f \iso g\qquad f \eqv g\]
\[A \defeq B\]

\[\tup xn,\]
\[\tup xn|\]
\[\tup [2]xn|\]
\[\tup xn\otimes\]
\[\tup {ab}n\otimes\]


\[\iter [2][\le]n\]
Prova Enrico

\begin{definition}
Una categoria $\ctC$ è regolare se ha tutti i limiti finiti e se ha una zia a Bologna.
\end{definition}

\clearpage
\appendix

\fosco{Prova commento}
\ivan{Prova commento}
\beppe{Prova commento}
\enricoV{Prova commento}
\enricoG{Prova commento}
\paolo{Prova commento}
\backmatter


\end{document}

